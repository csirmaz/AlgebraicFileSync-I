%% Reconcilation algorithm
\section{Algorithm for reconciliation}
\label{sec:alg}

In this section we provide an algorithm for reconciliation. The reconciler
algorithm takes sequences leading from the original filesystem to the
replicas (\(S_1, S_2, \ldots S_n\)) and creates sequences of commands for
each replica \(S_1^*, S_2^*, \ldots S_n^*\) which make the
filesystems as close as possible. 
%%%\(S_i^*\) contains all
%%%non--conflicting commands from \(S_1, S_2, \ldots S_n\) which is not
%%%already applied to \(S_i\). 
%%%\begin{notrsi}
%%%We also would like to be sure that if there
%%%are no conflicting commands, then after synchronization, all replicas
%%%should be the same.
%%%\end{notrsi}

%%%Thus, 
When creating the algorithm, we keep in mind the following:\\
\emph{a command 
\(C\in {S}_1\cup{S}_2\cup\ldots\cup{S}_n\) 
should be propagated to replica \(R_i\) iff:}
\begin{itemize}
\item \(C\) is not already applied to \(R_i\)
\item there are no conflicts on command \(C\)
\item there are no conflicts on any command which must precede \(C\) 
\end{itemize}

\medskip
{\small{
Why do we need the last criterion? Consider the following case: in the
original replica we had \(O(\pi)=\cfile{X}\). 
We modified replica \(A\) with the following
commands: \(edit(\pi,\cdir{Y}); create(\pi\cdot p,\cfile{W})\).
Replica \(B\) was modified by the command \(edit(\pi,\cfile{Z})\).
Now \(edit(\pi,\cdir{Y})\) and \(edit(\pi,\cfile{Z})\) are
conflicting commands, so we cannot apply \(edit(\pi,\cdir{Y})\) to replica
\(B\). But that way, we cannot propagate
\(create(\pi\cdot p,\cfile{W})\) to it either.
\remark{However, if \(O(\pi)\) was a directory, it could be.}
}}\medskip

%We also need to determine which commands must precede another
%one. Clearly
A
command \(C_1\) must precede command \(C_2\) iff 
\(\neg(C_1; C_2\eqifnbr C_2; C_1)\), it originally preceded \(C_2\),
and they appear in the same sequence \(S_i\). 
%Note that we define this property only
%if \(C_1\) and \(C_2\) are from the same sequence since otherwise they
%would be conflicting commands.

Now, in order to preserve the order of commands in sequences 
\(S_1, S_2, \ldots S_n\), we define the algorithm as follows:
\begin{notrsi}
\\
\emph{To determine \(S_i^*\):}
\begin{enumerate}
\item Detect conflicting commands between sequences
\item Detect must--follow conflicting commands
\item Omit these commands from all sequences
\item Omit commands in \(S_i\) from all other sequences
\item Define \(S_i^*\) as 
\(S_1; S_2; \ldots S_{i-1}; S_{i+1}; \ldots S_n\)
\end{enumerate}
Or, more formally:
\end{notrsi}
\begin{alltt}
FOR every sequence \(S\sb{i}\)
  FOR each command \(C\in S\sb{i}\)  
    FOR every sequence \(S\sb{j}\)
      IF \(C\) should be propagated to replica \(R\sb{j}\) THEN
        append \(C\) to \(S\sb{j}\sp{*}\).
\end{alltt}

We can get the sequence of conflicting commands using a similar
algorithm. The reconciler produces two sequences for every replica: the
first one (\(S_i^*\)) can be applied to the replica immediately, while
the second one (conflicting commands, \(S_i^C\)) needs the user's or
other algorithms' help to be resolved.

See Appendix~\ref{app:impl} for test result of the implementation of the
algorithm.

\begin{comment}
\subsection{Move and subtree-detection}

In order that we should apply less commands on the filesystem or the user
would have to deal with smaller amount of conflicting commands we should
try to detect operations on whole subtrees (e.g. removing a directory not
by removing all files under it first, but by using one delete command).

Also, to ease the user that no information will be lost if a file has been
moved we should detect the move operation rather than use delete and
create commands. 

Both of these problems can be solved easily.
Subtree--detection by majority voting which means that we should use
an operation on a subtree if the same operation would be applied to the
majority of its descendants.

Move--detection can be done by tracking the contents of files.
We use data--fingerprints, a compressed identification string for each
file to determine whether the contents of a new file 
%which should be created
are the same as that of a file already present in the replica, say 
\(contents_F(\pi)=contents_F(\varphi)\). Suppose that we are about to
create \(F(\pi)\). If (according to the sequence of commands) we should
remove \(F(\varphi)\), we will use \(move(\varphi\to\pi)\). If not, we can
use \(copy\). In both cases, we avoided to transmit a huge amount of data 
between the replicas.
\end{comment}
