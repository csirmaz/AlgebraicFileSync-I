%% Description of implementation
\section{Implementation}
\label{app:impl}

We created an implementation based on the algorithms described in Section
\ref{sec:alg} and Appendix \ref{app:upd}. Its main purpose was to
verify that the algorithms are implementable and they work as we expect
them. The program is written in Perl and it runs under UNIX
systems. It handles two replicas and does not modify the filesystem; it
only detects updates and conflicts among commands. Then it provides the
sequences of commands which should be applied to the replicas.

It has no archives of the filesystem, therefore we must also provide it
the original version of the filesystems. This is because the
implementation was built to examine the algorithms, not to make a complete
synchronizer. It also does not simplify the outcoming sequences with
\(move\) commands or commands on subtrees.

When referring to the contents of directories, it looks at the writable
flags of a directory. That is, the contents of two directories are
different if one of them is writable for the program and the other one is
not.

The brief description of its method is the following:
\begin{enumerate}
\item Creates flat representations of the original filesystem and the
replicas \(O(), A(), B()\).
\item Update--detection: defines the minimal sequences \(S_A, S_B\) for
which \(S_AO=A\) and
\(S_BO=B\).
\item Orders these sequences using the method discussed in 
section \ref{app:upd}.
\item Reconciling: detects conflicting commands in sequences.
\item Provides sequences \(S_A^*\) and \(S_B^*\) for
reconciliation.
\item Provides a list of paths where conflicts occurred.
\end{enumerate}

\subsection{Equivalence of \emph{edit} and \emph{create} commands} 

To be able to determine whether two \(edit\) or \(create\) commands are
equivalent, we introduced a \emph{list of samples} in the program, which
is a list of paths.
When the program detects an \(edit(\pi,X)\) or a
\(create(\pi,X)\) command as an update, it looks for a path \(\gamma\) in
the list for
which \(A(\pi)=A(\gamma)\) (in case it is detecting updates at replica
\(A\)). If it finds such a \(\gamma\), it will attach its number in the
list to the command. If not, it appends \(\pi\) to the list and attaches
the new number.

That way, two \(edit\) or \(create\) commands are equivalent if and only
if they are applied to the same path and they have the same
sample--number.

\subsection{Notation}

The program uses the ordinary \texttt{directory/directory/file} notation
for paths. For commands, it uses:
\begin{center}\begin{tabular}{rcl}
\texttt{EF0/path} &for& \(edit(path,\cfile{X})\) \\
\texttt{ED0/path} &for& \(edit(path,\cdir{X})\) \\
\texttt{CF0/path} &for& \(create(path,\cfile{X})\) \\
\texttt{CD0/path} &for& \(create(path,\cdir{X})\) \\
\texttt{RM/path} &for& \(remove(path)\), 
\end{tabular}\end{center}
where \texttt{0} is the sample--number for which 
\(A(Samplelist(\texttt{0}))=A(\texttt{path})\).

When marking conflicting commands, it uses `\texttt{<->A3}' if the command
conflicts with the 3\({}^{rd}\) command in sequence \(S_A\) and
`\texttt{==>A3}' if the command \(S_A[3]\) must precede the current
command but it already conflicts, so the current command cannot be
applied.

\subsection{Examples}

In this section we provide some examples of how the program runs. In the
first section of the output the program lists the flat representation of
the filesystems (the original and the two replicas). The next section is
the result of the detection of updates and conflicting commands. It lists
the samples and the sequences of commands (\(S_A\) and \(S_B\)). Then
sequences \(S_A^*\) and \(S_B^*\) follow which should be applied to
replicas to propagate non--conflicting commands. In the last section it
lists paths where there were conflicts.

\subsubsection{Example 1}
This example is the case we discussed in section \ref{subsec:pathimplcom}.
\begin{center}
{\small{
\begin{alltt}File Synchronizer 
  by Elod Csirmaz 21.07.2000
#==List of filesystems============
---Original Filesystem------------
FS-O/dir/
FS-O/dir/file
---Replica A----------------------
---Replica B----------------------
FS-B/dir/

#==Update-detection and conflicts=
---Edit Samples-------------------
---Sequence O->A------------------
0:RM/dir/file 
1:RM/dir/ 
---Sequence O->B------------------
0:RM/dir/file 

#==Reconciliation=================
---Sequence A->O'-----------------
---Sequence B->O'-----------------
1:RM/dir/

#==Paths of conflicting commands==
==================================
\end{alltt}
}}
\end{center}
As we can see, a situation like this does not cause conflicting updates.
\subsubsection{Example 2}
This example shows a case when we have conflicts because a conflicting
command must precede another. \texttt{/dir/file} was modified in replica
\(A\).
\begin{center}
{\small{
\begin{alltt}File Synchronizer 
  by Elod Csirmaz 21.07.2000
#==List of filesystems============
---Original Filesystem------------
FS-O/dir/
FS-O/dir/file
---Replica A----------------------
FS-A/dir/
FS-A/dir/file
---Replica B----------------------


#==Update-detection and conflicts=
---Edit Samples-------------------
0:FS-A/dir/file
---Sequence O->A------------------
0:EF0/dir/file <->B0
---Sequence O->B------------------
0:RM/dir/file <->A0
1:RM/dir/ ==>B0

#==Reconciliation=================
---Sequence A->O'-----------------
---Sequence B->O'-----------------
#==Paths of conflicting commands==
dir/
dir/file
==================================
\end{alltt}
}}
\end{center}
Actually, in this case the synchronizer cannot do anything, because all
commands conflicted.
