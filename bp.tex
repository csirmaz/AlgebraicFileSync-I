%% Comparing our definition to B&P
\section{Equivalence of definitions of conflicting updates}
\label{cbp:bp}

In this section we try to find a relation between
a definition for conflicting updates found in \BP~(\cite{BP:whatis}) and
our definition.

\subsection{Detecting conflicts using ``dirtiness''}
\label{subsec:defdirty}

In \BP, we can read about an update--detection method whose result when
applied to filesystems \(O\) and \(A\) is a set
\(dirty_A\) for which the following holds:
\begin{itemize}
\item
\(\pi\not\in dirty_A \implies A(\pi)=O(\pi)\) holds where \(O\) is the
original filesystem (Definition 3.1.1 in \BP).
\item
\(dirty_F\) is up--closed for any filesystem,
that is, if \(\pi\preceq\piesub\) and \(\piesub\in dirty_F\) then \(\pi\in
dirty_F\) (Fact 3.1.3).
\end{itemize}
This set is a \emph{safe estimate} of paths where updates have been
made, that is, it may contain paths where the replica has not
changed.

According to the paper, there's a conflict at path \(\pi\) iff
\(\pi\in dirty_A, dirty_B\) and \(A(\pi)\neq B(\pi)\) and
\((A(\pi)\neq\cdir{X})\vee(B(\pi)\neq\cdir{Y})\).
We write \(conflict(\pi)\) if there are conflicting updates at
path \(\pi\) according to this definition. (Definition 4.1.2)

\subsection{Does a conflict at paths imply that there are
conflicting commands?}
\label{subsec:pathimplcom}

Consider the following case. 
We have the original filesystem:
\[\begin{array}{r@{}l@{}l}
O=\{&root&\mapsto\cdir{X};\\
&root\cdot dir&\mapsto\cdir{Y};\\
&root\cdot dir\cdot file&\mapsto\cfile{Z}\}\end{array}\]
In replica A, we apply the following commands to \(O\): 
\(remove(root\cdot dir\cdot file); remove(root\cdot dir)\).
In replica B, we use
\(remove(root\cdot dir\cdot file)\).
Now, according to the definitions in subsection
\ref{subsec:defdirty}, \(root\cdot dir\cdot file\in dirty_A\) and 
\(root\cdot dir\in dirty_A\). We know that \(root\cdot dir\cdot file\in
dirty_B\) and therefore \(root\cdot dir\in dirty_B\) (\(dirty_B\) is
up--closed). Thus we have a conflicting command at path 
\(root\cdot dir\), because it's dirty in both replicas and 
\(A(root\cdot dir)=\ud\), so one of the filesystem entries at that path is
not a directory.

If we investigate this case using sequences, there's no conflict. Since we
know that 
\(S_A=remove(root\cdot dir\cdot file); remove(root\cdot dir)\) and
\(S_B=remove(root\cdot dir\cdot file)\), we can apply the second
command in \(S_A\) to replica \(B\) because it does not conflict
with any command in \(S_B\). (We think this case is not a conflicting
update, however, some file synchronizers, like \emph{unison}, detect such
an update here.)

Therefore, unfortunately, conflicting paths do not imply 
conflicting commands.

\subsection{A conflict at commands implies a conflict at paths}
\label{subsec:comimplpath}

%Again, consider the following case.
%
%In this example we show problems which emerge when we try to edit
%directories or distinguish between them by their contents (properties).
%Suppose that we have an original filesystem \(O\) with one directory
%\(D\). Then we change the properties of this directory in replica
%\(A\) in
%a different way than in replica \(B\) (for example we modify the
%permissions of it). Clearly path \(D\) is dirty in both
%replicas. But the definition does consider this neither a conflict nor a
%resolvable update according to the specification of the reconciler
%provided in section 4.1.2 in \BP~since both \(A(D)\) and \(B(D)\) are
%directories. In other words, this specification does not refer to such a
%case.
%
%This is why we disallow editing directories or telling them apart 
%by comparing their contents. Therefore we modify or model for the
%proof such that
%\remark{I'm not sure this is enough How to formalize this?}
%the contents of every directories are the same and directories cannot be
%modified, that is, if \(F(\pi)=\cdir{X}\) then 
%\(edit(\pi,\cdir{Y})F=\ud\).

We prove this theorem for two replicas.

We prove that if we have filesystems \(O,A,B\neq\ud\) and we have
the minimal sequences \(S_A\) and \(S_B\) and the dirty--sets
\(dirty_A\) and \(dirty_B\) from the update--detectors then if we have
conflicting commands in the sequences we have dirty--conflicts too.

The fact that we gained the sequences from update--detectors allows us to
assume that there are no superfluous commands in the sequences, e.g. a
command which leaves the filesystem in the same state.

Since we cannot distinguish between directories by their contents in the
model of \BP, we have some preconditions.

Our theorem is true if:
\begin{itemize}
\item commands do not modify directories to directories (that is,
\(O(\pi)=\cdir{X}\implies edit(\pi,\cdir{Y})\not\in\) \(S_A\) or
\(S_B\)) 
\emph{and}
\item all directories have the same contents (that is, for any \(F\) and
\(G\), if
\(F(\pi)\) is a directory and \(G(\gamma)\) is a directory then
\(F(\pi)=G(\gamma)\));
\item and there are no superfluous commands in \(S_A\) and \(S_B\).
\end{itemize} 

First of all, notice that \(C_A(\pi)\confl C_B(\gamma)\) can be true only
if \(\pi\preceq\gamma\) or \(\gamma\preceq\pi\). (Otherwise they
would commute.) Without loss of generality suppose the first
one is true. Now we know: \(C_A(\pi)\confl C_B(\piesub)\), where
\(\piesub=\gamma\).

Our theorem is that 
\[\mbox{\bf if~} C_A(\pi)\confl C_B(\piesub) \mbox{~\bf then~}
conflict(\pi),\]
that is, there's a dirty--conflict at path \(\pi\).

\emph{Proof.} We know that \(C_A(\pi)\in S_A\) therefore 
\(O(\pi)\neq S_AO(\pi)\) where \(O\) is the original filesystem (remember
that we don't have superfluous commands). 
Hence \(\pi\in dirty_A\) according to the definition of this set.

The same holds for \(C_B(\piesub)\) therefore \(\piesub\in dirty_B\). This
implies that \(\pi\in dirty_B\) since \(dirty_B\) is up--closed.

Now we need to show that \(A(\pi)\) or \(B(\pi)\) is not a
directory. Suppose that both of them are directories. That is,
\(A(\pi)=S_AO(\pi)=\cdir{X}\). In other words (\(S_A\) is
minimal) \(C_A(\pi)O(\pi)=\cdir{X}\) since there are no other
command on this path. 
Thus \(C_A(\pi)\) is \(create(\pi,\cdir{X})\) 
or \(edit(\pi,\cdir{X})\), but we can be sure that 
\(O(\pi)\neq\cdir{X}\) because otherwise we would edit (modify) a
directory to a directory, and by assumption we cannot do that.

We also know that \(B(\pi)=S_BO(\pi)=\cdir{X}\) since we don't distinguish
between directories. Since we know that \(O(\pi)\neq\cdir{X}\),
we must have a command on \(\pi\) in \(S_B\), namely \(C_B^*(\pi)\). 
This command can be either \(create(\pi,\cdir{X})\) or
\(edit(\pi,\cdir{X})\) but it is equivalent to \(C_A(\pi)\) since
we must use \(create\) in both cases if \(O(\pi)=\ud\) and \(edit\) if
\(O(\pi)=\cfile{X}\). But then \(C_A(\pi)\in S_B\) since
\(C_A(\pi)=C_B^*(\pi)\in S_B\) which
contradicts the definition of conflicting commands.

We also need to show that \(A(\pi)\neq B(\pi)\). If so, similarly
to the above, \(C_A(\pi)=C_B^*(\pi)\) would hold.

Now we know that every condition of a dirty--conflict holds. Our theorem
is proved.
