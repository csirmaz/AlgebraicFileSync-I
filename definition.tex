%% Definitions: filesystem, path, commands
\section{Definitions}
\label{theorem:def}
\subsection{The filesystem}
We introduce some known notations on filesystems and paths.

A filesystem can be \emph{broken} (\(F=\ud\))
or it can be a function mapping whole paths to their
contents. The filesystem is broken if a command caused an error, for
example deleting a directory with files under it.

A \emph{path} is either empty (/), or a finite sequence of names
separated by \(\cdot\). The concatenation of paths \(\pi\) and
\(\varphi\) can be written as \(\pi\cdot\varphi\).
We write
\(\pi\preceq\gamma\) iff \(\pi\) is a prefix of \(\gamma\), i.e., if
\(\gamma=\pi\cdot\alpha\) for some path \(\alpha\) which might be
empty. We also write
\(\pi\prec\gamma\) if \(\pi\) is a proper prefix of \(\gamma\), that is,
\(\pi\preceq\gamma\) and \(\pi\ne\gamma\). If we refer to the contents of
path \(\pi\) in filesystem \(F\), we write simply \(F(\pi)\).

In the paper, \(\pisub\) always refers to a file or directory under
\(\pi\), that is, \(\pi\prec\pisub\). For \(\piesub\),
\(\pi\preceq\piesub\) holds. The same for the opposite direction:
\(\pisup\prec\pi\) and \(\piesup\preceq\pi\). Paths \(\pi\) and
\(\varphi\) are usually incomparable paths.

In a filesystem, the contents of a path can be broken (\(F(\pi)=\ud\),
if the file or directory does not exist), they can be
a file (\(F(\pi)=\cfile{X}\)) or a directory (\(F(\pi)=\cdir{Y}\)).  
We also write \(X\) or \(Y\) for some unspecified but non-\(\ud\)
contents.
\begin{notrsi} 
Both files and directories can carry additional information
(modtimes, permissions, the bytes in a file, etc.), which we leave
unspecified, simply writing the contents as shown above.
\end{notrsi}

Filesystems are functions and are compared using extensional equality.
Two filesystems \(F_1\) and \(F_2\) are equivalent iff
\((F_1=\ud)\wedge(F_2=\ud)\),
\emph{or} \(\forall\pi: F_1(\pi)=F_2(\pi)\).

We write \(F\{\pi \mapsto X\}\) for the function that is like \(F\),
except it maps \(\pi\) to \(X\).
\begin{notrsi}
So
\[F\{\pi \mapsto X\}(\gamma)=
\begin{cases}
X,&\mbox{if $\pi=\gamma$}\\
F(\gamma), &\mbox{otherwise}
\end{cases}\]
\end{notrsi}

We write \(childless_F(\pi)\) iff \(F(\pi)\) has no descendants,
i.e. \(\forall\gamma: \pi\prec\gamma\implies F(\gamma)=\ud\).

\(parent(\pi)\) denotes the path which immediately precedes \(pi\), that
is, for some name \(\mathbf{p}\), \(parent(\pi)\cdot \mathbf{p}=\pi\).

\(S\) always refers to sequences of commands. \(SF\) is the filesystem
obtained by applying all commands in \(S\) to \(F\). We may also write a
command instead of \(S\). \(S_1;S_2\) refers to the concatenation of
sequences \(S_1\) and \(S_2\).

We have some restrictions on filesystems: all filesystems must satisfy
the \emph{tree property}, that
is, if \(\pi\prec\gamma\) and \(F(\gamma)\neq\ud\) then
\(F(\pi)=\cdir{X}\). It means that every path which has a descendant must
be a directory.

\subsection{Commands}

Now we introduce the basis of our algebra: the commands on filesystems.

At first, we
thought about using four commands: \(create\), \(remove\), \(edit\) and
\(move\), as the most common commands on a filesystem. Later, \(move\)
turned out to be hard to handle in the algebra. We decided to introduce
this command only in the user interface, after running the synchronizer.
We also needed a command to reason about commands that cause errors
(\(break\)). So we took another set of commands; it consists of the
commands \(create\), \(remove\), \(edit\) and \(break\). In appendix
\ref{app:update-move} we provide an argument to justify using a
move--free
algebra.

\begin{notrsi}
All the commands have the following property: if applied to a filesystem,
then it either breaks the system (\(command(\pi,X)F=\ud\)) or 
\(command(\pi,X)F=F\{\pi\mapsto X\}\); that is, if a command does not
break the filesystem, it only affects the filesystem at the path on which
it was applied. 
\end{notrsi}

Now we define the exact behavior of each command.
%We have four commands: \emph{create, edit, remove} and \emph{break.} 
Applied to the broken filesystem, they all have the same effect:  leaving
the filesystem in the broken state. Otherwise, 
the definition of their effect is the following: 

\begin{notrsi}
\begin{itemize}
\item
\(create(\pi,X)\) modifies \(F(\pi)\) to \(X\) iff
\(F(\pi)\) is broken and its parent is a directory; i.e.
\[create(\pi,X)F=
\begin{cases}
F\{\pi\mapsto X\}, &\mbox{iff~} F(\pi)=\ud
\wedge F(parent(\pi))=\cdir{Y}\\
\ud, &\mbox{otherwise.}
\end{cases}\]
\item
The command \(edit\) can change the type of a path (file or
directory), according to its second argument.
Since the tree--property must be preserved, the definition is:
\begin{align*}
edit(\pi,\cdir{X})F&=
\begin{cases}
F\{\pi\mapsto \cdir{X}\}, &\mbox{iff~}
F(\pi)\neq\ud\\
\ud &\mbox{otherwise;}
\end{cases}\\
edit(\pi,\cfile{X})F&=
\begin{cases}
F\{\pi\mapsto\cfile{X}\} &\mbox{iff~}
F(\pi)\neq\ud\wedge childless_F(\pi)\\
\ud &\mbox{otherwise.}
\end{cases}
\end{align*}
\item
\(remove(\pi)\) only removes files or directories without files:
\[remove(\pi)F=
\begin{cases}  
F\{\pi\mapsto\ud\} &\mbox{iff~}
F(\pi)\neq\ud \wedge childless_F(\pi)\\
\ud &\mbox{otherwise.}
\end{cases}\]
\item
\(break\) breaks the filesystem:
\[\begin{array}{r@{}ll}
breakF&{}=\ud &\mbox{for every $F$.}\end{array}\]
\end{itemize}
\end{notrsi}
%
\begin{forrsi}
{\small{
%\begin{align}
\begin{gather*}\begin{array}{r@{}ll}
create(\pi,X)F&{}=F\{\pi\mapsto X\}, 
&\mbox{iff~}(F(\pi)=\ud)\wedge(F(parent(\pi))=\cdir{Y})\\
&{}=\ud, &\mbox{otherwise.}\\
edit(\pi,\cdir{X})F&{}=F\{\pi\mapsto \cdir{X}\}, 
&\mbox{iff~}F(\pi)\neq\ud\\
&{}=\ud &\mbox{otherwise;}\\
edit(\pi,\cfile{X})F&{}=F\{\pi\mapsto\cfile{X}\} 
&\mbox{iff~}(F(\pi)\neq\ud)\wedge(childless_F(\pi))\\  
&{}=\ud &\mbox{otherwise.}\\
remove(\pi)F&{}=F\{\pi\mapsto\ud\} 
&\mbox{iff~}(F(\pi)\neq\ud)\wedge(childless_F(\pi))\\
&{}=\ud &\mbox{otherwise.}\\
breakF&{}=\ud &\mbox{for every $F$.}
\end{array}\end{gather*}
%\end{align}
}}
\end{forrsi}
%%%%%%%%%% TAGS!
We write \(\emptyseq\) for the empty sequence of commands.

