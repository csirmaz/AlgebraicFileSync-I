%% Definition of conflicts
\section{Definition of conflicting commands}
\label{cbp:defconfl}

We define conflicting updates using minimal sequences provided by an
update detector algorithm. More specifically, we define conflicting \emph{pairs
of commands} since only two modifications can interfere with each other.
Later, every command will be marked as conflicting if it is a member of a
conflicting pair.

Let us consider two commands, \(C_A(\pi)\in S_A\) and \(C_B(\gamma)\in S_B\) 
where \(S_A\) is the minimal sequence which
leads the original filesystem \(O\) to replica \(A\) and similarly,
\(S_B\) leads \(O\) to replica \(B\).

Now \(C_A\) and \(C_B\) are conflicting commands iff
\begin{center}
\((C_B\not\in S_A)\wedge(C_A\not\in S_B)\)\\\medskip
\emph{and} one of the following holds:\\\medskip
\begin{minipage}{12cm}
\begin{itemize}
\item \(C_A(\pi); C_B(\gamma) \neqifnbr C_B(\gamma); C_A(\pi)\)
 (they do not commute), \emph{or}
\item \(C_A(\pi); C_B(\gamma) \equiv break\), \emph{or}
\(C_B(\gamma); C_A(\pi) \equiv break\)\\(they are equivalent to \(break\);
from which we know they break every filesystem).
\end{itemize}\end{minipage}
\end{center}

We write \(C_1\confl C_2\) if \(C_1\) and \(C_2\) are conflicting
commands.
\begin{notrsi}
Clearly this relation is symmetrical, i.e.
\(C_1\confl C_2\implies C_2\confl C_1\).
\end{notrsi}

In appendix \ref{cbp:bp} we determine the relation between this
definition and another one found in \BP.

