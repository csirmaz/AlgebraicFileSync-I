%% Why don't we use move; Update-detector algorithm
\section{A solution for update-detection in a move--free algebra}
\label{app:update-move}

% TODO v
\begin{comment}
\subsection{A simple way}
A ``simple way'' means that when determining the commands which lead from
the
original to the present state of the filesystems we do not use commands
like moving a whole subtree or removing a subtree. Commands only allowed
at leaves of the tree of the filesystem, e.g. files or directories
containing no files.

Consider the original state, called \(O\) and the present state called
\(A\). In order to be able to determine the \emph{move} operations, we'd
like to compare the containings of two files in the original and in the
present state. It doesn't mean that we have to store the original
filesystem. This can be achieved by a method called
\emph{data--fingerprint} with which we can compare contents of files
without comparing the whole amount of data stored in them. 

A function called \(origin(\pi)\) gives
us a path for a file in the filesystem \(O\) which has the same contents 
as the file \(A\cdot\pi\), or empty--file (\(\perp\)) if such a file does
not exist. Note that more files may be found in \(O\) with the same
contents.

Another function, \(\#(\pi)\) gives the number of paths \(\varphi\) for
which \(origin(\varphi)=\pi\) holds.

Then to gain a possible sequence of commands, compare \(O\cdot\pi\) and
\(A\cdot\pi\). (For this purpose, we can use timestamps, inode numbers,
or any other data on files provided by the operation system.) 
If they're the same, we
don't need to do anything. If not, we have three cases. First, if the
file \(A\cdot\pi\) has been deleted, we can delete this file iff
\(\#(\pi)=0\). If the file exists in \(A\) it may have an origin or not.
If it has, and \(\#(\pi)=0\), than we can \emph{move} \(origin(\pi)\) to
\(\pi\) iff \(\#(origin(\pi))=1\), or \emph{copy} it if
\(\#(origin(\pi))>1\). In both cases, we should decrease
\(\#(origin(\pi))\) by one.

If we couldn't do anything because \(\#(\pi)\) was greater than \(0\), we
can do one of the followings: wait till it decreases due to another
commands, or, if there's a \emph{loop} of file displacings (for example
\(origin(a)=b\) and \(origin(b)=a\)) than we should rename the origin of
one of the files in order to be able to continue (solving the
example: \emph{move} \(b\to b^\prime\), \(a\to b\) and \(b^\prime\to
a\)).

That way we gained a possible sequence of commands which if we apply
to 
filesystem \(O\) we'll get filesystem \(A\).
\end{comment}
% TODO ^

First, we provide an example which explains why we chose working on
a move--free algebra.

\subsection{Moving and removing subtrees}

When trying to determine the minimal sequence of commands which has been
applied to the original filesystem \(O\) to arrive at its present state
\(A\), we should try to move
or remove subtrees instead of (re)moving all files in them
in order to gain the most simple sequence possible.
We can achieve
this by counting the files under a directory and how many of
them need to be (re)moved. Then by majority voting, if there are enough files to be moved, 
we should move the subtree first and then the remaining files back to their
places.

\subsection{Problems with move}

But, unfortunately, difficulties emerge when using such commands.

First of all, consider the following situation: we have files \(X\cdot1,
X\cdot2, X\cdot3, X\cdot4, X\cdot5, X\cdot6\) in filesystem \(O\). In
replica A, we have moved \(X\cdot1, \ldots X\cdot4\) to \(Y\cdot1, \ldots
Y\cdot4\). In replica B, we have moved only \(X\cdot1, X\cdot2\) under
the new directory \(Y\).

In replica \(A\), without using commands on subtrees, we would
have \emph{move} \(X\cdot1\to Y\cdot1, X\cdot2\to Y\cdot2,
X\cdot3\to Y\cdot3, X\cdot4\to Y\cdot4\). But with subtree--moving,
we will
notice that the sequence \emph{move} \(X\to Y, Y\cdot5\to X\cdot5,
Y\cdot6\to X\cdot6\) is shorter. In replica \(B\), \emph{move}
\(X\cdot1\to Y\cdot1, X\cdot2\to Y\cdot2\) is the shortest sequence.

The next step is to choose commands from each sequence so that they would not
cause conflicting updates when propagated to other replicas.
It is easily to prove that such subsequences of commands
do not exist.
 
We reached the conclusion that it is not always the
shortest sequence that should be used to generate the interleaving; or, in
other words, that we should be able to modify the sequences according to the
algebraic laws before choosing the one used.
% It would make the problem much more
% complicated since we have a lot of possibilities here as well as at
% choosing the subsequences.  
Using \(move\), we also have a 
complicated relationship among the commands when considering which pairs
commute and which do not as \(move\) is the only command
that modifies two paths at the same time.

Therefore, we decided to focus on a \emph{move--free} algebra and delay
detecting possibilities of simplification (e.g. subtree--detection) until
after the update--detection and reconciliation.

\subsection{A simple algorithm for update--detection}
\label{app:upd}

With no \(move\)s and subtree--commands, a simple algorithm can be
used for update--detection. It takes information from the filesystems
and gives a minimal sequence of commands which transforms one into
the other. 

Let \(P_O\) be all the paths in the original
filesystem, and \(P_A\) all the paths in the current state of the replica. 
(We assume \(O,A\ne\ud\).)
For
every \(\pi\in P_O\cup P_A\) add the following command to sequence \(S\):
\begin{itemize}
\item \(create(\pi,X)\) iff \(O(\pi)=\ud\) and \(A(\pi)=X\ne\ud\),
\item \(edit(\pi,X)\) iff \(O(\pi)\ne\ud\) and \(A(\pi)=X\ne\ud\) and
\(X\ne O(\pi)\),
\item \(remove(\pi)\) iff \(O(\pi)\ne\ud\) and \(A(\pi)=\ud\).
\end{itemize}
Then arrange the commands so that
\begin{itemize}
\item all \(edit(\pi,\cdir{X})\) commands precede all other commands, 
\item all \(edit(\pi,\cfile{X})\) commands follow all other commands,
\item \(create(\pi,X)\) commands precede \(create(\pisub,Y)\) commands,
\item \(remove(\pisub,X)\) commands precede \(remove(\pi,Y)\) commands.
\end{itemize}
This can be done by the method discussed below.

First, we move all \(edit(\pi,\cdir{X})\) commands to the front and then
all \(edit(\pi,\cfile{X})\) commands to the end of the sequence. 
Regarding the middle part, since there is only one command on each path, we have a partial
ordering on the set of the commands (see section
\ref{theorem:conc}) determining which command must precede another one.
Two commands can be interchanged freely if they refer to incomparable
paths; otherwise, they will be ordered in one of the following ways:
\begin{itemize}
\item \(create(\pi) \ldots create(\pisub)\)
\item \(remove(\pisub) \ldots remove(\pi)\).
\end{itemize}
That is, every \(create\) must precede all other \(create\)s on its
descendants and all \(remove\)s must precede \(remove\)s on its parents.

This ordering can be represented by a graph. This graph is a tree (or
forest). It is worth remarking that every tree of the forest is formed
from the same commands, \(remove\)s or \(create\)s, since 
we cannot have two different
types of commands at comparable paths. 

Starting at the leaves of the trees, let us move the leaves as close as
possible to their parents to form a row on the left side of the
parent command in \(remove\) trees and on the other side in
\(create\) trees. Then we order the commands alphabetically and create a
\emph{group--command} from them and their parent. This
group--command
will act similarly to a single command as all of its members commute with the
same commands. We continue this method until every tree is
grouped into one group--command, and order these groups again.
This way we gain a complete order of the commands. It is worth noting that
this is a deterministic ordering.

We know that \(SO=A\) since because of the ordering it does not break the
filesystem. Also, \(S\) is a minimal sequence. \emph{Proof.} 
By contradiction.
If \(S'\) is shorter than \(S\), then we have a path \(\varphi\) on which
we have a command in \(S\) but not in \(S'\) since all paths are different
in \(S\). We know that \(SF(\varphi)\ne F(\varphi)\) but 
\(S'F(\varphi)=F(\varphi)\). (Note that we have no superfluous commands
in \(S\) since we generated it using the update detector algorithm.)

