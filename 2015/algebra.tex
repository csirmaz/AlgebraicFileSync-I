
\subsection{Towards and Algebraic System}\label{sec_algebra}

Similarly to \cite{NREC} we can consider creating an algebra over command sequences
which will enable us to draw conclusions about the behaviours of the sequences
independently of filesystems.
This would provide a secondary model of (filesystem) commands above the model a filesystems
defined in this paper. The new model would no longer describe filesystems as such,
but use known relationships between sequences as its starting point.

In this algebra, equivalence ($\equiv$) and extension ($\eqext$) would become relations,
and logical rules involving them
(e.g. $ A\equiv B \Rightarrow S\cc A\cc T\equiv S\cc B\cc T $) would be re-cast as inference rules.
In addition to these, the \namecrefs{ax_separate_commute} listed in \cref{section_axioms}
would become axioms that we accept as true, and from which other true statements can be derived
using the inference rules.
Such a system can allow us to deduce, for example, whether two sequences are equivalent
or one extends the other
without reference to filesystems.

We note, however, that the \namecrefs{ax_separate_commute} to be used as axioms 
specify the relationships between the nodes
the commands act on, which would require a more complicated set of symbols to represent
in the algebra.
To avoid this, we think it will be more fruitful to regard commands (and sequences)
on different nodes as different, and, in effect, have nine times as many commands 
in the algebra as there are nodes in the namespace of the filesystems.
This allows us to reduce the number of types of symbols in our algebra, and regard
the \namecrefs{ax_separate_commute} not as axioms, but as templates for axioms
from which all axioms (for pairs of separate nodes, etc.) can be generated.

We expect that the soundness and completeness of such an algebra can be proven
similarly to the proofs described in \cite{NREC}.
Indeed, results in the current paper can serve as building blocks of such a proof:
the computer program used to check the \namecrefs{ax_separate_commute}
in effect proves the soundness of all axioms,
and \cref{simple_reorder_equiv} proves completeness in a limited sense.
It is also worth noting that the majority of proofs presented in this paper
do not actually refer to specific filesystems or the definition of commands,
but draw on known relationships and the
\namecrefs{ax_separate_commute} in \cref{section_axioms}.
In other words, many proofs are transferable to the algebra that we describe here.

In a more complete algebra from which the correctness and completeness
of reconciliation could also be derived,
one can also include symbols, inference rules and axioms for 
$\works{\seqset{X}}$ and $\worksc{\seqset{X}}{\seqset{Y}}$.
We see proving the soundness and completeness of this extended algebra
an intriguing problem that is worthy of further research.
