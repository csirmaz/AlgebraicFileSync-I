
\section{Introduction}

Synchronization of data structures
is a mechanism behind many services we take for granted today:
accessing and editing calendar events, documents and spreadsheets
on multiple devices, version control systems and
geographically distributed internet and web services that
guarantee a fast and responsive user experience.

In this paper we investigate an operation- or command-based approach
to filesystem synchronization.
Presented with multiple copies (replicas) of the same filesystem that have been independently modified,
we regard the aim of the synchronizer to modify these replicas further
to make them as similar as possible.
The synchronization algorithm we describe
follows the two main steps described by Balasubramaniam and Pierce (\cite{BP}):
update detection, where we identify modifications that have been applied to the replicas
since they diverged,
and reconciliation, where we identify updates that can be propagated to other replicas and do so.
The remaining updates, which could not be propagated, are conflicting updates;
resolving these requires a separate system or human intervention.

In this paper we build on our previous work done 
with Prof. Norman Ramsey
on an algebraic approach to synchronization
(\cite{NREC}),
and add to the theoretical understanding
of synchronization by providing rigorous proofs that the algorithms
we describe work as intended
and offer a solution in our framework that cannot be improved on.

A central problem of command-based synchronizers 
during both update detection and reconciliation
is ordering.
If update detection is based on comparing the original state of the filesystem
to its new state, then we can easily collect a set of updates,
but we need to order these in a way that they could be applied
to the filesystems without causing an error 
(for example, a directory needs to be created before a file can be created under it).
Similarly, during reconciliation when we consider all the updates that
have been applied to the different replicas, we need to find a way of
merging these and creating a global ordering that we can apply
to the filesystems.
In fact, an insight in \cite{NREC} is that
if the ordering changes the effect of two commands (updates),
then under certain circumstances they are not compatible and will give rise to conflicts.
Accordingly, in this paper command-pairs and their properties
play a special role.
Other research also shows the essential nature of ordering:
in IceCube multiple orders are tested to find an acceptable one \cite{KRSD},
or in Bayou reconciling updates happens by redoing them in an acceptable order \cite{TTPDSH}.

In \cref{sec_def} we define a model of filesystems and a set of commands
that we will use to describe updates and modifications.
This is followed by investigating the properties and behaviors
of command pairs in \cref{section_axioms}.
\Cref{sec_update} describes update detection, and it is
in \cref{ordering} that we describe how commands can be ordered
and prove that under some simple conditions all valid orders
have the same effect.

\Cref{sec_rec} defines our reconciliation 
and conflict detection algorithms.
We then proceed to prove that the output of reconciliation
is correct in the sense that 
it can be applied to the replicas without causing an error,
and it is maximal
inasmuch as no further updates can be applied under any circumstances.
In order to be able to do this, we introduce
the concept of \emph{conditional operation} of sequences of commands,
and show that it has a number of highly convenient properties.
We finish by comparing our results to other research
and outlining directions for further work
on the introduced algebraic system.

The approach in the present paper offers an improvement
over results presented in \cite{NREC} in multiple ways.
We introduce a new set of commands that is symmetric and
captures more information about the updates.
This has the effect that reasoning about commands
becomes simpler as there are fewer edge cases,
and more powerful as 
predictions made by the reasoning are more accurate
due to the additional information content.
In fact, the new command set not only simplifies
our reconciliation algorithm, but also makes it maximal.

The previous work also lacked proofs that the update detection
and reconciliation algorithms it presented work as intended,
which we provide in the current paper.
While the results are intuitive, providing rigorous proofs
is far from trivial.
During the process, we define a number of useful concepts
and show their relationships and the properties they possess.
In our view these construct a special algebraic model
that is worthy of interest and of further research on its own.

