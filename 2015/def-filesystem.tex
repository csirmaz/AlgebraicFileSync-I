
\subsection{The Filesystem}

% TODO namespace, data objects / inodes

We model a filesystem using a function $\FS$ with a set of nodes (filesystem paths) $\setn$ as its domain,
and a set of possible contents or values, $\setv$, as its codomain:
\[ \FS : \setn \rightarrow \setv \] 
In our model, $\setn$ 
serves as a namespace or ``skeleton'' for the filesystem, and it
contains all possible nodes, including the ones where there is no file or directory.
To model empty nodes, $\setv$ 
contains a special element ($\empt$) which is the value of $\FS$ at these nodes.
We consider the contents of files, as well as any meta-information of files
and directories (e.g. permissions and other flags) part of the values in $\setv$.


The nodes form a disjoint union of rooted directed trees.
\begin{mydef}
The function $\parent(n)$ returns the parent node of $n$, or
returns $\topnode$ if $n$ is the root of a tree.
\end{mydef}

Every filesystem has a so-called \textbf{tree property}, which means that
if the filesystem is not empty at a node, and the node has a parent,
then there must be a directory at the parent node.

To model this, in $\setv$ we distinguish between 
file contents ($\setf$) and directory contents ($\setd$), that is,
if $\setb = \{\empt\}$ then:
\[ \setv = \setb \cup \setf \cup \setd \]
The tree propety can then be defined as
\begin{align*}
\forall n\in\setn: &\FS(n) \neq \empt \\ % TODO alignment problem
&\quad\wedge \parent(n) \neq \topnode \\
&\Rightarrow \FS(\parent(n)) \in \setd 
\end{align*}
This means that as we move down from the root of a tree of nodes,
the types of values we encounter in the filesystem can only change according to the following
transition diagram:

% Graphic for TeX using PGF
% Creator: Dia v0.97.3
% The following commands are not supported in PSTricks at present
% We define them conditionally, so when they are implemented,
% this pgf file will use them.
\ifx\du\undefined
  \newlength{\du}
\fi
\setlength{\du}{15\unitlength}
\begin{tikzpicture}
\pgftransformxscale{1.000000}
\pgftransformyscale{-1.000000}
\definecolor{dialinecolor}{rgb}{0.000000, 0.000000, 0.000000}
\pgfsetstrokecolor{dialinecolor}
\definecolor{dialinecolor}{rgb}{1.000000, 1.000000, 1.000000}
\pgfsetfillcolor{dialinecolor}
\pgfsetlinewidth{0.075000\du}
\pgfsetdash{}{0pt}
\pgfsetdash{}{0pt}
\pgfsetmiterjoin
\definecolor{dialinecolor}{rgb}{1.000000, 1.000000, 1.000000}
\pgfsetfillcolor{dialinecolor}
\fill (5.000000\du,5.000000\du)--(5.000000\du,7.000000\du)--(7.000000\du,7.000000\du)--(7.000000\du,5.000000\du)--cycle;
\definecolor{dialinecolor}{rgb}{0.000000, 0.000000, 0.000000}
\pgfsetstrokecolor{dialinecolor}
\draw (5.000000\du,5.000000\du)--(5.000000\du,7.000000\du)--(7.000000\du,7.000000\du)--(7.000000\du,5.000000\du)--cycle;
% setfont left to latex
\definecolor{dialinecolor}{rgb}{0.000000, 0.000000, 0.000000}
\pgfsetstrokecolor{dialinecolor}
\node at (6.000000\du,6.391250\du){D};
\pgfsetlinewidth{0.075000\du}
\pgfsetdash{}{0pt}
\pgfsetdash{}{0pt}
\pgfsetmiterjoin
\definecolor{dialinecolor}{rgb}{1.000000, 1.000000, 1.000000}
\pgfsetfillcolor{dialinecolor}
\fill (10.000000\du,5.000000\du)--(10.000000\du,7.000000\du)--(12.000000\du,7.000000\du)--(12.000000\du,5.000000\du)--cycle;
\definecolor{dialinecolor}{rgb}{0.000000, 0.000000, 0.000000}
\pgfsetstrokecolor{dialinecolor}
\draw (10.000000\du,5.000000\du)--(10.000000\du,7.000000\du)--(12.000000\du,7.000000\du)--(12.000000\du,5.000000\du)--cycle;
% setfont left to latex
\definecolor{dialinecolor}{rgb}{0.000000, 0.000000, 0.000000}
\pgfsetstrokecolor{dialinecolor}
\node at (11.000000\du,6.391250\du){F};
\pgfsetlinewidth{0.075000\du}
\pgfsetdash{}{0pt}
\pgfsetdash{}{0pt}
\pgfsetmiterjoin
\definecolor{dialinecolor}{rgb}{1.000000, 1.000000, 1.000000}
\pgfsetfillcolor{dialinecolor}
\fill (15.000000\du,5.000000\du)--(15.000000\du,7.000000\du)--(17.000000\du,7.000000\du)--(17.000000\du,5.000000\du)--cycle;
\definecolor{dialinecolor}{rgb}{0.000000, 0.000000, 0.000000}
\pgfsetstrokecolor{dialinecolor}
\draw (15.000000\du,5.000000\du)--(15.000000\du,7.000000\du)--(17.000000\du,7.000000\du)--(17.000000\du,5.000000\du)--cycle;
% setfont left to latex
\definecolor{dialinecolor}{rgb}{0.000000, 0.000000, 0.000000}
\pgfsetstrokecolor{dialinecolor}
\node at (16.000000\du,6.391250\du){$\empt$};
\pgfsetlinewidth{0.080000\du}
\pgfsetdash{}{0pt}
\pgfsetdash{}{0pt}
\pgfsetbuttcap
{
\definecolor{dialinecolor}{rgb}{0.000000, 0.000000, 0.000000}
\pgfsetfillcolor{dialinecolor}
% was here!!!
\pgfsetarrowsend{latex}
\definecolor{dialinecolor}{rgb}{0.000000, 0.000000, 0.000000}
\pgfsetstrokecolor{dialinecolor}
\draw (7.000000\du,6.000000\du)--(10.000000\du,6.000000\du);
}
\pgfsetlinewidth{0.080000\du}
\pgfsetdash{}{0pt}
\pgfsetdash{}{0pt}
\pgfsetbuttcap
{
\definecolor{dialinecolor}{rgb}{0.000000, 0.000000, 0.000000}
\pgfsetfillcolor{dialinecolor}
% was here!!!
\pgfsetarrowsend{latex}
\definecolor{dialinecolor}{rgb}{0.000000, 0.000000, 0.000000}
\pgfsetstrokecolor{dialinecolor}
\draw (12.000000\du,6.000000\du)--(15.000000\du,6.000000\du);
}
\pgfsetlinewidth{0.080000\du}
\pgfsetdash{}{0pt}
\pgfsetdash{}{0pt}
\pgfsetbuttcap
{
\definecolor{dialinecolor}{rgb}{0.000000, 0.000000, 0.000000}
\pgfsetfillcolor{dialinecolor}
% was here!!!
\pgfsetarrowsend{latex}
\definecolor{dialinecolor}{rgb}{0.000000, 0.000000, 0.000000}
\pgfsetstrokecolor{dialinecolor}
\pgfpathmoveto{\pgfpoint{16.999937\du}{6.999915\du}}
\pgfpatharc{144}{-143}{1.666667\du and 1.666667\du}
\pgfusepath{stroke}
}
\pgfsetlinewidth{0.080000\du}
\pgfsetdash{}{0pt}
\pgfsetdash{}{0pt}
\pgfsetbuttcap
{
\definecolor{dialinecolor}{rgb}{0.000000, 0.000000, 0.000000}
\pgfsetfillcolor{dialinecolor}
% was here!!!
\pgfsetarrowsend{latex}
\definecolor{dialinecolor}{rgb}{0.000000, 0.000000, 0.000000}
\pgfsetstrokecolor{dialinecolor}
\pgfpathmoveto{\pgfpoint{5.000011\du}{5.000014\du}}
\pgfpatharc{324}{37}{1.666667\du and 1.666667\du}
\pgfusepath{stroke}
}
\pgfsetlinewidth{0.080000\du}
\pgfsetdash{}{0pt}
\pgfsetdash{}{0pt}
\pgfsetbuttcap
{
\definecolor{dialinecolor}{rgb}{0.000000, 0.000000, 0.000000}
\pgfsetfillcolor{dialinecolor}
% was here!!!
\pgfsetarrowsend{latex}
\definecolor{dialinecolor}{rgb}{0.000000, 0.000000, 0.000000}
\pgfsetstrokecolor{dialinecolor}
\pgfpathmoveto{\pgfpoint{6.999469\du}{6.999717\du}}
\pgfpatharc{119}{62}{8.500000\du and 8.500000\du}
\pgfusepath{stroke}
}
\end{tikzpicture}


In this paper, $n$, $m$ and $o$ denote nodes in $\setn$.
We use $\valf$ for an arbitrary element in $\setf$, 
and $\vald$ for an arbitrary element in $\setd$,
and by $\valvx$ we mean a value of type $X$ in $\setvx{X}$.



Another simplification we will assume in this paper is regarding directories.
As Bill Zissimopoulos pointed out (\cite{BZ}), we often do not want to consider metadata stored in
directories during synchronization. In other words, the contents of directories are all equal,
which can be modelled by assuming that $|\setd|=1$.
% This also creates a convenient symmetry where there is only a single value
% for each type that can be repeated as we move from a path to its child ($D$ and $\empt$).
