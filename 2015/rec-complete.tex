
\subsection{Reconciliation is Maximal}

% Reconciliation is maximal
% -------------------------
% This is where we're using |D|=1

The reconciliation algorithm defined above is also maximal, that is,
it is not possible to apply any further commands from $A\setminus B$ to $\FS_B$.
To show this, we are going to prove the following \namecref{rec_is_complete}:

\begin{myth}\label{rec_is_complete}
If $A$ and $B$ are simple sequences,
and we select a sequence $S\subset \amb$ so that
it contains a command $\cxynv$ for which $\cxynv\nindep\bma$,
then $B\cc S$ either breaks all filesystems $A$ and $B$ do not break
(that is, $S$ breaks all possible $\FS_B$ replicas),
or it changes a node that $B$ has already changed to a different value,
that is, it overrides a change in $\FS_B$.
\end{myth}
Such an override could occur if a given node was modified differently in
$A$ and $B$ (to different values but the same type), which is clearly
a conflict that should be marked for review
by a human or a different system.
\begin{proof}
The proof is similar to the ones we have seen above.
Without loss of generality, we assume that $\cxynv$ is the first command in $S$
that is not independent of $\bmanp$.
If so, we can split $S$ into $S_0\cc\cxynv\cc S_1$ where $S_0\indep\bma$.
Let us also split $\bmanp$ into $T_0\cc\czwmv\cc T_1$ where $T_1\indep\cxynv$.
From \cref{can_move_intersection} we know that
$B\cc S \equiv \acb\cc\bma\cc S$.
We also know that commands in $S_0$ and $\bmanp$ commute, and so this is equivalent to
$\acb\cc S_0\cc\bma\cc\cxynv\cc S_1$.
Expanding $\bmanp$ and swapping $T_1$ and $\cxynv$ we get
\[ \acb\cc S_0\cc T_0 \cc \czwmv \cc \cxynv\cc T_1\cc S_1. \]

First, we prove that this sequence breaks all filesystems
unless $n=m$.
Let us therefore suppose $n\neq m$ and, to use an inverse proof,
that the sequence works on $\FS$ where $A\FS\neq\fsbroken$.
If so, its initial segment, $\acb\cc S_0\cc T_0 \cc \czwmv \cc \cxynv$ must also work.

We know $\cxynv\nindep\czwmv$, and so from \cref{incomparable_is_independent}, $n\nunrel m$.
As the sequence works, $\czwmv\cc\cxynv$ can only be a construction or destruction pair.
As $\czwmv$ and $T_0$ originate from $\bmanp$ and $B$ is simple, we also know there are
no commands on $m$ in $\acbnp$ or $T_0$,
and that there are no commands on $m$ in $S_0$ as $S_0\indep\bma$.

On the one hand we therefore know that $\FS(m)$ must be of type $Z$,
but on the other, from \cref{ax_directchild_breaks,ax_directparent_breaks},
we also know that $\cxynv$ cannot be applied to $\FS$ without
changing $\FS(m)$ first.
As $A\FS\neq\fsbroken$, this means that $A$ must contain a command on $m$.
Let this command be $\cqrmv$.

Clearly $Q=Z$ (from \cref{worksinputmatch}), and from the construction and destruction
pairs we also see that $R=W$ must hold, where $W=\cchard$ or $W=\ccharb$.
As $|\setd|=1$ and $|\setb|=1$, this also means that $v_R=v_W$ and therefore
$\cqrmv = \czwmv$.
This is a contradiction as $\cqrmv\in A$, but $\czwmv\in\bmanp$.

The only possibility is therefore $n=m$.
As $A$ and $B$ are simple, we therefore know that these are the first commands on $n$,
and so (again from \cref{worksinputmatch}) $Z=X$.
We also know $\valvw\neq\valvy$ as otherwise the two commands would be equal and
would be included in $\acbnp$,
but this means that $\cxynv$, from $A$, overrides a change introduced 
by $\czwnv$, from $B$, which should be treated as a conflict.
\end{proof}

From the proof it is apparent that this result depends on $|\setd|=1$.

