\documentclass[12pt]{article}

\usepackage{alltt}
\usepackage{comment}
\usepackage{multicol}
\usepackage{amsmath}
\usepackage{amsfonts}
\usepackage{amssymb}
\usepackage{setspace}
\usepackage{fullpage}
\usepackage{graphicx}
\usepackage{boxedminipage}
\usepackage{afterpage}

\newcommand{\setv}{\mathcal{V}}
\newcommand{\setf}{\mathcal{V}_F}
\newcommand{\setd}{\mathcal{V}_D}
\newcommand{\setb}{\mathcal{V}_\empt}
\newcommand{\setp}{\mathcal{P}}
\newcommand{\empt}{\bot}
\newcommand{\parent}{\mathtt{parent}}
\newcommand{\FS}{\Phi} % {\mathrm{FS}}

\newcommand{\cbrk}{\mathtt{break}}
\newcommand{\cbb}{\overrightarrow{\empt\empt}}
\newcommand{\cbf}{\overrightarrow{\empt{F}}}
\newcommand{\cbd}{\overrightarrow{\empt{D}}}
\newcommand{\cfb}{\overrightarrow{F\empt}}
\newcommand{\cff}{\overrightarrow{FF}}
\newcommand{\cfd}{\overrightarrow{FD}}
\newcommand{\cdb}{\overrightarrow{D\empt}}
\newcommand{\cdf}{\overrightarrow{DF}}
\newcommand{\cdd}{\overrightarrow{DD}}
\newcommand{\cxy}{\overrightarrow{XY}}

\title{Real-Life Algebraic File Synchronization}

\author{Elod Pal Csirmaz}

\begin{document}
\begin{abstract}
Abstract goes here
\end{abstract}

\section{Definition of a Filesystem}

We model a filesystem using a function $\FS$ with a set of paths $\setp$ as its domain,
and a set of possible contents $\setv$ as its codomain:
\[ \FS : \setp \rightarrow \setv \] 
In our model, $\setp$ contains all possible paths, and $\setv$ contains a special
element, $\empt$, which is the value of $FS$ at paths where there are no files
or directories.
We consider the contents of files, as well as any meta-information on files
and directories (e.g. permissions or flags) part of the values in $\setv$.

Every filesystem has a so-called \textbf{tree property}, which means that
if the filesystem is not empty at a path, and it is not a root path,
then it must have a directory at the path's parent.
This requires that a parent--child relationship is defined over $\setp$,
which we model using the function $\parent$, which returns the parent path
or $\empt$ if the path is a root path:
\[ \parent : \setp \rightarrow \setp \cup \{\empt\} \]

Moreover, in $\setv$ we distinguish between files ($\setf$) and directories ($\setd$), that is,
if $\setb = \{\empt\}$ then:
\[ \setv = \setb \cup \setf \cup \setd \]
The tree propety can then be defined as
\[ \forall \FS : \forall p\in\setp : \FS(p) \neq \empt \wedge \parent(p) \neq \empt \Rightarrow \FS(\parent(p)) \in \setd \]

% diagram
% /> D ---> F ---> BOT <\
% \--/               \--/

In this paper, we write $F$ for an arbitrary element of $\setf$, and $D$ for an arbitrary element
of $\setd$. $V$ is usually a value from $\setv$ and $p$ is a path in $\setp$.

\section{Commands}

Next we need to define instructions or commands on the filesystem about which we will reason
using our algebra.
We will aim to draw conclusions or judgements about commands and sequences of commands
based on the algebra we are aiming to construct.
For example, if we believe that two sequences of commands are equivalent as their effect
is the same, then we aim to be able to derive this from the axioms and inference rules of the
algebra.
As the algebra operates on commands only,
these judgements, by definition, need to hold regardless of the filesystem the commands are applied to.
Accordingly, we expect that the more information one encodes into the commands and the sequences,
the better predicitons we will be able to make using our algebra,
as then the commands and sequences will be more specific, and will select a smaller subset
of potential filesystems on which they can meaningfully operate.

Let us consider what kind of information may be encoded in the commands.
A usual set of commands, based on the most frequent tools implemented by filesystems,
might be $create(p,V)$, $edit(p,V)$ and $remove(p)$ where $p\in\setp$ and $V\in\setv$.
Clearly the commands need to contain information about the state they leave the filesystem
in at the path on which they operate, that is, they need to contain the ``end'' value, $V$.
Notice, however, that the commands above also encode some information about the filesystem
{\it before} the command is applied; namely, $create$ requires that there are no files
or directories at $p$, while $edit$ and $remove$ require the opposite.

We also know that after $create$ or $edit$, $\FS(p)\neq\empt$, whereas after $remove$,
$\FS(p)$ will be $\empt$. However, from \cite{NREC:alg} we know that a useful set of axioms
will need to distinguish between edits that result in directories ($edit(p,D)$) and
ones that result in files ($edit(p,F)$). This creates a seemingly arbitrary asymmetry where
there is more information encoded into commands about their results than about the
original state of the filesystem.

In line with the aim to encode in commands as much information as possible,
and in order to resolve this asymmetry, we propose a set of commands that encode
the type of the original state of $\FS(p)$ as well.
(Some real-life filesystem commands like $mkdir$ or $rmdir$ do this already.)
Please note that there is never any need to encode information about the
{\it exact} value of path $p$ in a command, merely its type ($D$, $F$ or $\empt$),
as the success or failure of subsequent commands only depend on the type of the value.

We therefore propose to have a command for each pair of types.
For want of a better system, we will simply name our commands by concatenating
two of $D$, $F$ and $\empt$ with an arrow, where the left sign notes the type of value
in the filesystem before the command is applied, and the right sign notes the type
afterwards. For example, $mkdir(p,D)$ is $\cbd(p,D)$ and $rmdir(p)$ is $\cdb(p)$.

% No "move"

\section{Applying Commands}

The commands can only succeed if the original value at $p$ has a type that matches
the type required by a command. If this is not the case, or if the resulting
filesystem no longer has the tree property, then we say that the command
\textbf{breaks} the filesystem. Broken filesystems are considered equal
(but not equal to any working filesystem), and we note them by $\FS=\empt$.

So that we could reason about sequences of commands that break every filesystem
in the algebra, we introduce the command $\cbrk$ that simply breaks every filesystem.
We note that a command or a sequence of commands $S$ is applied to a filesystem
by prefixing the command or sequence to it:
$S\FS$; or $\cbrk\FS=\empt$; or $\Xi = \cff(p,F)\FS$, for which $\Xi(p)=F$ holds
and $\Xi(q)=\FS(q)$ for any path $q\neq p$.

\section{Sequences of Commands}



\begin{thebibliography}{99}

\bibitem{NREC:alg} Ramsey, Norman and Elod Csirmaz: {\it An algebraic approach to
file synchronization...}

\end{thebibliography}

\end{document}
