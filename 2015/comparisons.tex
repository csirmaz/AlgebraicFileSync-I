
\section{Comparisons}

\subsection{Liberal and conservative Reconciliation}
We consider the reconciliation algorithm described here an improvement over
the one derved during our previous research (see \cite{NREC})
as the previous algorithm not only fails to propagate all possible commands
wherever it is possible (that is, where it does not break the filesystem),
but the current algorithm is also simpler.
This is because the previous reconciliation algorithm excludes
commands from being propagated which must be preceded by a command that conflicts.

The former observation is supported by Berzan and Ramsey, who in \cite{CBNR} 
describe between different reconciliation policies.
The liberal (maximal) policy propagates all updates to all replicas where
the command does not break the filesystem, while a conservative policy
refrains from updating any node that is below a node with conflicting commands.
They show that the reconciliation algorithm in \cite{NREC} implements
an intermediate policy as one can easily construct two- or three-replica scenarios
where an update could clearly be propagated, but it is excluded.
To rephrase the example in \cite{CBNR}, consider the following two
update sequences that have been applied to replicas $\FS_A$ and $\FS_B$:
\begin{align*}
A&=\cbdaa{\nn}{\valdi}\cc\cbfaa{n}{\valf} \\
B&=\cbdaa{\nn}{\valdii}
\end{align*}
where $\parent(n)=\nn$ (\cite{NREC} did not require all directories to have the same value).
Clearly $\cbfaa{n}{\valf}$ could be applied to $\FS_B$, but it is not as
it must be preceded by creating the directory, which conflicts with the same operation in $B$
due to the different value.
(Introducing a third replica which has not changed at all further complicates the picture.)
The current algorithm no longer needs to specify that there can be no conflicts
on preceding commands, and, to use the above terminology, implements a fully liberal policy.
