
\subsection{The Correctness of Update Detection}

Based on \cref{connected_changes} we can also show that
the update detector does function as intended as:

\begin{myth}\label{update_works}
A sequence of commands returned by the update detector
when comparing the non-broken $\FS^*$ to the original $\FS$
will not break $\FS$ if it is applied to it.
\end{myth}
\begin{proof}
Let $\FS\neq\fsbroken$ be the original filesystem, and $\FS^*\neq\fsbroken$
be the filesystem after the changes we intend to detect.
Let us assume that some mechanism recorded all changes that occurred
to $\FS$ until it reached the state $\FS^*$.
Our set if commands is sufficient to record any change, as
there are commands for every input and output type pair, therefore
we know that the changes can be recorded as a sequence of commands $S$
where $S\FS=\FS^*$.
Note that $S$ will not contain assertion commands, as they do not
represent an actual change in the filesystem.

Therefore, \cref{connected_changes} applies to $S$ and we know
if it contains commands on $\nn$ and $n$ where $\nn\descendant n$,
then it also contains commands on nodes in between;
which is equivalent to saying that if $\FS$ changed
at node $\nn$ and $n$, then it had to change on the nodes in between.

If so, then, similarly to what happens in the ordering algorithm,
we can separate $S$ into components of commands on related nodes,
as commands from separate components freely commute yielding equivalent sequences.
From the arguments above we also know that each component
must be made up of only destruction or only construction pairs,
but not a mixture of the two.
This means that in each component, commands on parents 
either strictly precede or succeed child nodes
(as otherwise $S\equiv\cbrk$ would hold),
from which it follows that once the components are separated
in a new sequence $S'\equiv S$,
commands on the same nodes are next to each other in $S'$.

Based on \cref{ax_same_emptyseq,ax_same_singlec} there is
an $S''\eqnrw S'$ in which multiple commands on the same nodes
are simplified into a single command or an empty sequence,
and therefore $S''$ is simple and $S\equiv S'\eqext S''$.
It is easy to see that $S''$ must contain the same commands
as the output of the update detector, $U$, as both are simple
but reflect all changes between $\FS$ and $\FS^*$.
If so, then $S''\in\orderset{U}$ and based on \cref{simple_reorder_equiv}
we have
$S\equiv S'\eqext S''\equiv U$, and as $S\FS = \FS^* \neq \fsbroken$,
therefore $U\FS\neq\fsbroken$.
\end{proof}
