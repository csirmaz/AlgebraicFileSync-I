%% Algebra, laws, inference rule, description of theorems
\section{Algebra on commands}
\label{theorem:laws}

In order to reason about commands, we define \emph{algebraic laws} on
commands. We use formal logic to build a proof system which is sound and
complete for its intended interpretation.

For sequences \(S_1\) and \(S_2\), we have two kinds of judgements:
\begin{itemize}
\item \(S_1\equiv S_2\), or \emph{\(S_1\) is algebraically equivalent to
\(S_2\)}. Its intended interpretation is that they act the same on all
filesystems, i.e. \(\forall F: S_1F=S_2F\).
\item \(S_1\eqexp S_2\), or \emph{\(S_2\) extends \(S_1\)}; its intended
interpretation is the following: if \(S_1\eqexp S_2\) and 
\(S_1F\ne\ud\) then \(S_1F=S_2F\).
%%%
%%% Transitivity? == => =C ?
%%%
\end{itemize}

%%%% REF!!
The axioms of our proof system are the laws listed in Table 1. We have two
inference rules, one for each judgement.

\subsection{The laws}

We created the laws with the help of pairs of commands.
Lines in the last section are not axioms; they are written
there to list all possible pairs.

\begin{notrsi}
\medskip
{\small{
How many pairs do we have? 
For the first group of laws which contain a pair of commands from
\(edit\), \(create\) and \(remove\), we can choose a pair 9 ways.
We have four types of
path-pairs: \(\pi;\pi\), \(\pisub;\pi\), \(\pi;\pisub\)
and \(\pi;\varphi\). That makes \(4\times9\) cases. With \(break\), we have
\(3\times 2\) cases
with an other non--break command (there are 3 types and they can follow
or precede \(break\)) and one case when we have a pair made of two
\(break\)s. Thus, we have \(4\times 9+3\times2+1=43\) pairs. 
}}
\medskip
\end{notrsi}

We subdivide some laws containing \(edit\) because of \(edit\)'s different
behavior when modifying paths to files or to directories. These laws are
numbered with A or B.
Extension laws (that is, where the relation is \(\eqexp\), not
\(\equiv\)) are marked with \(E\).
%%%
%%% reference
%%%
(See table 1 for the laws.)

\ifnum\Textsize=1%
\def\crlw{\)\\\indent\(}%
\else%
\def\crlw{\relax}%
\fi%
\newcommand{\lawsection}[1]{\par\medskip{\bf{#1}}\medskip\\}%
\def\lawi{1}%
\def\lawip{2}%
\def\lawiib{4A${}_E$}%
\def\lawiibp{3A}% not true semilaw
\def\lawiiia{5A}%
\def\lawiiiap{6A}%
\def\lawiv{7}%
\def\lawv{8}%
\def\lawvi{9}%
\def\lawvp{10}%
\def\lawvii{11}%
\def\lawviii{12}%
\def\lawvip{13}%
\def\lawviiip{14}%
\def\lawix{15}%
\def\lawiia{3B}%
\def\lawiiap{4B}%
\def\lawiiib{5B}%
\def\lawx{16}%
\def\lawxi{17}%
\def\lawxii{18}%
\def\lawxiii{19}%
\def\lawxiv{20}%
\def\lawxv{21}%
\def\lawxvi{22}%
\def\lawxvii{23}%
\def\lawxviii{24}%
\def\lawxix{25}%
\def\lawxx{26}%
\def\lawxxi{27}%
\def\lawxxii{28}%
\def\lawxxiii{29}%
\def\lawxxiv{30${}_E$}%
\def\lawxxv{31}%
\def\lawxxvi{32}%
\def\lawxxvii{33${}_E$}%
\def\lawxxviii{34${}_E$}%
\def\lawxxix{35}%
\def\lawxxx{36}%
\def\lawxxxi{37}%
\def\lawxxxii{38}%
\def\lawxxxiii{39}%
\def\lawxxxiv{40}%
\def\lawxxxv{41}%
\def\lawiiic{6B}%
\def\lawxxxvi{42}%
\def\lawxxxvii{43}%
%%%
\ifnum\Textsize=1
%%\begin{forrsi*}
\afterpage{\clearpage}
\begin{table}[H]
\singlespacing
\else
%%\end{forrsi*}
%%\begin{notrsi*}
\begin{table}[thb]
\fi
%%\end{notrsi*}
\relax
\label{table:laws}
\relax
\framebox{
{\small{
\begin{minipage}{8.3cm}
\setlength{\rightskip}{0cm plus 7cm}
\lawsection{Commuting laws}
\lawi. \(edit(\pi,X); edit(\pisub,Y) \equiv\crlw
edit(\pisub,Y); edit(\pi,X)\)\\
\lawip. \(edit(\pisub,Y); edit(\pi,X) \equiv\crlw
edit(\pi,X); edit(\pisub,Y)\)\\
%2. \(edit(\pi,X); create(\pisub,Y) \equiv
%create(\pisub,Y); edit(\pi,X)\)\\
%%\lawiibp. \(edit(\pi,\cdir{X}); create(\pisub,Y)\eqshr\)\\\indent\(%
%%create(\pisub,Y); edit(\pi,\cdir{X})\)\\
\lawiib. \(create(\pisub,Y); edit(\pi,\cdir{X})\eqexp\)\\\indent\(%
edit(\pi,\cdir{X}); create(\pisub,Y)\)\\
\lawiiia. \(edit(\pi,\cdir{X}); remove(\pisub) \equiv\)\\\indent\(%
remove(\pisub); edit(\pi,\cdir{X})\)\\
\lawiiiap. \(remove(\pisub); edit(\pi,\cdir{X})\equiv\)\\\indent\(%
edit(\pi,\cdir{X}); remove(\pisub)\)\\
\lawiv. \(edit(\pi,X); edit(\varphi,Y) \equiv
edit(\varphi,Y); edit(\pi,X)\)\\
\lawv. \(edit(\pi,X); create(\varphi,Y) \equiv\crlw
create(\varphi,Y); edit(\pi,X)\)\\
\lawvi. \(edit(\pi,X); remove(\varphi) \equiv\crlw
remove(\varphi); edit(\pi,X)\)\\
\lawvp. \(create(\varphi,Y); edit(\pi,X) \equiv\crlw
edit(\pi,X); create(\varphi,Y)\)\\
\lawvii. \(create(\pi,X); create(\varphi,Y) \equiv\)\\\indent\(%
create(\varphi,Y); create(\pi,X)\)\\
\lawviii. \(create(\pi,X); remove(\varphi) \equiv\crlw
remove(\varphi); create(\pi,X)\)\\
\lawvip. \(remove(\varphi); edit(\pi,X) \equiv\crlw
edit(\pi,X); remove(\varphi)\)\\
\lawviiip. \(remove(\varphi); create(\pi,X) \equiv\crlw
create(\pi,X); remove(\varphi)\)\\
\lawix. \(remove(\pi); remove(\varphi) \equiv\crlw
remove(\varphi); remove(\pi)\)
\lawsection{Breaking laws}
%1*. \(edit(\pi,\cfile{X}); edit(\pisub,Y) \equiv 
%edit(\pisub,Y); edit(\pi,\cfile{X}) \equiv break\)\\
\lawiia. \(edit(\pi,\cfile{X}); create(\pisub,Y) \equiv
break\)\\
\lawiiap. \(create(\pisub,Y); edit(\pi,\cfile{X}) \equiv
break\)\\
\lawiiib. \(edit(\pi,\cfile{X}); remove(\pisub) \equiv break\)\\
\lawx. \(edit(\pi,X); create(\pi,Y) \equiv break\)\\
\lawxi. \(edit(\pisub,X); create(\pi,Y) \equiv break\)\\
\lawxii. \(edit(\pisub,X); remove(\pi) \equiv break\)\\
\lawxiii. \(create(\pi,X); edit(\pisub,Y) \equiv break\)\\
\lawxiv. \(create(\pi,X); create(\pi,Y) \equiv break\)\\
\end{minipage}\begin{minipage}{7.2cm}\setlength{\rightskip}{0cm plus 7cm}
\lawxv. \(create(\pisub,X); create(\pi,Y) \equiv break\)\\
\lawxvi. \(create(\pi,X); remove(\pisub) \equiv break\)\\
\lawxvii. \(create(\pisub,X); remove(\pi) \equiv break\)\\
\lawxviii. \(remove(\pi); edit(\pi,X) \equiv break\)\\
\lawxix. \(remove(\pi); edit(\pisub,X) \equiv break\)\\
\lawxx. \(remove(\pi); create(\pisub,X) \equiv break\)\\
\lawxxi. \(remove(\pisub); create(\pi,X) \equiv break\)\\
\lawxxii. \(remove(\pi); remove(\pi) \equiv break\)\\
\lawxxiii. \(remove(\pi); remove(\pisub) \equiv break\)
\lawsection{Simplifying laws}
\lawxxiv. \(edit(\pi,X); edit(\pi,Y) \eqexp edit(\pi,Y)\)\\
\lawxxv. \(edit(\pi,X); remove(\pi) \equiv remove(\pi)\)\\ 
\lawxxvi. \(create(\pi,X); edit(\pi,Y) \equiv create(\pi,Y)\)\\
\lawxxvii. \(create(\pi,X); remove(\pi) \eqexp \emptyseq\)\\
\lawxxviii. \(remove(\pi); create(\pi,X) \eqexp edit(\pi,X)\)
\lawsection{Laws for \(break\)}
\lawxxix. \(break; edit(\pi,X) \equiv break\)\\
\lawxxx. \(break; create(\pi,X) \equiv break\)\\
\lawxxxi. \(break; remove(\pi) \equiv break\)\\
\lawxxxii. \(edit(\pi,X); break \equiv break\)\\
\lawxxxiii. \(create(\pi,X); break \equiv break\)\\
\lawxxxiv. \(remove(\pi); break \equiv break\)\\
\lawxxxv. \(break; break \equiv break\)
\lawsection{Remaining pairs\\(no substitution)}
\lawiibp. \(edit(\pi,\cdir{X}); create(\pisub,Y)\eqshr\)\\\indent\(%
create(\pisub,Y); edit(\pi,\cdir{X})\)\\
\lawiiic. \(remove(\pisub); edit(\pi,\cfile{X})\)\\
\lawxxxvi. \(create(\pi,X); create(\pisub,Y)\)\\
\lawxxxvii. \(remove(\pisub); remove(\pi)\)
\end{minipage}
}}
}
%\begin{forrsi}
%\doublespacing
%\end{forrsi}
\caption{Algebraic laws}
\end{table}
%}

\subsection{Inference rules}

An inference rule makes it possible to derive new statements from 
statements already true or from the axioms. 
Our inference rules are:\\
\emph{For any sequences \(S_1, S_2, S, S'\):}
\begin{itemize}
\item if \(S_1\equiv S_2\) then \(S;S_1;S'\equiv S;S_2;S'\)
\item if \(S_1\eqexp S_2\) then \(S;S_1;S'\eqexp S;S_2;S'\)
\item if \(S_1\equiv S_2\) then \(S_1\eqexp S_2\).
\end{itemize}
As we can see, the first inference rules are mere substitutions of a part
of the sequence to another sequence. 
%%\remark{``...'' is only for app. \ref{app:theosound}}
If ``\(S_1\equiv S_2\)'' or 
``\(S_1\eqexp S_2\)'' is a law itself, we say that we applied that law to
the sequence \(S;S_1;S'\).

\subsection{Soundness theorem}
\label{theorem:soundness}
In order to be able to use our algebra,
we must show that algebraic relation between sequences also means
that the sequences act the same on filesystems.
%
It can be proven that for every two sequences of commands 
\begin{gather*}
S\equiv S^* \implies \forall F: SF=S^*F,\\
(S\eqexp S^*)\wedge(SF\ne\ud)\implies \forall F: SF=S^*F.%\\
%S\eqifnbr S^*\wedge SF\ne\ud \wedge S^*F\ne\ud 
%\implies \forall F: SF=S^*F.
\end{gather*}
%%%\begin{forrsi}
The proof for the two cases are very similar. We use induction on the
number of times the inference rules were applied to the sequences. 
We obtain these results by
showing the soundness of each individual law (which can be done by
investigating numerous cases) and the inference rules.

The detailed proof can be found in Appendix \ref{app:theosound}.
%%%\end{forrsi}
%%%\begin{notrsi}
%%%%% Proof for the soundness theorem + foreword
%%%\begin{forrsi}
\section{Proof for the soundness theorem}
\label{app:theosound}

Let us repeat our statement:
We prove that for every two sequences of commands 
\begin{gather*}
S\equiv S^* \implies \forall F: SF=S^*F,\\
S\eqexp S^*\wedge SF\ne\ud \implies \forall F: SF=S^*F.%\\
%S\eqifnbr S^*\wedge SF\ne\ud \wedge S^*F\ne\ud 
%\implies \forall F: SF=S^*F.
\end{gather*}
%%%\end{forrsi}

Both statement is true if our axioms and inference rules are sound since
we can only gain equivalent (or extending) sequences from axioms
or by using inference rules.

First, we prove the soundness of our inference rules. The proof for the
first two inference rules are quite similar. For the first one, we know
that the intended interpretation of \(S_1\equiv S_2\) is true and our
statement is that for every \(S,S'\), the interpretation of
\(S;S_1;S'\equiv S;S_2;S'\) holds. 
We know that for every filesystem
\(F\), \((S;S_1;S')F=S'(S_1(SF))\) and similarly
\(((S;S_2;S')F=S'(S_2(SF))\). We know that for every
\(G\), \(S_1G=S_2G\) holds, and therefore
as a special case \(S_1(SF)=S_2(SF)\). Thus
\(\forall F: (S;S_1;S')F=((S;S_2;S')F\). This is the interpretation of our
statement.

For the second law, we now that the interpretation of 
\(S_1\eqexp S_2\) is true and we show that in that
case \(S;S_1;S'\eqexp S;S_2;S'\) also true for filesystems. 
We know that for every \(F\) if \(S_1F\ne\ud\) then \(S_1F=S_2F\). If
\((S;S_1;S')F=\ud\) then the relation is clearly true (since it refers to
only the case if the filesystem is not broken). In the other case, if
\((S;S_1;S')F=S'(S_1(SF))\ne\ud\), we know that \(SF\ne\ud\) since
%%\remark{`defined' is not quite defined}
a filesystem cannot be made non--broken again by applying commands to it.
Now since \(S_1\eqexp S_2\) is true therefore either 
\(S_1(SF)=\ud\) and the relation is true similarly to the case above;
or \(S_1(SF)\ne\ud\) and therefore
\(S_1(SF)=S_2(SF)\ne\ud\) and \(S'(S_1(SF))=S'(S_2(SF))\ne\ud\).

The soundness proof of the third law is quite simple since if two
sequences act the same on \emph{all} filesystems then they act the same on
a subset of the filesystems, too.

It remains to show the soundness of each individual law,
but all the laws were derived from the definitions of the
commands. For precise proofs,
Appendix~\ref{app:soundness} shows an example.

Our theorem is proved.


%%%\end{notrsi}

\subsection{Theorem of completeness}
\label{theorem:completeness}
In this section, we show that our proof system is complete for its
interpretation. This is also mandatory to be able to use our algebra.

Let us introduce some notation. We write 
\(S_1\eqifnbr S_2\), or \emph{\(S_1\) and \(S_2\) have a common upper 
bound} iff
\(\exists S^*: S_1\eqexp S^* \wedge S_2\eqexp S^*\), that is, iff
both \(S_1\) and \(S_2\) can be extended to the same sequence.
It is a symmetric relation, but not transitive.

\(S_1\equiv S_2\implies S_1\eqexp S_2\implies S_1\eqifnbr S_2\) also
holds.

%To show that we have enough laws to describe the algebra of
%filesystems,
Now we prove that if 
two sequences of commands act the same on any filesystem neither of
them breaks, \emph{and} there is a filesystem neither of them
breaks then they have a common upper bound.
Formally, 
\[\begin{array}{r@{}l}
\forall S,S':&\\ 
&((\forall G: (SG\ne\ud\wedge S'G\ne\ud)\implies SG=S'G)\\
&\wedge\\
&(\exists F: SF\ne\ud\wedge S'F\ne\ud)\\
&\implies S \eqifnbr S'),
\end{array}\]
where \(G\) and \(F\) refer to filesystems. 

%%%\begin{notrsi}
%%%%% Proof for completeness theorem +foreword
\section{Proof of the completeness theorem}
\label{app:compl}

Let us repeat our theorem and introduce minimal sequences again:
\[\begin{array}{r@{}l}
\forall S,S':&\\
&(\forall G: (SG\ne\ud\wedge S'G\ne\ud)\implies SG=S'G)\\
&\wedge\\
&(\exists F: SF\ne\ud\wedge S'F\ne\ud)\\
&\implies S \eqifnbr S',
\end{array}\]
where \(G\) and \(F\) refer to filesystems.
\newcommand{\varsection}[1]{\subsection{#1}}%

Throughout the proof, 
% by \(F\), we
% mean a filesystem which satisfies the second condition. 
by \(G\), we
refer to any filesystem which satisfies \(SG\ne\ud\wedge S'G\ne\ud\).

Now consider the set of sequences \(\wp_S = \{ S^* | S \eqexp S^*\}\).
Because of our preconditions, the sequence \(break\) is not in
\(\wp_S\) (if it was, that is, \(S\eqexp break\), \(SG=\ud\) would hold
based on the soundness theorem).

Let \(S_0\) be (one of) the shortest sequence(s) in \(\wp_S\) and
similarly
\(S^\prime_0\) (one of) the shortest sequence(s) in \(\wp_{S^\prime}\). 
%
The following holds for these minimal sequences:
\[(SG\ne\ud\wedge S'G\ne\ud\implies) \quad SG=S'G=S_0G=S'_0G\ne\ud.\]
\begin{notrsi}
\emph{Proof.} Since 
\(SG\ne\ud\wedge S\eqexp S_0\implies SG=S_0G\) and
\(S'G\ne\ud\wedge S'\eqexp S'_0\implies S'G=S'_0G\); and since
\(SG\ne\ud\wedge S'G\ne\ud\) by assumption \(SG=S'G\ne\ud\) holds,
we have \(S_0G=SG=S'G=S'_0G\ne\ud\).

% As a special case, we also know that \(SF=S'F=S_0F=S'_0F\ne\ud\).

\medskip
\end{notrsi}
Now let us investigate these minimal sequences. 

\varsection{Investigating the command \emph{edit}}

We know that the command \(edit(\pi,\cdir{X})\) commutes or collapses
(i.e., a commuting or a simplifying law can be applied to it) with
every command on its left side (see Laws \lawi, \lawip, \lawiv,
\lawvp, \lawvip,
\lawiiiap, \lawxiii, \lawxviii, \lawxix, \lawxxiv, \lawxxvi, \lawiib).
%
We also know that \(edit(\pi,\cfile{X})\) does the same on its right side
(see Laws \lawi, \lawip, \lawiv, \lawv, \lawvi, \lawiia, \lawiiib, \lawx,
\lawxi, \lawxii, \lawxxiv, \lawxxv).
%
From Laws \lawi, \lawip~and \lawxxiv~we know that \(edit\)s commute
amongst each other.

Therefore in a minimal sequence all \(edit(\pi,\cdir{X})\) commands can be
moved to the beginning, and all \(edit(\pi,\cfile{X})\) commands to the
end of the sequence. Since they commute amongst themselves, we can order
the two groups alphabetically. Therefore if there were two commands on
the same path, they would be neighbors and would be simplified by the
algebraic laws. 
Therefore we can be sure that there are at most one
\(edit(\pi,\cfile{X})\) or \(edit(\pi,\cdir{X})\) command on each path.

This way we also separated these commands from the rest.
Now we can focus on the remaining part: on the \(create\) and 
\(remove\) commands.

\varsection{Investigating \emph{create} and \emph{remove}}
\label{theorem:crrm}

We will prove some lemmas about minimal sequences.
\begin{description}
\item[Lemma 0]
If a sequence is minimal, then it cannot have any pairs that match the
left-hand sides of Laws \lawx--\lawxxxv, \lawiia, \lawiiap, \lawiiib.
Also, if we apply any of the
commutative laws 
(\lawi, \lawip, \lawiib, \lawiiia, \lawiiiap, \lawiv--\lawix),
the resulting sequence is still minimal.

\emph{Proof.}
For the first part, the laws mentioned all give an equivalent shorter
sequence, but by hypothesis, there is no equivalent shorter sequence.
For the second part, the commutative laws don't change the length of a
sequence.

\item[Lemma 1] {\bf It is impossible that a command
\(remove(\pi)\) precedes (not necessarily as a neighbor) a command
\(remove(\piesub)\) in a minimal sequence.} 
\begin{notrsi}
(A parent can't be removed before any descendant, or
more precisely, \(\not\exists i,j: i<j \land S[i] = remove(\pi) \land
S[j] = remove(\pi') \land \pi \preceq \pi'\).)
\end{notrsi}
\begin{notrsi}
\begin{center}
\begin{tabular}{c|c|c|c|c|c|c|c|c}
\hline
\(\cdots\) & & \(S[i]: remove(\pi)\) & & & & &
\(S[j]: remove(\piesub)\) & \(\cdots\) \\
\hline
\end{tabular}
\end{center}
\end{notrsi}
\emph{Proof.}
By contradiction; we assume there is such an \(i\) and \(j\), and we show
that implies \(S \eqexp break\).

We show the contradiction by induction on \(j-i\).
The base case is \(j = i+1\).
In this case, by Laws \lawxxii~and~\lawxxiii, 
\(S[i]; S[i+1] \eqexp break\),
and therefore \(S\eqexp break\).

For the induction step, we perform a case analysis on \(S[j-1]\).
\begin{itemize}
\item
If it mentions a path that is disjoint with \(\piesub\),
%or if $\pi \prec \piesub$ and it is $edit(\pi, \cdir X)$,
we can swap it
with \(S[j]\) to get an equivalent sequence, and by the induction   
hypothesis and transitivity of equivalence, \(S\eqexp break\).
%\item
%By Lemma 0 (Laws~3B, 12, and~25), it cannot be any other edit operation.
\item
Also by Lemma 0 (Laws~\lawxvi, \lawxvii, and~\lawxxvii), it cannot be a
non-disjoint create operation. (Note that there are no \(edit\) operations
in this part of the sequence according to our preconditions.)
\item
If $S[j-1] =remove(\hat\pi)$, if $\hat\pi \preceq \piesub$, then by laws
\lawxxii~and~\lawxxiii, $S[j-1]; S[j] \equiv \{break\}$, and therefore 
$S\eqexp\{break\}$.
But if $\piesub \prec \hat\pi$, then by transitivity $\pi \preceq
\hat\pi$, so the induction hypothesis applies, and again 
$S \eqexp break$.
\end{itemize}

\item[Lemma 2]
{\bf A command \(create(\pi)\) can not precede a command  
\(remove(\piesub)\).} 
\emph{Proof.} 
This can be proved the same way, only
the base case differs; now 
\(create(\pi); remove({\piesub})\) equals \(break\) or expands 
to \(skip\) by Law \lawxvi~and \lawxxvii.
The induction step is the same.

\item[Lemma 3]
{\bf A command \(remove(\piesub)\) can not precede a command
\(create(\pi)\).} 
\emph{Proof.} This can be proved similarly, in this 
case the base step is
\(remove(\piesub); create({\pi})\) equals \(break\) or
expands to \(edit\) (Law \lawxxi~and \lawxxviii);
in the induction step we can prove that if
\(S[j-1]=create(\pisup)\) precedes \(S[j]=create(\pi)\), according to
the induction hypothesis, \(S\eqexp break\) 
(we know that \(\pisup\prec\pi\)). Otherwise it would be possible to
shorten the sequence (which contradicts our precondition) or commuted (and therefore the induction hypothesis
still holds). (Laws \lawvii,
\lawviii, \lawxiv, \lawxv, \lawxx, \lawxxi, \lawxxviii.)

\item[Lemma 4]
{\bf A command \(create(\piesub)\) can not precede a command
\(create(\pi)\).} 
\emph{Proof.} 
We obtain this result from Lemma 3: the base step is 
\(create(\piesub); create({\pi}) \equiv (\eqexp) break\) (Law 15); the
induction step is the same.

\item[Lemma 5]
{\bf A command \(create(\piesub)\) can not precede a command
\(remove(\pi)\).} 
\emph{Proof.} 
Our base step now is 
\(S[j-1];S[j]=create({\piesub}); remove(\pi)\eqexp skip\) or equals
\(break\) according to Laws \lawxxvii~and \lawxvii.
But in our induction step, we go in the opposite direction and perform
analysis on \(S[i+1]\).
If \(S[i+1]=create({\piesub}_{sub})\) follows \(S[i]=create(\piesub)\),
the induction hypothesis applies, therefore \(S\eqexp break\). Otherwise
the commands could be simplified or commuted (Laws \lawxviii, \lawxiv,
\lawxv, \lawxvi, \lawxvii, \lawxxvii.)

\item[Lemma 6] 
{\bf It is impossible that a command \(remove(\pi)\)
precedes a command \(create(\piesub)\).} 
\emph{Proof.} 
Now our base step is
\(remove(\pi); create(\piesub)\eqexp edit\) or equals to
\(break\) (Laws \lawxxviii~and \lawxx). In the induction step we
examine command \(S[i+1]\). 
If it's not commutable or simplifyable,
it can be only \(remove(\pisup)\).
In that case the induction hypothesis applies.
\end{description}

\varsection{More lemmas on commands}

Now we go back to \(edit\) commands. We will prove two additional lemmas.
\begin{description}
\item[Lemma E1] {\bf A command \(remove(\pi)\) cannot precede a command
\(edit(\pi,\cfile{X})\).} \emph{Proof.} 
By contradiction; we assume there are such two commands in the sequence,
\(S[i]=remove(\pi)\) and \(S[j]=edit(\pi,\cfile{X})\) where \(i<j\). We
use induction on \(j-i\). The base
step is when \(j-i=1\). Now, according to Law \lawxviii, \(S\equiv
break\) which contradicts our condition on \(S\).
For the induction step, we assume that \(S\eqexp break\) if
\(S[j-1]=edit(\pi,\cfile{X})\). Now we investigate \(S[j-1]\). If it is
not \(remove(\pisub)\), then a commuting (or simplifying) law can be
applied to \(S[j-1];S[j]\). That way
\(S[j-1]\) would be \(edit(\pi,\cfile{X})\), and according to the
induction hypothesis \(S\eqexp break\). If \(S[j-1]\) is
\(remove(\pisub)\), we have the same result by Lemma 1.

\item[Lemma E2] {\bf A command \(create(\pi)\) cannot precede a command 
\(edit(\pi)\).} 
\emph{Proof.} The same; last step is made according to
Lemma 2.

\item[Lemma E3] {\bf A command \(remove(\pi)\) cannot follow a
command \(edit(\pi,\cdir{X})\).} 
\emph{Proof.} 
This proof is also very similar to the aboves. The base step is 
\(edit(\pi,\cdir{X}); remove(\pi)\equiv remove(\pi)\) (Law \lawxxv). In the
induction step we investigate command \(S[i+1]\). If it was not
\(create(\pisub)\), a commuting (or simplifying) law could be applied. If
it is, we have \(S\eqexp break\) according to Lemma 5.

\item[Lemma E4] {\bf A command \(create(\pi)\) cannot follow a
command \(edit(\pi,\cdir{X})\).} 
\emph{Proof.} The same; last step
according to Lemma 4.
\end{description}

\noindent
Keeping in mind that we have all the \(edit\) commands at the ends of the
sequence, it follows from Lemma E1, E2, E3 and E4 that there cannot be an
\(edit\) and another command on the same path.

\begin{description}
\item[Lemma E5] {\bf A command \(edit(\pi,\cdir{X})\) cannot precede a law
\(edit(\pi,\cfile{X})\).} \emph{Proof.} Now we know that between these
commands there are no commands on path \(\pi\). We also know from Lemmas
1---6 that there is at most one \(remove\) or \(create\) command on each
path. Therefore
there cannot be a \(create(\pisub)\) command since we would have to
remove it before modifying \(\pi\) to a file.
Thus we could have moved \(edit(\pi,\cdir{X})\) to the end of the sequence
and simplify it with \(edit(\pi,\cfile{X})\).
\end{description}

\varsection{Conclusions}
\label{theorem:conc}

As a result of the lemmas we know that
\emph{there is at most one command on each path in a minimal sequence.}

Now we prove that \(C\in S_0\Longleftrightarrow C\in S'_0\) for any
command \(C\). (Without loss of generality, it is enough to prove that 
\(C\in S_0\implies C\in S'_0\).)
\begin{itemize}
\item \(create(\pi,X)\in S_0\implies create(\pi,X)\in S'_0\). 
\emph{Proof.} By contradiction. First of all, we know that
\(G(\pi)=\ud\) otherwise \(S_0\) would break \(G\) and that would
contradict our assumption. Now assume that 
\(create(\pi,X)\not\in S'_0\). 
We have three cases. If there is no command on path \(\pi\) in
\(S'_0\) then \(S'_0G(\pi)=\ud\) and therefore 
\(S_0G(\pi)\ne S'_0G(\pi)\), that is, \(S_0G\ne S'_0G\) which contradicts
our assumption. If there was a command \(edit\) or
\(remove\) on path \(\pi\), it would break \(G\) since \(G(\pi)=\ud\) and
that again contradicts the precondition.

\item \(remove(\pi)\in S_0\implies remove(\pi)\in S'_0\).
\emph{Proof.} 
Now we can be sure that \(F(\pi)\ne\ud\) and
\(S_0G(\pi)=\ud\). If (instead of
\(remove\)) there was no command on \(\pi\) in \(S'_0\) then 
\(S'_0G(\pi)\ne\ud\)  would hold. In case of \(edit\), 
\(S'_0G(\pi)\ne\ud\) would hold (contradicting
\(S_0G(\pi)=\ud\)). \(create\) would break \(G\) since
\(G(\pi)\ne\ud\). Again, we have contradiction in all cases.

\item \(edit(\pi,X)\in S_0\implies edit(\pi,X)\in S'_0\).
\emph{Proof.} Similarly we cannot have \(create\) or \(remove\) on path
\(\pi\) in sequence \(S'_0\) since \(G(\pi)\ne\ud\) and 
\(S'G(\pi)=SG(\pi)\ne\ud\). 
But now we might have no commands on
\(\pi\) in \(S'_0\): that way if \(G(\pi)=X\) then \(S_0G(\pi)=X\) and
\(S'_0G(\pi)=X\), too. 
% But clearly \(\neg(S_0\eqifnbr S'_0)\). 
To prove that this is also
impossible, we need the first condition about all filesystems. 

Define
\(H\) as \(G\{\pi\mapsto Y\}\) where \(Y\) has the same type
(file/directory) as \(G(\pi)\). If \(SG\ne\ud\) then \(SH\ne\ud\) since no
command breaks the filesystem because of the contents (not the type) of a
file or directory. Also, \(S'G\ne\ud\implies S'H\ne\ud\). Now, 
from the coditions of the theorem
and \(SH\ne\ud\wedge S'H\ne\ud\) we know
that \(S_0H=S'_0H\). But \(S_0H(\pi)=edit(\pi,X)H(\pi)=X\) and
\(S'_0H(\pi)=H(\pi)=Y\). We have a contradiction, therefore the lemma is
proved.
\end{itemize}

This means that
\emph{\(S_0\) and \(S^\prime_0\) contain the same commands;} they can be
different only in the order of the commands.

\medskip
Now we will show that they can be made exactly the same using the
commuting algebraic laws. 

We know that \(edit\)s can be moved to the ends of the sequences and
ordered. Let us suppose they are arranged this way.
Since we know that the two sequences contain the same
commands, these parts of the sequences is the same.
Therefore we can omit them in the discussion. Now we focus on
\(create\)s and \(remove\)s as we did above.

%%%\begin{notrsi}
If there are two commands in a minimal sequence
referring to comparable directories, according to the lemmas, they
can only be
%not be
%\begin{itemize}
%\item \(remove(\pi) \ldots remove(\pisub)\)
%\item \(remove(\pi) \ldots create(\pisub)\)
%\item \(create(\pisub) \ldots create(\pi)\)
%\item \(create(\pisub) \ldots remove(\pi)\)
%\item \(create(\pi) \ldots remove(\pisub)\)
%\item \(remove(\pisub) \ldots create(\pi)\).
%\end{itemize}
%The remaining possibilities are:
\begin{itemize}
\item \(create(\pi) \ldots create(\pisub)\)
\item \(remove(\pisub) \ldots remove(\pi)\).
\end{itemize}

Therefore any two \(create\) or \(remove\) commands in a (minimal)
sequence can be freely interchanged if they refer to incomparable
directories, but they have a well--defined order otherwise. (Keep in
mind that we have no \(edit\) commands in this part of the sequences.)
That is, we have a partial order over the set of these commands. Note
that we cannot have cycles in this order (since that way \(\pi\) would 
be equal to one of its descendants). Because of this, there is always
a minimal element, which has no predecessors. Also, this ordering is the
same on both sequences. 
%%%\end{notrsi}

We prove that the commands can be ordered the same way by induction on the
length of the sequences. If it
is \(1\), the problem can be solved easily. If the length of the sequences
is \(i>1\), we can choose a command from \(S_0\) which has no
command that must precede it.
It is sure that it has no precedents in \(S'_0\). Now we
can move this command to the front of the sequences. Now the rest of
the sequences have length \(i-1\) therefore we can order them according to
the induction hypothesis.

We obtain \(S_0^*\) and \({S^\prime}_0^*\) by applying this method to \(S_0\) and
\(S^\prime_0\). Since they contain the same
commands, and the resulting order is the same,
\(S_0^*\) and
\({S^\prime_0}^*\) are \emph{exactly the same}. Now we have
\(S\eqexp S_0\eqexp S_0^*\equiv{S^\prime_0}^*\eqshr S^\prime_0\eqshr
S^\prime\), that is, \(S\eqifnbr S^\prime\). Our theorem is proved.



%%%\end{notrsi}
%%%\begin{forrsi}

In the proof we define \emph{minimal sequences} in the following way:
Consider the set of sequences \(\wp_S = \{ S^* | S \eqexp S^*\}\).
Because of our preconditions, the sequence \(break\) is not in
\(\wp_S\) (if it was, that is, \(S\eqexp break\), \(SF=\ud\) would hold
since by definition \(SF=(break)F\) if \(SF\ne\ud\)).

Let \(S_0\) be (one of) the shortest sequence(s) in \(\wp_S\) and
similarly
\(S^\prime_0\) (one of) the shortest sequence(s) in \(\wp_{S^\prime}\). 
These sequences are called minimal sequences. It can be shown that for
every \(G\) which satisfies the condition \(SG\ne\ud\wedge S'G\ne\ud\), 
\(S_0G=S'_0G\ne\ud\) applies.
As a special case, we know that \(S_0F=S'_0F\ne\ud\).

The proof has three main steps.
\begin{itemize}
\item[i.]
We have some constraints on \(S_0\) and \(S'_0\). Since they do not break
every filesystem, no breaking laws can be applied to them. And since
they have minimal length, we cannot apply a simplifying law either. From
these properties we can prove that there is at most one command on each
path in a minimal sequence. (If there were more, such a law would
be appliable.)
\item[ii.]
We know that a command on path \(\pi\) only affects the filesystem at
\(\pi\) if it does not break the filesystem. Therefore \(S_0\) and
\(S'_0\) must contain commands on the same paths since
they do not break \(F\) and \(S_0F=S'_0F\), so they modify \(F\) at 
the same points. 
And we can also prove that the sequences must contain the same
commands on each path from the fact that they act the same on every \(G\).
Therefore they consist of the very same commands.
\item[iii.]
We prove that \(S_0\) and \(S'_0\) can be reordered by commuting
laws so that they would be the same sequences. That is, a sequence
\(S^*\) exists for which \(S_0\eqexp S^*\eqshr S'_0\). It means that
\(S\eqexp S_0\eqexp S^*\eqshr S'_0\eqshr S'\), i.e. \(S\eqifnbr S'\).
\end{itemize}

In appendix \ref{app:compl} we provide the detailed version of the
proof.

\medskip
Now, since we know that our algebra is sound and complete for its
intended interpretation, we can use it to define the specification of the
file synchronizer.
%%%\end{forrsi}
