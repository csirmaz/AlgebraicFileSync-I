%% Algebra, laws, inference rule, description of theorems
\section{Algebra on commands}
\label{theorem:laws}

In order to reason about commands, we define \emph{algebraic laws} on
commands. We use formal logic to build a proof system which is sound and
complete for its intended interpretation.

For sequences \(S_1\) and \(S_2\), we have two kinds of judgements:
\begin{itemize}
\item \(S_1\equiv S_2\), or \emph{\(S_1\) is algebraically equivalent to
\(S_2\)}. Its intended interpretation is that they act the same on all
filesystems, i.e. \(\forall F: S_1F=S_2F\).
\item \(S_1\eqexp S_2\), or \emph{\(S_2\) extends \(S_1\)}; its intended
interpretation is that if \(S_1\) does not break a filesystem, then they act
the same on it, i.e. \(\forall F: S_1F\ne\ud \implies S_1F=S_2F\).
\end{itemize}

The axioms of our proof system are the laws listed in Table 1. We have two
inference rules, one for each judgement.

\subsection{The laws}

We created the laws with the help of pairs of commands.
Lines in the last section are not axioms; they are written
there to list all possible pairs.

\medskip
{\small{
How many pairs do we have? 
From
\(edit\), \(create\) and \(remove\), we can choose a pair in 9 ways.
Moreover, there are four types of
path-pairs: equivalence \(\pi;\pi\), subordination \(\pisub;\pi\),
superordination \(\pi;\pisub\)
and incomparable paths \(\pi;\varphi\). These make \(4\times9\) cases. With \(break\), we have
\(3\times 2\) additional cases
with another non--break command (there are 3 types and they can follow
or precede \(break\)), and one case when we have a pair made of two
\(break\)s. Thus, we have \(4\times 9+3\times2+1=43\) pairs. 
}}
\medskip

We subdivide some laws containing \(edit\) because of \(edit\)'s different
behavior when modifying paths to files or to directories. These laws are
numbered with A or B.
Extension laws (that is, where the relation is \(\eqexp\), not
\(\equiv\)) are marked with \(E\).

\ifnum\Textsize=1%
\def\crlw{\)\\\indent\(}%
\else%
\def\crlw{\relax}%
\fi%
\newcommand{\lawsection}[1]{\par\medskip{\bf{#1}}\medskip\\}%
\def\lawi{1}%
\def\lawip{2}%
\def\lawiib{4A${}_E$}%
\def\lawiibp{3A}% not true semilaw
\def\lawiiia{5A}%
\def\lawiiiap{6A}%
\def\lawiv{7}%
\def\lawv{8}%
\def\lawvi{9}%
\def\lawvp{10}%
\def\lawvii{11}%
\def\lawviii{12}%
\def\lawvip{13}%
\def\lawviiip{14}%
\def\lawix{15}%
\def\lawiia{3B}%
\def\lawiiap{4B}%
\def\lawiiib{5B}%
\def\lawx{16}%
\def\lawxi{17}%
\def\lawxii{18}%
\def\lawxiii{19}%
\def\lawxiv{20}%
\def\lawxv{21}%
\def\lawxvi{22}%
\def\lawxvii{23}%
\def\lawxviii{24}%
\def\lawxix{25}%
\def\lawxx{26}%
\def\lawxxi{27}%
\def\lawxxii{28}%
\def\lawxxiii{29}%
\def\lawxxiv{30${}_E$}%
\def\lawxxv{31}%
\def\lawxxvi{32}%
\def\lawxxvii{33${}_E$}%
\def\lawxxviii{34${}_E$}%
\def\lawxxix{35}%
\def\lawxxx{36}%
\def\lawxxxi{37}%
\def\lawxxxii{38}%
\def\lawxxxiii{39}%
\def\lawxxxiv{40}%
\def\lawxxxv{41}%
\def\lawiiic{6B}%
\def\lawxxxvi{42}%
\def\lawxxxvii{43}%
%%%
\ifnum\Textsize=1
\afterpage{\clearpage}
\begin{table}[H]
\singlespacing
\else
\begin{table}[thb]
\fi
\relax
\label{table:laws}
\relax
\framebox{
{\small{
\begin{minipage}{8.3cm}
\setlength{\rightskip}{0cm plus 7cm}
\lawsection{Commuting laws}
\lawi. \(edit(\pi,X); edit(\pisub,Y) \equiv\crlw
edit(\pisub,Y); edit(\pi,X)\)\\
\lawip. \(edit(\pisub,Y); edit(\pi,X) \equiv\crlw
edit(\pi,X); edit(\pisub,Y)\)\\
\lawiib. \(create(\pisub,Y); edit(\pi,\cdir{X})\eqexp\)\\\indent\(%
edit(\pi,\cdir{X}); create(\pisub,Y)\)\\
\lawiiia. \(edit(\pi,\cdir{X}); remove(\pisub) \equiv\)\\\indent\(%
remove(\pisub); edit(\pi,\cdir{X})\)\\
\lawiiiap. \(remove(\pisub); edit(\pi,\cdir{X})\equiv\)\\\indent\(%
edit(\pi,\cdir{X}); remove(\pisub)\)\\
\lawiv. \(edit(\pi,X); edit(\varphi,Y) \equiv
edit(\varphi,Y); edit(\pi,X)\)\\
\lawv. \(edit(\pi,X); create(\varphi,Y) \equiv\crlw
create(\varphi,Y); edit(\pi,X)\)\\
\lawvi. \(edit(\pi,X); remove(\varphi) \equiv\crlw
remove(\varphi); edit(\pi,X)\)\\
\lawvp. \(create(\varphi,Y); edit(\pi,X) \equiv\crlw
edit(\pi,X); create(\varphi,Y)\)\\
\lawvii. \(create(\pi,X); create(\varphi,Y) \equiv\)\\\indent\(%
create(\varphi,Y); create(\pi,X)\)\\
\lawviii. \(create(\pi,X); remove(\varphi) \equiv\crlw
remove(\varphi); create(\pi,X)\)\\
\lawvip. \(remove(\varphi); edit(\pi,X) \equiv\crlw
edit(\pi,X); remove(\varphi)\)\\
\lawviiip. \(remove(\varphi); create(\pi,X) \equiv\crlw
create(\pi,X); remove(\varphi)\)\\
\lawix. \(remove(\pi); remove(\varphi) \equiv\crlw
remove(\varphi); remove(\pi)\)
\lawsection{Breaking laws}
\lawiia. \(edit(\pi,\cfile{X}); create(\pisub,Y) \equiv
break\)\\
\lawiiap. \(create(\pisub,Y); edit(\pi,\cfile{X}) \equiv
break\)\\
\lawiiib. \(edit(\pi,\cfile{X}); remove(\pisub) \equiv break\)\\
\lawx. \(edit(\pi,X); create(\pi,Y) \equiv break\)\\
\lawxi. \(edit(\pisub,X); create(\pi,Y) \equiv break\)\\
\lawxii. \(edit(\pisub,X); remove(\pi) \equiv break\)\\
\lawxiii. \(create(\pi,X); edit(\pisub,Y) \equiv break\)\\
\lawxiv. \(create(\pi,X); create(\pi,Y) \equiv break\)\\
\end{minipage}\begin{minipage}{7.2cm}\setlength{\rightskip}{0cm plus 7cm}
\lawxv. \(create(\pisub,X); create(\pi,Y) \equiv break\)\\
\lawxvi. \(create(\pi,X); remove(\pisub) \equiv break\)\\
\lawxvii. \(create(\pisub,X); remove(\pi) \equiv break\)\\
\lawxviii. \(remove(\pi); edit(\pi,X) \equiv break\)\\
\lawxix. \(remove(\pi); edit(\pisub,X) \equiv break\)\\
\lawxx. \(remove(\pi); create(\pisub,X) \equiv break\)\\
\lawxxi. \(remove(\pisub); create(\pi,X) \equiv break\)\\
\lawxxii. \(remove(\pi); remove(\pi) \equiv break\)\\
\lawxxiii. \(remove(\pi); remove(\pisub) \equiv break\)
\lawsection{Simplifying laws}
\lawxxiv. \(edit(\pi,X); edit(\pi,Y) \eqexp edit(\pi,Y)\)\\
\lawxxv. \(edit(\pi,X); remove(\pi) \equiv remove(\pi)\)\\ 
\lawxxvi. \(create(\pi,X); edit(\pi,Y) \equiv create(\pi,Y)\)\\
\lawxxvii. \(create(\pi,X); remove(\pi) \eqexp \emptyseq\)\\
\lawxxviii. \(remove(\pi); create(\pi,X) \eqexp edit(\pi,X)\)
\lawsection{Laws for \(break\)}
\lawxxix. \(break; edit(\pi,X) \equiv break\)\\
\lawxxx. \(break; create(\pi,X) \equiv break\)\\
\lawxxxi. \(break; remove(\pi) \equiv break\)\\
\lawxxxii. \(edit(\pi,X); break \equiv break\)\\
\lawxxxiii. \(create(\pi,X); break \equiv break\)\\
\lawxxxiv. \(remove(\pi); break \equiv break\)\\
\lawxxxv. \(break; break \equiv break\)
\lawsection{Remaining pairs\\(no substitution)}
\lawiibp. \(edit(\pi,\cdir{X}); create(\pisub,Y)\eqshr\)\\\indent\(%
create(\pisub,Y); edit(\pi,\cdir{X})\)\\
\lawiiic. \(remove(\pisub); edit(\pi,\cfile{X})\)\\
\lawxxxvi. \(create(\pi,X); create(\pisub,Y)\)\\
\lawxxxvii. \(remove(\pisub); remove(\pi)\)
\end{minipage}
}}
}
\caption{Algebraic laws}
\end{table}

\subsection{Inference rules}

An inference rule makes it possible to derive new statements from 
the axioms or from statements already known to be true. 
The inference rules are:\\
\emph{For any sequences \(S_1, S_2, S, S'\):}
\begin{itemize}
\item if \(S_1\equiv S_2\) then \(S;S_1;S'\equiv S;S_2;S'\)
\item if \(S_1\eqexp S_2\) then \(S;S_1;S'\eqexp S;S_2;S'\)
\item if \(S_1\equiv S_2\) then \(S_1\eqexp S_2\).
\end{itemize}
The first two inference rules merely substitute part of a sequence for
another sequence.
If ``\(S_1\equiv S_2\)'' or 
``\(S_1\eqexp S_2\)'' is a law, we say that we applied that law to
the sequence \(S;S_1;S'\).

\subsection{Soundness theorem}
\label{theorem:soundness}
In order to be able to use our algebra,
we must show that an algebraic relation between sequences also means
that the sequences act the same on filesystems.
%
It can be proven that for every two sequences of commands
\(S\) and \(S^*\) 
\begin{gather*}
S\equiv S^* \implies \forall F: SF=S^*F,\\
S\eqexp S^* \implies \forall F: (SF\ne\ud \implies SF=S^*F).
\end{gather*}
The proofs for the two cases are very similar. We use induction on the
number of times the inference rules are applied to the sequences.
The soundness of each individual law
(by investigating individual cases)
and that of the inference rules are shown separately.
For the detailed proof see Appendix \ref{app:theosound}.

\medskip\noindent{\small{As special consequences of the soundness theorem, we know
that both \(S\equiv break\)
and \(S\eqexp break\) means that \(S\) breaks all filesystems.
--- EC, 2015
}}

\subsection{Theorem of completeness}
\label{theorem:completeness}
In this section, we show that our proof system is complete for its
interpretation. This is also mandatory to be able to use our algebra.

Let us introduce some notation. We write 
\(S_1\eqifnbr S_2\), or \emph{\(S_1\) and \(S_2\) have a common upper 
bound} iff
\(\exists S^*: S_1\eqexp S^* \wedge S_2\eqexp S^*\), that is, iff
both \(S_1\) and \(S_2\) can be extended to the same sequence.
It is a symmetric relation, but not transitive.
\(S_1\equiv S_2\implies S_1\eqexp S_2\implies S_1\eqifnbr S_2\) also
holds.

Now we prove that if 
two sequences of commands act the same on any filesystem neither of
them breaks, \emph{and} there is a filesystem neither of them
breaks, then they are also related algebraically; namely, they have a common upper bound.
Formally, 
\[\begin{array}{r@{}l}
\forall S,S':&\\ 
&(\forall G: (SG\ne\ud\wedge S'G\ne\ud)\implies SG=S'G)\\
&\wedge\\
&(\exists F: SF\ne\ud\wedge S'F\ne\ud)\\
&\implies S \eqifnbr S',
\end{array}\]
where \(G\) and \(F\) refer to filesystems. 


In the proof we define \emph{minimal sequences} in the following way:
Consider the set of sequences \(\wp_S = \{ S^* | S \eqexp S^*\}\).
Because of our preconditions, the sequence \(break\) is not in
\(\wp_S\) (if it was, that is, \(S\eqexp break\), \(\forall F:SF=\ud\) would
hold).

Let \(S_0\) be (one of) the shortest sequence(s) in \(\wp_S\) and
similarly
\(S^\prime_0\) (one of) the shortest sequence(s) in \(\wp_{S^\prime}\). 
We will call these minimal sequences. It can be shown that for
every \(G\) which satisfies the condition \(SG\ne\ud\wedge S'G\ne\ud\), 
\(S_0G=S'_0G\ne\ud\) applies.
As a special case, we know that \(S_0F=S'_0F\ne\ud\) where \(F\) is the
fileystem from the formalization of the theorem above.

The proof has three main steps.
\begin{itemize}
\item[i.]
We have some constraints on \(S_0\) and \(S'_0\). Since they do not break
every filesystem, no breaking laws can be applied to them. And since
they have minimal length, we cannot apply a simplifying law either. From
these properties we can prove that there is at most one command on each
path in a minimal sequence. (If there were more, a breaking or simplifying law
could be applied.)
\item[ii.]
We know that a command on path \(\pi\) only affects the filesystem at
\(\pi\) if it does not break the filesystem. Therefore \(S_0\) and
\(S'_0\) must contain commands on the same paths since
they do not break \(F\) and \(S_0F=S'_0F\), that is, they modify \(F\) at 
the same points. 
And we can also prove that the sequences must contain the same
commands on each path from the fact that they act the same on every \(G\).
Therefore they consist of the very same commands.
\item[iii.]
We prove that \(S_0\) and \(S'_0\) can be reordered using commuting
laws so that they would be the same sequences. That is, a sequence
\(S^*\) exists for which \(S_0\eqexp S^*\eqshr S'_0\). It means that
\(S\eqexp S_0\eqexp S^*\eqshr S'_0\eqshr S'\), i.e. \(S\eqifnbr S'\).
\end{itemize}

In appendix \ref{app:compl} we provide the detailed version of the
proof.

\medskip
Now, since we know that our algebra is sound and complete for its
intended interpretation, we can use it to define the specification of the
file synchronizer.
