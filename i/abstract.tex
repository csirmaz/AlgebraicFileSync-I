%% Abstract
The problem of file synchronization (making two, differently modified
copies of a filesystem the same again without losing information) emerges
in many cases, and is solved in many different ways. Our goal was to
create a \emph{general mathematical model} for file synchronization and
use it as a basis to define the behavior of a synchronizer.

We developed a specific algebraic model on filesystem commands and proved
that it is sound and complete for its intended interpretation on real
systems, that is, if commands are considered to be equivalent according to
the algebra then they are equivalent when applied to a real filesystem and
vice versa.

Then we defined algorithms for synchronization using our algebra and
created an implementation which was tested on various filesystems. This
method turned out to be an effective way of creating the specification of a
synchronizer since it simplified both the definition and the
implementation.

The methods used in the proofs for soundness and completeness are likely
be usable on other algebras, too. Thus using this algebraic approach could make
it possible to extend synchronization to other types of datasystems and
to create a general theory of synchronization in the future.




