%% Introduction
\section{Introduction}

If we have multiple copies (called \emph{replicas}) of a filesystem or a
part of a filesystem (for example on a laptop and on a desktop
computer), it is often the case that changes are made to each
replica independently and as a result they no longer contain the same
information. In that case, a \emph{file synchronizer} is used to make them
similar (consistent) again, without losing any information.

The goal of a file synchronizer is to \emph{detect conflicting updates}
and \emph{propagate non--conflicting updates on each replica.} Due to the
various cases of file updates and the various behaviors of filesystem
commands it is difficult to give a complete and correct definition of its
behavior.

It can be helpful in solving this problem if we have a
formalized mathematical model for describing file synchronization.

We investigated algebraic models of synchronization in a specific case to
see the properties of such a system and the method of proving theorems in
it, which will hopefully aid us to create a general theory of
synchronization based on algebraic structures in the future. This specific
case was the synchronization of filesystems.

In the paper we develop an algebraic representation of filesystem
commands. Then we show that this model is both complete and sound,
and we define the expected behavior of a file synchronizer and
create its specification with the help of our algebra. 

This method will
hopefully simplify the specification as well as the implementation of the
synchronizer, and it might make it possible to extend the synchronizer to
other types of datasystems, such as mail folders or databases.

\subsection{The main idea of synchronization}

In general, there are two phases of synchronization:
\emph{update detection}, when the program recognizes where updates have
been made since the last synchronization, and \emph{reconciliation}, when
it combines updates to yield a new, synchronized version of each replica.
If there are no conflicting updates, then at the end of the
synchronization the replicas will be the same. Otherwise, a synchronizer
may leave them unchanged at paths where conflicts occurred and warn the
user.
In this paper,
we focus on the commands that were applied to these replicas while they
were modified independently, as the main aim of a synchronizer is to
propagate and perform
the most possible commands on all replicas.

%\remark{Explanation why we DO need O?}
Let us introduce some notation. We have the original system, \(O\); 
and the current replicas \(R_1, R_2, \ldots R_n\).
Let us assume that we already know the sequences of commands which have been applied to
\(O\) in each replica (for example, from the update detector). These
sequences are
\(S_1, S_2, \ldots S_n\). The synchronizer algorithm will neeed to provide us
with new sequences of commands for each replica (\(S_1^*, S_2^*, \cdots S_n^*\))
that will move each
replica to a common state \(O'\), or as close to it as possible.
%, that is, 
%\(S_1^*R_1=S_2^*R_2=\cdots=S_n^*R_n=O'\).

In order to determine each sequence \(S_i^*\), we take all commands
applied
to the systems (\(S_1\cup S_2\cup \cdots\cup S_n\)) and omit commands
in \(S_i\) which have already been applied to replica \(R_i\). 

However, this way we only gain a \emph{set} of commands which we
need to order to be able to apply the commands to the
system. It is possible that some orders cause errors in a filesystem
(for example, trying to remove a directory before removing the files in
it). It is also possible that not all error--free orders have the same
effect on the filesystem. For example, modifying the contents of a file to
``xxx'' and modifying them to ``yyy'' in a different replica are not commutable commands: they
leave the system in different states if applied in different order.
In this case the synchronizer cannot synchronize the systems fully, since
``last modification wins'' is not a preferable solution in every case.  
Therefore, a better aim of the synchronizer is to find commands for which 
\emph{all error--free orders have
the same effect} and apply these to the system.

The reason we may have two differently ordered sequences of commands which
do not lead a system to the same state is that we have incommutable pairs
of commands in them. This is because if all pairs of commands
commuted (that is, a pair of commands \(C_1; C_2\) had the same effect on
a system as \(C_2; C_1\)) then clearly one of the sequences could be
transformed into the other one by commuting commands
without any change in the effect of the sequence on the system. 

Because of this, we focus on the commutability of pairs of commands.  
We define this property with the help of an algebraic 
proof system on commands.% for commutivity. %%%

%\subsection{Synchronizing filesystems}
%This specific case was the synchronization of a filesystem. 

