%% Proof for completeness theorem +foreword
\section{Proof of the completeness theorem}
\label{app:compl}

Let us repeat our theorem and introduce minimal sequences again:
\[\begin{array}{r@{}l}
\forall S,S':&\\
&(\forall G: (SG\ne\ud\wedge S'G\ne\ud)\implies SG=S'G)\\
&\wedge\\
&(\exists F: SF\ne\ud\wedge S'F\ne\ud)\\
&\implies S \eqifnbr S',
\end{array}\]
where \(G\) and \(F\) refer to filesystems.
\newcommand{\varsection}[1]{\subsection{#1}}%

Throughout the proof, 
% by \(F\), we
% mean a filesystem which satisfies the second condition. 
by \(G\), we
refer to any filesystem which satisfies \(SG\ne\ud\wedge S'G\ne\ud\).

Now consider the set of sequences \(\wp_S = \{ S^* | S \eqexp S^*\}\).
Because of our preconditions, the sequence \(break\) is not in
\(\wp_S\) (if it was, that is, \(S\eqexp break\), \(SG=\ud\) would hold
based on the soundness theorem).

Let \(S_0\) be (one of) the shortest sequence(s) in \(\wp_S\) and
similarly
\(S^\prime_0\) (one of) the shortest sequence(s) in \(\wp_{S^\prime}\). 
%
The following holds for these minimal sequences:
\[(SG\ne\ud\wedge S'G\ne\ud\implies) \quad SG=S'G=S_0G=S'_0G\ne\ud.\]
\begin{notrsi}
\emph{Proof.} Since 
\(SG\ne\ud\wedge S\eqexp S_0\implies SG=S_0G\) and
\(S'G\ne\ud\wedge S'\eqexp S'_0\implies S'G=S'_0G\); and since
\(SG\ne\ud\wedge S'G\ne\ud\) by assumption \(SG=S'G\ne\ud\) holds,
we have \(S_0G=SG=S'G=S'_0G\ne\ud\).

% As a special case, we also know that \(SF=S'F=S_0F=S'_0F\ne\ud\).

\medskip
\end{notrsi}
Now let us investigate these minimal sequences. 

\varsection{Investigating the command \emph{edit}}

We know that the command \(edit(\pi,\cdir{X})\) commutes or collapses
(i.e., a commuting or a simplifying law can be applied to it) with
every command on its left side (see Laws \lawi, \lawip, \lawiv,
\lawvp, \lawvip,
\lawiiiap, \lawxiii, \lawxviii, \lawxix, \lawxxiv, \lawxxvi, \lawiib).
%
We also know that \(edit(\pi,\cfile{X})\) does the same on its right side
(see Laws \lawi, \lawip, \lawiv, \lawv, \lawvi, \lawiia, \lawiiib, \lawx,
\lawxi, \lawxii, \lawxxiv, \lawxxv).
%
From Laws \lawi, \lawip~and \lawxxiv~we know that \(edit\)s commute
amongst each other.

Therefore in a minimal sequence all \(edit(\pi,\cdir{X})\) commands can be
moved to the beginning, and all \(edit(\pi,\cfile{X})\) commands to the
end of the sequence. Since they commute amongst themselves, we can order
the two groups alphabetically. Therefore if there were two commands on
the same path, they would be neighbors and would be simplified by the
algebraic laws. 
Therefore we can be sure that there are at most one
\(edit(\pi,\cfile{X})\) or \(edit(\pi,\cdir{X})\) command on each path.

This way we also separated these commands from the rest.
Now we can focus on the remaining part: on the \(create\) and 
\(remove\) commands.

\varsection{Investigating \emph{create} and \emph{remove}}
\label{theorem:crrm}

We will prove some lemmas about minimal sequences.
\begin{description}
\item[Lemma 0]
If a sequence is minimal, then it cannot have any pairs that match the
left-hand sides of Laws \lawx--\lawxxxv, \lawiia, \lawiiap, \lawiiib.
Also, if we apply any of the
commutative laws 
(\lawi, \lawip, \lawiib, \lawiiia, \lawiiiap, \lawiv--\lawix),
the resulting sequence is still minimal.

\emph{Proof.}
For the first part, the laws mentioned all give an equivalent shorter
sequence, but by hypothesis, there is no equivalent shorter sequence.
For the second part, the commutative laws don't change the length of a
sequence.

\item[Lemma 1] {\bf It is impossible that a command
\(remove(\pi)\) precedes (not necessarily as a neighbor) a command
\(remove(\piesub)\) in a minimal sequence.} 
\begin{notrsi}
(A parent can't be removed before any descendant, or
more precisely, \(\not\exists i,j: i<j \land S[i] = remove(\pi) \land
S[j] = remove(\pi') \land \pi \preceq \pi'\).)
\end{notrsi}
\begin{notrsi}
\begin{center}
\begin{tabular}{c|c|c|c|c|c|c|c|c}
\hline
\(\cdots\) & & \(S[i]: remove(\pi)\) & & & & &
\(S[j]: remove(\piesub)\) & \(\cdots\) \\
\hline
\end{tabular}
\end{center}
\end{notrsi}
\emph{Proof.}
By contradiction; we assume there is such an \(i\) and \(j\), and we show
that implies \(S \eqexp break\).

We show the contradiction by induction on \(j-i\).
The base case is \(j = i+1\).
In this case, by Laws \lawxxii~and~\lawxxiii, 
\(S[i]; S[i+1] \eqexp break\),
and therefore \(S\eqexp break\).

For the induction step, we perform a case analysis on \(S[j-1]\).
\begin{itemize}
\item
If it mentions a path that is disjoint with \(\piesub\),
%or if $\pi \prec \piesub$ and it is $edit(\pi, \cdir X)$,
we can swap it
with \(S[j]\) to get an equivalent sequence, and by the induction   
hypothesis and transitivity of equivalence, \(S\eqexp break\).
%\item
%By Lemma 0 (Laws~3B, 12, and~25), it cannot be any other edit operation.
\item
Also by Lemma 0 (Laws~\lawxvi, \lawxvii, and~\lawxxvii), it cannot be a
non-disjoint create operation. (Note that there are no \(edit\) operations
in this part of the sequence according to our preconditions.)
\item
If $S[j-1] =remove(\hat\pi)$, if $\hat\pi \preceq \piesub$, then by laws
\lawxxii~and~\lawxxiii, $S[j-1]; S[j] \equiv \{break\}$, and therefore 
$S\eqexp\{break\}$.
But if $\piesub \prec \hat\pi$, then by transitivity $\pi \preceq
\hat\pi$, so the induction hypothesis applies, and again 
$S \eqexp break$.
\end{itemize}

\item[Lemma 2]
{\bf A command \(create(\pi)\) can not precede a command  
\(remove(\piesub)\).} 
\emph{Proof.} 
This can be proved the same way, only
the base case differs; now 
\(create(\pi); remove({\piesub})\) equals \(break\) or expands 
to \(skip\) by Law \lawxvi~and \lawxxvii.
The induction step is the same.

\item[Lemma 3]
{\bf A command \(remove(\piesub)\) can not precede a command
\(create(\pi)\).} 
\emph{Proof.} This can be proved similarly, in this 
case the base step is
\(remove(\piesub); create({\pi})\) equals \(break\) or
expands to \(edit\) (Law \lawxxi~and \lawxxviii);
in the induction step we can prove that if
\(S[j-1]=create(\pisup)\) precedes \(S[j]=create(\pi)\), according to
the induction hypothesis, \(S\eqexp break\) 
(we know that \(\pisup\prec\pi\)). Otherwise it would be possible to
shorten the sequence (which contradicts our precondition) or commuted (and therefore the induction hypothesis
still holds). (Laws \lawvii,
\lawviii, \lawxiv, \lawxv, \lawxx, \lawxxi, \lawxxviii.)

\item[Lemma 4]
{\bf A command \(create(\piesub)\) can not precede a command
\(create(\pi)\).} 
\emph{Proof.} 
We obtain this result from Lemma 3: the base step is 
\(create(\piesub); create({\pi}) \equiv (\eqexp) break\) (Law 15); the
induction step is the same.

\item[Lemma 5]
{\bf A command \(create(\piesub)\) can not precede a command
\(remove(\pi)\).} 
\emph{Proof.} 
Our base step now is 
\(S[j-1];S[j]=create({\piesub}); remove(\pi)\eqexp skip\) or equals
\(break\) according to Laws \lawxxvii~and \lawxvii.
But in our induction step, we go in the opposite direction and perform
analysis on \(S[i+1]\).
If \(S[i+1]=create({\piesub}_{sub})\) follows \(S[i]=create(\piesub)\),
the induction hypothesis applies, therefore \(S\eqexp break\). Otherwise
the commands could be simplified or commuted (Laws \lawxviii, \lawxiv,
\lawxv, \lawxvi, \lawxvii, \lawxxvii.)

\item[Lemma 6] 
{\bf It is impossible that a command \(remove(\pi)\)
precedes a command \(create(\piesub)\).} 
\emph{Proof.} 
Now our base step is
\(remove(\pi); create(\piesub)\eqexp edit\) or equals to
\(break\) (Laws \lawxxviii~and \lawxx). In the induction step we
examine command \(S[i+1]\). 
If it's not commutable or simplifyable,
it can be only \(remove(\pisup)\).
In that case the induction hypothesis applies.
\end{description}

\varsection{More lemmas on commands}

Now we go back to \(edit\) commands. We will prove two additional lemmas.
\begin{description}
\item[Lemma E1] {\bf A command \(remove(\pi)\) cannot precede a command
\(edit(\pi,\cfile{X})\).} \emph{Proof.} 
By contradiction; we assume there are such two commands in the sequence,
\(S[i]=remove(\pi)\) and \(S[j]=edit(\pi,\cfile{X})\) where \(i<j\). We
use induction on \(j-i\). The base
step is when \(j-i=1\). Now, according to Law \lawxviii, \(S\equiv
break\) which contradicts our condition on \(S\).
For the induction step, we assume that \(S\eqexp break\) if
\(S[j-1]=edit(\pi,\cfile{X})\). Now we investigate \(S[j-1]\). If it is
not \(remove(\pisub)\), then a commuting (or simplifying) law can be
applied to \(S[j-1];S[j]\). That way
\(S[j-1]\) would be \(edit(\pi,\cfile{X})\), and according to the
induction hypothesis \(S\eqexp break\). If \(S[j-1]\) is
\(remove(\pisub)\), we have the same result by Lemma 1.

\item[Lemma E2] {\bf A command \(create(\pi)\) cannot precede a command 
\(edit(\pi)\).} 
\emph{Proof.} The same; last step is made according to
Lemma 2.

\item[Lemma E3] {\bf A command \(remove(\pi)\) cannot follow a
command \(edit(\pi,\cdir{X})\).} 
\emph{Proof.} 
This proof is also very similar to the aboves. The base step is 
\(edit(\pi,\cdir{X}); remove(\pi)\equiv remove(\pi)\) (Law \lawxxv). In the
induction step we investigate command \(S[i+1]\). If it was not
\(create(\pisub)\), a commuting (or simplifying) law could be applied. If
it is, we have \(S\eqexp break\) according to Lemma 5.

\item[Lemma E4] {\bf A command \(create(\pi)\) cannot follow a
command \(edit(\pi,\cdir{X})\).} 
\emph{Proof.} The same; last step
according to Lemma 4.
\end{description}

\noindent
Keeping in mind that we have all the \(edit\) commands at the ends of the
sequence, it follows from Lemma E1, E2, E3 and E4 that there cannot be an
\(edit\) and another command on the same path.

\begin{description}
\item[Lemma E5] {\bf A command \(edit(\pi,\cdir{X})\) cannot precede a law
\(edit(\pi,\cfile{X})\).} \emph{Proof.} Now we know that between these
commands there are no commands on path \(\pi\). We also know from Lemmas
1---6 that there is at most one \(remove\) or \(create\) command on each
path. Therefore
there cannot be a \(create(\pisub)\) command since we would have to
remove it before modifying \(\pi\) to a file.
Thus we could have moved \(edit(\pi,\cdir{X})\) to the end of the sequence
and simplify it with \(edit(\pi,\cfile{X})\).
\end{description}

\varsection{Conclusions}
\label{theorem:conc}

As a result of the lemmas we know that
\emph{there is at most one command on each path in a minimal sequence.}

Now we prove that \(C\in S_0\Longleftrightarrow C\in S'_0\) for any
command \(C\). (Without loss of generality, it is enough to prove that 
\(C\in S_0\implies C\in S'_0\).)
\begin{itemize}
\item \(create(\pi,X)\in S_0\implies create(\pi,X)\in S'_0\). 
\emph{Proof.} By contradiction. First of all, we know that
\(G(\pi)=\ud\) otherwise \(S_0\) would break \(G\) and that would
contradict our assumption. Now assume that 
\(create(\pi,X)\not\in S'_0\). 
We have three cases. If there is no command on path \(\pi\) in
\(S'_0\) then \(S'_0G(\pi)=\ud\) and therefore 
\(S_0G(\pi)\ne S'_0G(\pi)\), that is, \(S_0G\ne S'_0G\) which contradicts
our assumption. If there was a command \(edit\) or
\(remove\) on path \(\pi\), it would break \(G\) since \(G(\pi)=\ud\) and
that again contradicts the precondition.

\item \(remove(\pi)\in S_0\implies remove(\pi)\in S'_0\).
\emph{Proof.} 
Now we can be sure that \(F(\pi)\ne\ud\) and
\(S_0G(\pi)=\ud\). If (instead of
\(remove\)) there was no command on \(\pi\) in \(S'_0\) then 
\(S'_0G(\pi)\ne\ud\)  would hold. In case of \(edit\), 
\(S'_0G(\pi)\ne\ud\) would hold (contradicting
\(S_0G(\pi)=\ud\)). \(create\) would break \(G\) since
\(G(\pi)\ne\ud\). Again, we have contradiction in all cases.

\item \(edit(\pi,X)\in S_0\implies edit(\pi,X)\in S'_0\).
\emph{Proof.} Similarly we cannot have \(create\) or \(remove\) on path
\(\pi\) in sequence \(S'_0\) since \(G(\pi)\ne\ud\) and 
\(S'G(\pi)=SG(\pi)\ne\ud\). 
But now we might have no commands on
\(\pi\) in \(S'_0\): that way if \(G(\pi)=X\) then \(S_0G(\pi)=X\) and
\(S'_0G(\pi)=X\), too. 
% But clearly \(\neg(S_0\eqifnbr S'_0)\). 
To prove that this is also
impossible, we need the first condition about all filesystems. 

Define
\(H\) as \(G\{\pi\mapsto Y\}\) where \(Y\) has the same type
(file/directory) as \(G(\pi)\). If \(SG\ne\ud\) then \(SH\ne\ud\) since no
command breaks the filesystem because of the contents (not the type) of a
file or directory. Also, \(S'G\ne\ud\implies S'H\ne\ud\). Now, 
from the coditions of the theorem
and \(SH\ne\ud\wedge S'H\ne\ud\) we know
that \(S_0H=S'_0H\). But \(S_0H(\pi)=edit(\pi,X)H(\pi)=X\) and
\(S'_0H(\pi)=H(\pi)=Y\). We have a contradiction, therefore the lemma is
proved.
\end{itemize}

This means that
\emph{\(S_0\) and \(S^\prime_0\) contain the same commands;} they can be
different only in the order of the commands.

\medskip
Now we will show that they can be made exactly the same using the
commuting algebraic laws. 

We know that \(edit\)s can be moved to the ends of the sequences and
ordered. Let us suppose they are arranged this way.
Since we know that the two sequences contain the same
commands, these parts of the sequences is the same.
Therefore we can omit them in the discussion. Now we focus on
\(create\)s and \(remove\)s as we did above.

%%%\begin{notrsi}
If there are two commands in a minimal sequence
referring to comparable directories, according to the lemmas, they
can only be
%not be
%\begin{itemize}
%\item \(remove(\pi) \ldots remove(\pisub)\)
%\item \(remove(\pi) \ldots create(\pisub)\)
%\item \(create(\pisub) \ldots create(\pi)\)
%\item \(create(\pisub) \ldots remove(\pi)\)
%\item \(create(\pi) \ldots remove(\pisub)\)
%\item \(remove(\pisub) \ldots create(\pi)\).
%\end{itemize}
%The remaining possibilities are:
\begin{itemize}
\item \(create(\pi) \ldots create(\pisub)\)
\item \(remove(\pisub) \ldots remove(\pi)\).
\end{itemize}

Therefore any two \(create\) or \(remove\) commands in a (minimal)
sequence can be freely interchanged if they refer to incomparable
directories, but they have a well--defined order otherwise. (Keep in
mind that we have no \(edit\) commands in this part of the sequences.)
That is, we have a partial order over the set of these commands. Note
that we cannot have cycles in this order (since that way \(\pi\) would 
be equal to one of its descendants). Because of this, there is always
a minimal element, which has no predecessors. Also, this ordering is the
same on both sequences. 
%%%\end{notrsi}

We prove that the commands can be ordered the same way by induction on the
length of the sequences. If it
is \(1\), the problem can be solved easily. If the length of the sequences
is \(i>1\), we can choose a command from \(S_0\) which has no
command that must precede it.
It is sure that it has no precedents in \(S'_0\). Now we
can move this command to the front of the sequences. Now the rest of
the sequences have length \(i-1\) therefore we can order them according to
the induction hypothesis.

We obtain \(S_0^*\) and \({S^\prime}_0^*\) by applying this method to \(S_0\) and
\(S^\prime_0\). Since they contain the same
commands, and the resulting order is the same,
\(S_0^*\) and
\({S^\prime_0}^*\) are \emph{exactly the same}. Now we have
\(S\eqexp S_0\eqexp S_0^*\equiv{S^\prime_0}^*\eqshr S^\prime_0\eqshr
S^\prime\), that is, \(S\eqifnbr S^\prime\). Our theorem is proved.


