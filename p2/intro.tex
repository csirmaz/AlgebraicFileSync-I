
\section{Introduction}

Synchronization of data structures
is a mechanism behind many services we take for granted today:
accessing and editing calendar events, documents and spreadsheets
on multiple devices, version control systems and
geographically distributed internet and web services that
guarantee a fast and responsive user experience.

In this paper we investigate a command-based approach
to synchronizing filesystem trees.
Presented with multiple copies (replicas) of the same filesystem that have been independently modified,
we regard the aim of the synchronizer to modify these replicas further
to make them as similar as possible.
The synchronization algorithm we describe
follows the two main steps described by Balasubramaniam and Pierce \cite{BP}:
update detection, where we identify modifications that have been applied to the replicas
since they diverged and represent them as sequences of commands;
and reconciliation, where we identify commands that can be propagated to other replicas and do so.
The remaining commands, which could not be propagated, represent conflicting updates;
resolving these requires a separate system or human intervention.

We extend significantly the results of the previous work done
with Prof. Norman Ramsey
on an algebraic approach to synchronization
\cite{NREC},
and add to the theoretical understanding
of synchronization by providing rigorous proofs that the algorithms
we describe work as intended
and offer a solution in our framework that cannot be improved on.

A central problem of command-based synchronizers 
during both update detection and reconciliation
is the ordering (scheduling) of commands.
If update detection is based on comparing the original state of the filesystem
to its new state, then we can easily collect a set of updates,
but we need to order these in a way that they could be applied
to the filesystem without causing an error 
(for example, a directory needs to be created before a file can be created under it).
Similarly, during reconciliation when we consider all the commands that
have been applied to the different replicas, we need to find a way of
merging these and creating a global ordering that we can apply
to the filesystems.
In fact, an insight in \cite{NREC} is that
if the ordering changes the effect of two commands,
then under certain circumstances they are not compatible and will give rise to conflicts.
Accordingly, in this paper command pairs, their commutativity and other properties
play a special role.

In terms of the usual classification of synchronizers \cite{TSR, PV, SSH},
our approach to reconciliation is therefore operation-based as we investigate sets of commands,
while the update detection algorithm makes the full synchronizer 
similar to state-based synchronizers 
as its starting point is the states of the replicas themselves.
Moreover, our ordering algorithms only depend on properties of commands 
such as commutativity or idempotence,
and therefore they can be classified as semantic schedulers.
This is in opposition to syntactic scheduling, which uses 
the origin and the original temporal order of updates, 
and which may give rise to otherwise resolvable conflicts \cite{SSH}.
In general, ordering plays a central role in operation-based synchronizers.
We refer to an excellent survey by Saito and Shapiro of so-called optimistic replication algorithms \cite{SSH};
and, as examples,
to IceCube, where multiple orders are tested to find an acceptable one (\cite{KRSD}, see also \cite{MPV}),
or Bayou, where reconciling updates happens by redoing them in a globally determined order \cite{TTPDSH}.

The approach presented in this paper offers an improvement
over results desribed in \cite{NREC} in multiple ways.
We introduce a new set of commands that is symmetric and
captures more information about the updates,
and we restrict information content in directories
to exploit a further hidden symmetry in the command set.
These have the effect that reasoning about commands
becomes simpler as there are fewer edge cases,
and more powerful as 
predictions made by the reasoning are more accurate
due to the additional information content.
In fact, the new command set not only simplifies
our reconciliation algorithm, but also makes it maximal.
\cite{NREC} also lacked proofs that the update detection
and reconciliation algorithms it presented work as intended,
which we provide in the current paper.
While the results are intuitive, providing rigorous proofs
is far from trivial.
During the process, we define a number of useful concepts
and show their relationships and the properties they possess.
In our view these construct a special algebraic model
that is worthy of interest and of further research on its own.

The paper is organised as follows.
We start by defining a model of filesystems and a set of commands
that we will use to describe updates and modifications in \cref{sec_def}.
This is followed by investigating the properties and behaviors
of command pairs in \cref{section_axioms}.
\Cref{sec_update} describes update detection and ordering filesystem commands.
The main result of this section if that under some simple conditions,
a set of filesystem commands executed in any feasible order leads
to the same changes in the filesystem.
The reconciliation and conflict detection algorithm is defined in
\cref{sec_rec}, which merges two sets of commands.
We then proceed to prove that the output of reconciliation
is correct in the sense that 
it can be applied to the replicas without causing an error,
and it is maximal
inasmuch as no further updates can be applied under any circumstances.
In order to be able to do this, we introduce
domains of sets of command sequences,
and show that it has a number of highly convenient properties.
\Cref{sec_relatedwork,sec_algebra,sec_conclusion}
discuss related work, comparing our results to other research,
outlining directions for further work
on the introduced algebraic system, and provide the conclusion.
\Cref{axiom_proof} contains technical proofs for propositions on the behavior of command pairs.

