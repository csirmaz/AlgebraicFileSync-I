
\subsection{The Correctness of Reconciliation}

We are now ready to prove that the proposed algorithm for reconciliation is correct,
that is, applying its result is not going to break the replicas.
We reformulate the original proposition, $\reca\aFS_B\neq\fsbroken$,
based on $\FS_B=B\aFS$ and so we aim to prove that
$B\cc\reca$ works wherever $A$ and $B$ work:

\begin{myth}
If $A, B$ are simple, then $\worksc{B\cc \reca}{A,B}$,
where, to restate \cref{def:reconciliation},
% { % scope
% \renewcommand{\themydef}{\ref{def:reconciliation}}
% \begin{mydef}
\begin{align*}
\reca = \{\alpha|{}& \\
& \alpha\in A\setminus B \\
& \wedge \\
& \forall \beta\in B\setminus A: \alpha\indep\beta\\
\}. &
\end{align*}
% \end{mydef}
% } % scope
\end{myth}

This is trivial unless $\works{A,B}$, so we assume that it is true.
Let us first investigate the part of $A$ and $B$ that is excluded from
$\reca$: their intersection.

\begin{mylem}\label{can_move_intersection}
Let $A$ and $B$ be two simple sequences so that $\works{A, B}$.
Then their intersection, in some suitable order, also works on any filesystem
either of them works on:
$\worksc{\acbnp}{A}$, and, by symmetry,
$\worksc{\acbnp}{B}$.
\end{mylem}

\begin{proof}
The difficulty is that commands in $\acbnp$ can occur anywhere in $A$.
Therefore, first we prove the following:
if all commands are marked in $A$ that also appear in $B$,
then there is an $A'$ in $\orderset{A}$ in which all marked commands are at the beginning.

We show that $A$ can be transformed via equivalences into $A'$ for which it is true.
Our proposition is that if the marked commands are not at the beginning, then
the sequence contains an unmarked command followed by a marked command.
We show that these can be swapped resulting in an equivalent sequence.
Then, by repeating this process similarly to bubble sorting, we can generate 
a suitable $A'$.

Let us consider therefore the marked command preceded by an unmarked command in $A$,
and let the marked command be $\cxynv$, 
and the preceding unmarked command be $\czwmv$:
\begin{gather*}
A = \cdots\cc  \czwmv\cc  \cxynv\cc  \cdots
\end{gather*}
As $A$ is simple and $\works{A}$, from 
\cref{ax_distantrel_breaks,ax_directchild_breaks,ax_directparent_breaks}
we know that these commands can only be on incomparable nodes or form a construction or destruction pair.
In the first case, swapping the commands results in a sequence equivalent to $A$,
and we show that the last two cases are impossible as they would lead to contradiction.

In these cases, either $m=\parent(n)$ or $n=\parent(m)$.
If $B$ has a command on $m$, then
from \cref{lemma:neighbor}
we know that there is a $B'$ in $\orderset{B}$ where it is next to $\cxynv$ (also in $B$),
and from \cref{simple_reorder_equiv} we know that $B'\equiv B$.
When describing our ordering algorithm in \cref{ordering} we saw
that the command on the parent path determines whether a pair of commands
is a construction or destruction pair,
and that this, in turn, determines whether the command on the child path must
precede or follow the other command as otherwise the sequence would break all filesystems.
This argument holds for both $A$ and $B$, and so the command on $m$ must be on
the same side of the command on $n$ in both sequences.

If $B$ has no command on $m$, then let $B'$ be $B$, 
but with an extra command added to it just before $\cxynv$
according to \cref{ax_child_assert,ax_parent_assert}, 
from which we know that $B'\equiv B$.
We also know that $B'$ is still minimal (but no longer simple).

In either case, therefore,
we have a minimal $B'$ which has a command on $m$ just before $\cxynv$ and $B'\equiv B$.
Let this command on $m$ be $\cqrmv$.
\begin{gather*}
A = \cdots\cc  \czwmv\cc  \cxynv\cc  \cdots \\
B' = \cdots\cc  \cqrmv\cc  \cxynv\cc  \cdots
\end{gather*}

As $\works{A,B}$ and so $\works{A,B'}$, 
from \cref{worksinputmatch}
we know that $Q=Z$. 

Going back to the construction and destruction pairs we see that the output type of the first command
is always the same, and it is either $\cchard$ or $\ccharb$ depending on the relationship between $n$ and $m$.
Therefore $R=W$ and $\cqrmv=\czwmv$
as $|\setd|=|\setb|=1$. 
If $B$ originally had a command on $m$,
this is a contradiction as $\czwmv$ was not marked, but we see it must also be in $B'$ and therefore in $B$.
If $B$ had no command on $m$,
this is a contradiction because the command on $m$ in $B'$ is an assertion command, so $Z=Q=R=W$, 
but $A$ contained no assertion commands.

We now know that there is an $A'\equiv A$ in which commands in $\acbnp$
are at the beginning, and therefore 
from \cref{worksextpostfix} we know that
$\worksc{\acbnp}{A}$ and by symmetry $\worksc{\acbnp}{B}$.
\end{proof}

\bigskip

\noindent
We would like to prove that $\worksc{B\cc \reca}{A,B}$.
From the above we know that we can move commands in $\acbnp$
to the beginning of $B$ and so this is equivalent to
\[ \worksc{\acb\cc \bma\cc \reca}{A,B}. \]
As we already know that $\reca\indep\bma$
and that $\worksc{\acb\cc \bma}{A,B}$
(as $\worksc{B}{A,B}$),
we can prove our theorem based on \cref{indep_prefix_combine}
if we can prove that
\[ \worksc{\acb\cc \reca}{A,B}. \]
Using \cref{r_invmove} we can rephrase this as
\[ \worksc{\reca}{\acbi\cc A,\acbi\cc B}. \]

To restate some results above, we therefore already know the following things:
\begin{itemize}
\item $\amb$ and $\bma$ are simple as $A$ and $B$ are simple
\item $\amb \cap \bma = \emptyset$
\item $\reca$ is the largest subset of $\amb$ for which $\reca\indep\amb$
\item $\worksc{\amb, \bma}{\acbi\cc A,\acbi\cc B}$ 
because of \cref{r_invmove}
as $\worksc{A,B}{A,B}$,
and $A\equiv\acb\cc\amb$, and $B\equiv\acb\cc\bma$.
\end{itemize}

Our theorem is therefore proven if we prove the following lemma,
in which we rename $\ambnp$ to $S$ and $\bmanp$ to $T$:
\newcommand{\condSimple}{(c1)}
\newcommand{\condDisj}{(c2)}
\newcommand{\condApr}{(c3)}
\newcommand{\condWork}{(c4)}
\begin{mylem}
If
   \begin{itemize}
   \item[\condSimple] $S$ and $T$ are simple sequences,
   \item[\condDisj] $S\cap T=\emptyset$,
   \item[\condApr] $S^*$ is the largest subset of $S$ where $S^*\indep T$, and
   \item[\condWork] $\worksc{S,T}{\seqset{C}}$ for some set of sequences $\seqset{C}$,
   \end{itemize}
then
\[ \worksc{S^*}{\seqset{C}}. \]
\end{mylem}

\begin{proof}
This is trivial unless $\works{\seqset{C}}$, so we will assume that it is true
and therefore $\works{S,T}$.

The proof is similar to that of \cref{can_move_intersection}.
We mark all commands in $S$ that are in its subset $S^*$, and
we prove that $S$ can be transformed into $S'$ via equivalences
so that all marked commands would be at its beginning.
If so, then 
$\worksc{S^*}{\seqset{C}}$ 
based on \cref{worksextpostfix}
as $\worksc{S'}{\seqset{C}}$ and $S'=S^*\cc S_0$ 
where $S_0$ contains the remaining (conflicting) commands.

Again we know that if $S$ does not already have all marked commands at its beginning,
then there is an unmarked command followed by a marked one.
We show that these commands are independent and so they can be swapped
resulting in an equivalent sequence.
By repeating this process we can generate a suitable $S'$.

Let therefore the marked command in $S$ be $\cxynv$
and the preceding unmarked command be $\czwmv$.
As $S$ is simple and $\works{S}$, from 
\cref{ax_distantrel_breaks,ax_directchild_breaks,ax_directparent_breaks}
we know that these commands can only be on incomparable nodes or form a construction or destruction pair.
In the first case, swapping the commands results in a sequence equivalent to $S$,
and we show that the other two cases are not possible as they would lead to a contradiction.

In the last two cases, we know that either $m=\parent(n)$ or $n=\parent(m)$.
We also know that because of {\condApr} there must be 
a command $\cqrov$ in $T$ which is not independent of $\czwmv$
as $\czwmv$ is not part of $S^*$.
\begin{gather*}
S = \cdots\cc  \czwmv\cc  \cxynv\cc  \cdots \\
T = \cdots\cc  \cqrov\cc \cdots
\end{gather*}
We know none of these commands is an assertion command, and 
from {\condDisj} that $\cqrov\neq\czwmv$.
Therefore, based on \cref{incomparable_is_independent} we know this means that
either $m\descendantEq o$ or $o\descendantEq m$.
From {\condApr} we also know that $\cxynv\indep \cqrov$,
and so because of \cref{incomparable_is_independent},
$n\unrel o$.

We therefore have four cases considering the relationships between $n,m$ and $o$:
\begin{itemize}
\item $n=\parent(m) \wedge o\descendantEq m$.
   This would mean that $o\descendantEq n$ or $n=\parent(o)$, which contradicts $n\unrel o$.
\item $n=\parent(m) \wedge m\descendantEq o$.
   This would mean that $n\descendant o$, which contradicts $n\unrel o$.
\item $m=\parent(n) \wedge o\descendantEq m$.
   This would mean that $o\descendant n$, which contradicts $n\unrel o$.
\item $m=\parent(n) \wedge m\descendantEq o$.
   Let us continue with this case.
\end{itemize}

We know that $T$ cannot have a command on $m$ as $m=\parent(n)$ and $\cxynv\indep T$,
and so $m\neq o$ and therefore $m\descendant o$.
This means we can create a new sequence $T'\equiv T$ by inserting a command on 
$m$ into $T$ before $\cqrov$
according to \cref{ax_parent_assert}.
We now have $\cxynv\indep T\equiv T'$, which is a contradiction as
$T'$ has a command on $m$ and $m=\parent(n)$.
\end{proof}
