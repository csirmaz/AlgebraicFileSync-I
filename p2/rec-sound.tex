
\subsection{The Correctness of Reconciliation}

We are now ready to prove that the proposed algorithm for reconciliation is correct,
that is, applying its result is not going to break the replicas.
We reformulate the original proposition ($\reca\aFS_B\neq\fsbroken$)
based on $\FS_B=B\aFS$, and so we aim to prove that
$B\cc\reca$ is defined wherever $A$ and $B$ are defined:

\begin{myth}\label{reconciliation_correct}
If $A$ and $B$ are simple, then $\worksmbx{A,B}{B\cc \reca}$,
where, to restate \cref{def:reconciliation},
\[ \reca = \ords{\alpha \whr \alpha\in \ambnp  \ \wedge\   \alpha\indep \bmanp }. \]
\end{myth}

This is trivial unless $\worksmnilb{A,B}$, so we assume that it is true.
Let us first investigate the part of $A$ and $B$ that is excluded from
$\reca$: their intersection.

\begin{mylem}\label{can_move_intersection}
Let $A$ and $B$ be two simple sequences for which $\worksmnilb{A, B}$.
Then $A$ can be separated into their intersection and the remaining commands:
\[ \acb\cc\amb \in \orderset{A}. \]
\end{mylem}

\begin{proof}
We prove that
if all commands are marked in $A$ that also appear in $B$,
then there is an $A'$ in $\orderset{A}$ in which all marked commands are at the beginning.

We show that $A$ can be transformed into $A'$ for which this is true.
We know that if the marked commands are not at the beginning, then
the sequence contains an unmarked command followed by a marked command.
We show that these can be swapped resulting in an equivalent sequence.
Then, by repeating this process similarly to bubble sorting, we arrive at 
a suitable permutation of $A$, which is also equivalent to $A$, and therefore in $\orderset{A}$.

Let us consider therefore the marked command preceded by an unmarked command in $A$,
and let the marked command be $\cxynv$, 
and the preceding unmarked command be $\czwmv$:
\begin{gather*}
A = \cdots\cc  \czwmv\cc  \cxynv\cc  \cdots
\end{gather*}
As $A$ is simple and $\wrks{A}$, from 
\cref{ax_distantrel_breaks,simple_distant_pairs}
we know that these commands can only be on incomparable nodes or form a construction or destruction pair.
In the first case, swapping the commands results in a sequence equivalent to $A$,
and we show that the last two cases are impossible as they would lead to contradiction.

In the cases then $\czwmv\cc\cxynv$ is either a construction or a destruction pair,
and, depending on which is the case,
$m=\parent(n)$ or $n=\parent(m)$, respectively.

If $B$ has a command on $m$, let it be $\cqrmv$.
As \cref{simple_distant_pairs} applies to $B$, we know that
$\cxynv$ (which is in $B$), and $\cqrmv$ must also form a construction
or destruction pair.
As $\cxynv$ cannot be both a construction and a destruction command,
this pair must be of the same type as $\czwmv\cc\cxynv$,
and as the relationship between $n$ and $m$ is given,
we know $\cqrmv$ must precede $\cxynv$ in $B$.

Because the output value of the first command in construction
and destruction pairs is determined by the type of the pair,
$r=w$.
Also, as $\worksmnilb{A,B}$, 
from \cref{worksinputmatch}
we know that $[q]=[z]$.
Therefore $\cqrmv=\czwmv$,
which is a contradiction as $\czwmv$ was not marked,
but we see it must also be in $B$.

The last remaining case is when there is no command on $m$ in $B$.
Let $B'$ be $B$
with an extra assertion command added just before $\cxynv$
according to \cref{ax_child_assert,ax_parent_assert}, 
from which we know that $B'\equiv B$.
Let the new command be $\cqrmv$.
As $B'$ is still minimal (but no longer simple),
the argument above applies, and we again know
that $[q]=[z]$ and $r=w$.
As $\cqrmv$ is an assertion command, $[q]=[r]$ also holds,
from which $[z]=[w]$ (as $[z]=[q]=[r]=[w]$), which is a contradiction
as $A$ contains no assertion commands.
\end{proof}

\bigskip

\noindent
We would like to prove that $\worksmbx{A,B}{B\cc \reca}$.
From \cref{can_move_intersection} above we know that we can move commands in $\acbnp$
to the beginning of $A$ and $B$, and so this is equivalent to
\[ \worksmbx{\acb\cc\amb, \acb\cc\bma}{\acb\cc \bma\cc \reca}, \]
which is equivalent to
\[ \worksmbx{\amb, \bma}{\bma\cc \reca}. \]

For ease of reading, let us rename
$\amb$ to $S$,
$\reca$ to $S^*$,
and $\bma$ to $T$.
In this notation, we intend to prove that
\[ \worksmbx{S,T}{T\cc S^*}. \]
We can do so if we can show that
\[ \worksmbx{S,T}{S^*}. \]

This is because 
$\worksmbx{S,T}{T}$ is trivially true, and this, combined with the above,
yields 
$ \worksmbb{S,T}{S^*,T}. $
As $\reca\indep\bma$, that is, $S^*\indep T$,
from \cref{combine_independent_sequences}
we know
$ \worksmbx{S^*,T}{T\cc S^*}, $
and as $\worksmeqsign$ is transitive, combining the two statements above
gives $\worksmbx{S,T}{T\cc S^*}$.

\bigskip

TODO TODO TODO TODO TODO

\noindent
To restate some results above, we therefore already know the following:
\begin{itemize}
\item $\amb$ and $\bma$ are simple as $A$ and $B$ are simple
\item $\amb \cap \bma = \emptyset$
\item $\reca$ is the largest subset of $\amb$ for which $\reca\indep\amb$
% TODO Inverse is used here - how/why?
\item $\worksmbb{\acbi\cc A,\acbi\cc B}{\amb, \bma}$ 
because of \cref{r_invmove}
as $\worksmbb{A,B}{A,B}$,
and $A\equiv\acb\cc\amb$, and $B\equiv\acb\cc\bma$.
TODO HOW IS THIS USED?
\end{itemize}

Our theorem is therefore proven if we prove the following lemma,
in which we rename $\amb$ to $S$, $\reca$ to $S^*$, and $\bma$ to $T$:
\newcommand{\condSimple}{(c1)}
\newcommand{\condDisj}{(c2)}
\newcommand{\condApr}{(c3)}
\newcommand{\condWork}{(c4)}
\begin{mylem}\label{reconciliation_correct_part}
If
   \begin{itemize}
   \item[\condSimple] $S$ and $T$ are simple sequences,
   \item[\condDisj] $S\cap T=\emptyset$,
   \item[\condApr] $S^*$ is the largest subset of $S$ where $S^*\indep T$, and
   \item[\condWork] $\worksmxb{\sqs{C}}{S,T}$ for some set of sequences $\sqs{C}$,
   \end{itemize}
then
\[ \worksm{\sqs{C}}{S^*}. \]
\end{mylem}

\begin{proof}
This is trivial unless $\worksmnil{\sqs{C}}$, so we will assume that it is true
and therefore $\worksmnilb{S,T}$.

The proof is similar to that of \cref{can_move_intersection}.
We mark all commands in $S$ that are in its subset $S^*$, and
we prove that $S$ can be transformed into $S'$
so that all marked commands would be at its beginning.
If so, then 
$\worksm{\sqs{C}}{S^*}$ 
based on \cref{worksextpostfix}
as $\worksm{\sqs{C}}{S'}$ and $S'=S^*\cc S_0$ 
where $S_0$ contains the remaining (conflicting) commands.

Again we know that if $S$ does not already have all marked commands at its beginning,
then there is an unmarked command followed by a marked one.
We show that these commands are independent and so they can be swapped
resulting in an equivalent sequence.
By repeating this process we can generate a suitable $S'$.

Let therefore the marked command in $S$ be $\cxynv$
and the preceding unmarked command be $\czwmv$.
As $S$ is simple and $\wrks{S}$, from 
\cref{ax_distantrel_breaks,simple_distant_pairs}
we know that these commands can only be on incomparable nodes or form a construction or destruction pair.
In the first case, swapping the commands results in a sequence equivalent to $S$,
and we show that the other two cases are not possible as they would lead to a contradiction.

In the last two cases, we know that either $m=\parent(n)$ or $n=\parent(m)$.
We also know that because of {\condApr} there must be 
a command $\cqrov$ in $T$ which is not independent of $\czwmv$
as $\czwmv$ is not part of $S^*$.
\begin{gather*}
S = \cdots\cc  \czwmv\cc  \cxynv\cc  \cdots \\
T = \cdots\cc  \cqrov\cc \cdots
\end{gather*}
We know none of these commands is an assertion command, and 
from {\condDisj} that $\cqrov\neq\czwmv$.
Therefore, based on \cref{incomparable_is_independent} we know this means that
either $m\descendantEq o$ or $o\descendantEq m$.
From {\condApr} we also know that $\cxynv\indep \cqrov$,
and so because of \cref{incomparable_is_independent},
$n\unrel o$.

We therefore have four cases considering the relationships between $n,m$ and $o$:
\begin{itemize}
\item $n=\parent(m) \wedge o\descendantEq m$.
   This would mean that $o\descendantEq n$ or $n=\parent(o)$, which contradicts $n\unrel o$.
\item $n=\parent(m) \wedge m\descendantEq o$.
   This would mean that $n\descendant o$, which contradicts $n\unrel o$.
\item $m=\parent(n) \wedge o\descendantEq m$.
   This would mean that $o\descendant n$, which contradicts $n\unrel o$.
\item $m=\parent(n) \wedge m\descendantEq o$.
   Let us continue with this case.
\end{itemize}

We know that $T$ cannot have a command on $m$ as $m=\parent(n)$ and $\cxynv\indep T$,
and so $m\neq o$ and therefore $m\descendant o$.
This means we can create a new sequence $T'\equiv T$ by inserting a command on 
$m$ into $T$ before $\cqrov$
according to \cref{ax_parent_assert}.
We now have $\cxynv\indep T\equiv T'$, which is a contradiction as
$T'$ has a command on $m$ and $m=\parent(n)$.
\end{proof}
