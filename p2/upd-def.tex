
In a command-based reconciliation solution we assume that we have two sequences of commands
$A$ and $B$ that have modified a single filesystem $\FS$ yielding two different replicas $\FS_A$ and $\FS_B$ that we
need to reconcile. While it is conceivable that the sequences would be based on a record of
all operations that modified the two filesystems, in most filesystem implementations
such records are not implemented, and therefore we must construct suitable sequences
by comparing $\FS_A$ and $\FS_B$ to their common ancestor, $\FS$. This is called update detection.

A sequence that transforms $\FS$ into $\FS_A$ must contain at least one command on each node
that changes between the two states of the filesystem. Also, it is easy to see that this is also
sufficient, as there are commands available for any input and output value. 
Specifically, if at node $n$, $\FS(n)=\valvx$ and $\FS_A(n)=\valvy\neq \valvx$, then we add the command $\cxyaa{n}{\valvy}$ to $A$
where $\valvx\in\setvx{X}$ and $\valvy\in\setvx{Y}$.

To be able to describe such sequences better, let us introduce the following properties:

\begin{mydef}[Minimal set or sequence]
A sequence or set of commands is minimal if it contains at most one command on each node.
\end{mydef}

\begin{mydef}[Simple set or sequence]
A sequence or set of commands is simple if it is minimal and it does not contain assertion commands.
\end{mydef}

This update detector therefore yields a simple set of commands because we only add a single command
for each node, and we only add commands that are necessary, that is, there will be no 
assertion commands in the set.

The next step in generating the sequences is to order the commands collected.
As this task is at the heart of reconciliation itself independently of update detection,
we discuss it in the next section.
Then, in \cref{update_works}, we prove that the resulting sequence 
returned by the update detector actually works without breaking the filesystem.
