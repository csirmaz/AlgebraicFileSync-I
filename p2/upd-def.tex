
In a command-based reconciliation solution we assume that we have two sequences of commands
$A$ and $B$ that have modified a single filesystem $\FS$ 
yielding two different replicas $\FS_1$ and $\FS_2$ which we
need to reconcile. While it is conceivable that the sequences are based on a record of
all operations that modified the two filesystems, in most filesystem implementations
such records do not exist, and therefore we must construct suitable sequences
by comparing $\FS_1$ (and $\FS_2$) to their common ancestor, $\FS$. 
This is called \emph{update detection.}

The set of commands $U$ necessary to transform $\FS$ into $\FS_1$ can be collected
by inspecting the (finite) union of non-empty nodes in the two filesystems.
If at node $n$, $\FS(n)=x$ and $\FS_1(n)=y\neq x$, then we add the command $\cxyaa{n}$ to $U$.
We can always do so as there is a command available for all combinations of input and output types and values.
We see that any suitable $U$ necessarily contains commands on all nodes at which the values have changed.


% To be able to reason about sets and sequences of commands, let us introduce the following properties:

\begin{mydef}[Minimal and simple command sets and sequences]\mlabel{def_min_simp}
A sequence or set of commands is \emph{minimal} if it contains at most one command on each node.
It is \emph{simple} if it is minimal and it does not contain assertion commands.
\end{mydef}

This update detector therefore yields a simple set of commands because we only add a single command
for each node, and we only add commands that are necessary, that is, there will be no 
assertion commands in the set.

The next step in generating the sequences is to order the commands collected.
As this task is at the heart of reconciliation itself independently of update detection,
we discuss it in the next section.
Then, in \cref{update_works}, we prove that the resulting sequence 
returned by the update detector actually works without breaking the filesystem.
