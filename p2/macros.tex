
% TODO Remove debug
\newcommand{\mlabel}[1]{\label{#1}\marginpar{\tiny\texttt{[\detokenize{#1}]}}}

\newcommand{\typeeq}{\simeq} % Type equal ~_
\newcommand{\ntypeeq}{\not\typeeq} % Not type equal

\newcommand{\vald}{\mathtt{D}} % Directory value
\newcommand{\valdi}{\mathtt{D}_1}
\newcommand{\valdii}{\mathtt{D}_2}
\newcommand{\valf}{\mathtt{F}_i} % File value
\newcommand{\valff}{\mathtt{F}_j} % File value
\newcommand{\fil}{\mathtt{F}_0} % Selected file value for representative objects
\newcommand{\empt}{\ominus} % Empty value
\newcommand{\valv}{v} % Any value
\newcommand{\valvy}{::v_Y::} % TODO DELETE
\newcommand{\valvw}{::v_W::} % TODO DELETE

\newcommand{\setfs}{\mathbb{F}_\setn} % Set of filesystems
\newcommand{\setv}{\mathbb{V}} % Set of values
\newcommand{\setn}{\mathbb{N}} % Set of nodes / paths
\newcommand{\sets}{\mathbb{S}} % Set of sequences
\newcommand{\setssmp}{\sets_{\mathrm{simple}}} % Set of simple sequences
\newcommand{\setvx}[1]{::\setv_{#1}::} % Set of values with custom index % TODO DELETE
\newcommand{\setd}{[\vald]} % Set of directory values
\newcommand{\setb}{::\setvx{\empt}::} % Set of empty values TODO DELETE

\newcommand{\fsreplacement}[3]{{#1}_{[{#2}\rightarrow{#3}]}} % {FS}{node}{newvalue}

\newcommand{\parent}{\text{\normalfont\scshape{parent}}} % parent function symbol
\newcommand{\parentf}[1]{\parent(#1)} % parent function
\newcommand{\fsbroken}{\bot} % value when command is not defined / broken FS value
\newcommand{\FS}{\Phi} % a filesystem
\newcommand{\GS}{\Theta} % another filesystem
\newcommand{\aFS}{\,\FS} % a filesystem preceded by something applied to it
\newcommand{\aGS}{\,\GS} % a filesystem preceded by something applied to it
\newcommand{\nn}{n'} % parent node
\newcommand{\Tsign}{\tau} % type-equivalent class mapping
\newcommand{\T}[1]{{#1}_\Tsign} % type-equivalent class mapping
\newcommand{\TA}{\T{A}}
\newcommand{\TB}{\T{B}}
\newcommand{\TC}{\T{C}}
\newcommand{\TS}{\T{S}}

\newcommand{\cbrk}{c_\varnothing} % 'break' / empty command

\newcommand{\caa}[2]{\langle{[#1],[#2]}\rangle} % Command template
\newcommand{\caaa}[3]{\langle{[#1],#2,#3}\rangle}
\newcommand{\caaaa}[3]{\langle{[#1],#2,#3}\rangle}
\newcommand{\cTaaa}[3]{\langle{[{#1}],\T{#2},{#3}}\rangle} % Representative of command

\newcommand{\cbb}{\caa{\empt}{\empt}}
\newcommand{\cbba}[1]{\caaa{\empt}{\empt}{#1}}
\newcommand{\cbf}{\caa{\empt}{\valf}}
\newcommand{\cbfaa}[1]{\caaa{\empt}{\valf}{#1}}
\newcommand{\cbd}{\caa{\empt}{\vald}}
\newcommand{\cbdaa}[1]{\caaa{\empt}{\vald}{#1}}
\newcommand{\cfb}{\caa{\valf}{\empt}}
\newcommand{\cfba}[1]{\caaa{\valf}{\empt}{#1}}
\newcommand{\cff}{\caa{\valf}{\valf}}
\newcommand{\cfd}{\caa{\valf}{\vald}}
\newcommand{\cfdaa}[1]{\caaa{\valf}{\vald}{#1}}
\newcommand{\cdb}{\caa{\vald}{\empt}}
\newcommand{\cdba}[1]{\caaa{\vald}{\empt}{#1}}
\newcommand{\cdf}{\caa{\vald}{\valf}}
\newcommand{\cdfaa}[1]{\caaa{\vald}{\valf}{#1}}
\newcommand{\cdd}{\caa{\vald}{\vald}}
\newcommand{\cdda}[1]{\caaa{\vald}{\vald}{#1}}

\newcommand{\cxy}{\caa{x}{y}}
\newcommand{\cxyaa}[1]{\caaa{x}{y}{#1}}
\newcommand{\cxynv}{\caaa{x}{y}{n}}
\newcommand{\cxynnv}{\caaa{x}{y}{\nn}}
\newcommand{\cTxynv}{\cTaaa{x}{y}{n}}
\newcommand{\cTyxnv}{\cTaaa{y}{x}{n}}
\newcommand{\cxw}{\caa{x}{w}}
\newcommand{\cxwnv}{\caaa{x}{w}{n}}
\newcommand{\czw}{\caa{z}{w}}
\newcommand{\czwnv}{\caaa{z}{w}{n}}
\newcommand{\czwnnv}{\caaa{z}{w}{\nn}}
\newcommand{\czwmv}{\caaa{z}{w}{m}}
\newcommand{\cqr}{\caa{q}{r}}
\newcommand{\cqrnv}{\caaa{q}{r}{n}}
\newcommand{\cqrmv}{\caaa{q}{r}{m}}
\newcommand{\cqrov}{\caaa{q}{r}{o}}

\newcommand{\cc}{\circ} % Command / sequence concatenation
\newcommand{\descendant}{\prec}
\newcommand{\descendantEq}{\preceq}
\newcommand{\ancestor}{\succ}
\newcommand{\ancestorEq}{\succeq}

\newcommand{\eqext}{\sqsubseteq} % extended by <=
\newcommand{\eqnrw}{\sqsupseteq} % extends >=
\newcommand{\strext}{\sqsubset} % extended by (strict) <
\newcommand{\strnrw}{\sqsupset} % extends (strict) >
\newcommand{\nequiv}{\not\equiv}

\newcommand{\indep}{\mathrel{\wr\wr}} % Independent commands, sequences
\newcommand{\unrel}{\indep} % Incomparable (unrelated) nodes
\newcommand{\nindep}{\mathrel{\centernot{\wr\wr}}} % not \indep
\newcommand{\nunrel}{\nindep} % not \unrel

% \newcommand{\worksmsign}{\vartriangleleft} % works-more sign <
\newcommand{\worksmeqsign}{\trianglelefteq} % works-more-eq sign <=
\newcommand{\worksm}[2]{{#1}\worksmeqsign{#2}} % works-more-(eq)
\newcommand{\worksmxb}[2]{\worksm{#1}{\{{#2}\}}}
\newcommand{\worksmbx}[2]{\worksm{\{{#1}\}}{#2}}
\newcommand{\worksmbb}[2]{\worksm{\{{#1}\}}{\{{#2}\}}}

\newcommand{\wrks}[1]{\cbrk\strext{#1}} % works
\newcommand{\wrksx}[1]{{#1}\strnrw\cbrk} % works (reverse order)
\newcommand{\worksmnil}[1]{\worksm{\cbrk}{#1}} % works
\newcommand{\worksmnilb}[1]{\worksm{\cbrk}{\{{#1}\}}}

\newcommand{\emptyseq}{\lambda} % empty sequence

\newcommand{\ordersetsign}{\vec{P}}
\newcommand{\orderset}[1]{\ordersetsign({#1})}
\newcommand{\ordsign}{\vec{c}} % Set of commands with assumed order
\newcommand{\ords}[1]{\ordsign\{{#1}\}} % Set of commands with assumed order, {}
\newcommand{\ordp}[1]{\ordsign\,({#1})} % Set of commands with assumed order, ()

\newcommand{\seqset}[1]{\mathcal{#1}} % Set of sequences
\newcommand{\sqs}[1]{\seqset{#1}}

\newcommand{\recchar}[3]{{#1}^{#3}_{\mathcal{R}|{#2}}}
\newcommand{\reca}{\recchar{A}{B}{}} % Reconciled from A
\newcommand{\recb}{\recchar{B}{A}{}} % Reconciled from B

\newcommand{\mynegsp}{\nobreak\hspace{-1pt plus 1pt}\nobreak}
\newcommand{\acbnp}{A\mynegsp\cap\mynegsp B} % A^B
\newcommand{\acb}{\ordp{\acbnp}} % c(A^B)
\newcommand{\acbi}{{\T{\acb}}^{-1}} % c(A^B)T-1
\newcommand{\ambnp}{A\mynegsp\setminus\mynegsp B}
\newcommand{\amb}{\ordp{\ambnp}}
\newcommand{\bmanp}{B\mynegsp\setminus\mynegsp A}
\newcommand{\bma}{\ordp{\bmanp}}

\newcommand{\Dom}[1]{\textrm{Dom}({#1})}
\newcommand{\Img}[1]{\textrm{Img}({#1})}

\newcommand{\whr}{\mathrel{|}}

\newcommand{\rot}[3]{#3#2#1} % Rotate
\newcommand{\undersc}{\texttt{\detokenize{_}}}

\theoremstyle{definition}
\newtheorem{mydef}{Definition}
\crefname{mydef}{Definition}{Definitions}

\theoremstyle{definition}
\newtheorem{myax}{Rule}
\crefname{myax}{Rule}{Rules}

\theoremstyle{plain}
\newtheorem{mylem}{Lemma}
\crefname{mylem}{Lemma}{Lemmas}

\theoremstyle{plain}
\newtheorem{myth}{Theorem}
\crefname{myth}{Theorem}{Theorems}

\theoremstyle{plain}
\newtheorem{mycor}{Corollary}
\crefname{mycor}{Corollary}{Corollaries}

\theoremstyle{plain}
\newtheorem{myclm}{Claim}
\crefname{myclm}{Claim}{Claims}
