
\newcommand{\ccharb}{\empt} % character for empty type in commands
\newcommand{\ccharf}{\mathtt{F}} % Character for file type in commands
\newcommand{\cchard}{\mathtt{D}} % Character for dir type in commands

\newcommand{\typeset}{\{\cchard,\ccharf,\ccharb\}} % Set of types in commands
\newcommand{\rvalueset}{\{\vald,\fil,\empt\}} % Set of representative values

\newcommand{\setv}{\mathbb{V}} % Set of values
\newcommand{\setn}{\mathbb{N}} % Set of nodes / paths
\newcommand{\setvx}[1]{\setv_{#1}} % Set of values with custom index
\newcommand{\setf}{\setvx{\ccharf}} % Set of file values
\newcommand{\setd}{\setvx{\cchard}} % Set of directory values
\newcommand{\setb}{\setvx{\ccharb}} % Set of empty values

\newcommand{\vald}{d_0} % Directory value
\newcommand{\valdi}{d_1}
\newcommand{\valdii}{d_2}
\newcommand{\valf}{f} % File value
\newcommand{\fil}{f_0} % Selected file value for representative objects
\newcommand{\empt}{\bot} % Empty value
\newcommand{\valv}{v} % Any value
\newcommand{\valvx}{v_X} % Value of type X
\newcommand{\valvy}{v_Y} % Value of type Y
\newcommand{\valvw}{v_W}
\newcommand{\valvr}{v_R}

\newcommand{\parent}{\text{\normalfont\scshape{parent}}} % parent function
\newcommand{\topnode}{\text{\normalfont\scshape{none}}} % return value of parent()
\newcommand{\fsbroken}{\text{\normalfont\scshape{broken}}} % broken FS value
\newcommand{\FS}{\Phi} % a filesystem
\newcommand{\GS}{\Theta} % another filesystem
\newcommand{\aFS}{\,\FS} % a filesystem preceded by something applied to it
\newcommand{\aGS}{\,\GS} % a filesystem preceded by something applied to it
\newcommand{\nn}{n'} % parent node
\newcommand{\T}{\tau} % type-equivalent class mapping

\newcommand{\cbrk}{\mathtt{Break}} % 'break' command
\newcommand{\caa}[2]{\langle{#1,#2}\rangle} % Command template
\newcommand{\caaa}[3]{\langle{#1,#2,#3}\rangle}
\newcommand{\caaaa}[4]{\langle{#1,#2,#3,#4}\rangle}
\newcommand{\cbb}{\caa{\ccharb}{\ccharb}}
\newcommand{\cbba}[1]{\caaa{\ccharb}{\ccharb}{#1}}
\newcommand{\cbf}{\caa{\ccharb}{\ccharf}}
\newcommand{\cbfaa}[2]{\caaaa{\ccharb}{\ccharf}{#1}{#2}}
\newcommand{\cbd}{\caa{\ccharb}{\cchard}}
\newcommand{\cbdaa}[2]{\caaaa{\ccharb}{\cchard}{#1}{#2}}
\newcommand{\cfb}{\caa{\ccharf}{\ccharb}}
\newcommand{\cfba}[1]{\caaa{\ccharf}{\ccharb}{#1}}
\newcommand{\cff}{\caa{\ccharf}{\ccharf}}
\newcommand{\cfd}{\caa{\ccharf}{\cchard}}
\newcommand{\cfdaa}[2]{\caaaa{\ccharf}{\cchard}{#1}{#2}}
\newcommand{\cdb}{\caa{\cchard}{\ccharb}}
\newcommand{\cdba}[1]{\caaa{\cchard}{\ccharb}{#1}}
\newcommand{\cdf}{\caa{\cchard}{\ccharf}}
\newcommand{\cdfaa}[2]{\caaaa{\cchard}{\ccharf}{#1}{#2}}
\newcommand{\cdd}{\caa{\cchard}{\cchard}}
\newcommand{\cdda}[1]{\caaa{\cchard}{\cchard}{#1}}
\newcommand{\cxy}{\caa{X}{Y}}
\newcommand{\cxyaa}[2]{\caaaa{X}{Y}{#1}{#2}}
\newcommand{\cxynv}{\caaaa{X}{Y}{n}{\valvy}}
\newcommand{\cxynnv}{\caaaa{X}{Y}{\nn}{\valvy}}
\newcommand{\cxw}{\caa{X}{W}}
\newcommand{\cxwnv}{\caaaa{X}{W}{n}{\valvw}}
\newcommand{\czw}{\caa{Z}{W}}
\newcommand{\czwnv}{\caaaa{Z}{W}{n}{\valvw}}
\newcommand{\czwnnv}{\caaaa{Z}{W}{\nn}{\valvw}}
\newcommand{\czwmv}{\caaaa{Z}{W}{m}{\valvw}}
\newcommand{\cqr}{\caa{Q}{R}}
\newcommand{\cqrnv}{\caaaa{Q}{R}{n}{\valvr}}
\newcommand{\cqrmv}{\caaaa{Q}{R}{m}{\valvr}}
\newcommand{\cqrov}{\caaaa{Q}{R}{o}{\valvr}}

\newcommand{\cc}{\circ} % Command / sequence concatenation
\newcommand{\descendant}{\prec}
\newcommand{\descendantEq}{\preceq}
\newcommand{\ancestor}{\succ}
\newcommand{\ancestorEq}{\succeq}

\newcommand{\eqext}{\sqsubseteq} % extended by
\newcommand{\eqnrw}{\sqsupseteq} % extends
\newcommand{\nequiv}{\not\equiv}
\newcommand{\indep}{\mathrel{\wr\wr}} % Independent commands, sequences
\newcommand{\unrel}{\indep} % Incomparable (unrelated) nodes
\newcommand{\nindep}{\mathrel{\centernot{\wr\wr}}} % not \indep
\newcommand{\nunrel}{\nindep} % not \unrel

\newcommand{\workssign}{\mathbf{w}}
\newcommand{\works}[1]{\workssign({#1})} % works
\newcommand{\worksc}[2]{\workssign({#1}\mathrel{|}{#2})} % works on condition

\newcommand{\typeeq}{\simeq} % Type equality
\newcommand{\ntypeeq}{\not\typeeq}

\newcommand{\emptyseq}{\lambda} % empty sequence

\newcommand{\ordersetsign}{\vec{\wp}}
\newcommand{\orderset}[1]{\ordersetsign({#1})}
\newcommand{\seqset}[1]{\mathcal{#1}} % Set of sequences

\newcommand{\recchar}[3]{{#1}^{#3}_{\mathcal{R}|{#2}}}
\newcommand{\reca}{\recchar{A}{B}{}} % Reconciled from A
\newcommand{\recb}{\recchar{B}{A}{}} % Reconciled from B

\newcommand{\mynegsp}{\nobreak\hspace{-1pt plus 1pt}\nobreak}
\newcommand{\acbnp}{A\mynegsp\cap\mynegsp B}
\newcommand{\acb}{(\acbnp)}
\newcommand{\acbi}{\acb^{-1}}
\newcommand{\ambnp}{A\mynegsp\setminus\mynegsp B}
\newcommand{\amb}{(\ambnp)}
\newcommand{\bmanp}{B\mynegsp\setminus\mynegsp A}
\newcommand{\bma}{(\bmanp)}

\newcommand{\rot}[3]{#3#2#1}
\newcommand{\undersc}{\texttt{\detokenize{_}}}

\theoremstyle{definition}
\newtheorem{mydef}{Definition}
\crefname{mydef}{Definition}{Definitions}

\theoremstyle{definition}
\newtheorem{myax}{Rule}
\crefname{myax}{Rule}{Rules}

\theoremstyle{plain}
\newtheorem{mylem}{Lemma}
\crefname{mylem}{Lemma}{Lemmas}

\theoremstyle{plain}
\newtheorem{myth}{Theorem}
\crefname{myth}{Theorem}{Theorems}

\theoremstyle{plain}
\newtheorem{mycor}{Corollary}
\crefname{mycor}{Corollary}{Corollaries}
