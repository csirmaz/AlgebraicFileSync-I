
\subsection{The Correctness of Update Detection}

With $\ordersetsign$ we can now define our update detection algorithm.
Its inputs are the original filesystem $\FS$ which has been modified to yield $\FS^*$.
\begin{mydef}[Update detection]\mlabel{def_upddetect}~
\begin{enumerate}
    \item For each node $n$ where the value in $\FS$ and $\FS^*$ differ, add the command $\caaa{\FS(n)}{\FS^*(n)}{n}$ to
        a set of commands $U_S$. The result is a simple set of commands.
    \item Order the commands according to $\orderrel$, that is, 
        return any sequence $U$ from $\orderset{U_S}$.
\end{enumerate}
\end{mydef}

\Cref{update_works} proves that $U$ functions as expected, that is, $U\aFS=\FS^*$.
While this is trivial if we know that $U$ does not break $\FS$, we
still need to show that it is defined on $\FS$.

On systems where filesystem updates are recorded,
an alternative update detection algorithm is to
omit step 1, and use \cref{can_simplify} to simplify the series of recorded updates
into a simple sequence containing the necessary updates.
In fact, we use such a hypothetical record in the proof of \cref{update_works}.

% There is a sequence between any two states: destroy one completely and construct the other one. (finite FS)

\begin{myth}\mlabel{update_works}
For a simple sequence of commands $U$ returned by the update detector
when comparing the non-broken $\FS^*$ to the original $\FS$,
$U\aFS = \FS^*$.
\end{myth}

\begin{proof}
Let us assume that some mechanism recorded all changes that occurred
to $\FS$ until it reached the state $\FS^*$.
Our set of commands is sufficient to record any change, as
there are commands for every input and output type pair, therefore
we know that the changes can be recorded as a sequence of commands $S$
where $S\aFS=\FS^*$.
($S$ will not contain assertion commands, as they do not
represent an actual change in the filesystem.)
Based on \cref{can_simplify}, there is a simple sequence $S^*$
that is equivalent to $S$.

Therefore we know $S^*\aFS=\FS^*$, and that
$\FS^*$ and $\FS$ differ at exactly those nodes that $S^*$ has commands on.
$U$ must also contain commands on the same set of nodes,
and it must contain the same commands,
as for the command $\cxynv$, $[x]$ must be $[\FS(n)]$, and $y$ must be $\FS^*(n)$.
From this we know that $U$ is a permutation of $S^*$ and so $\orderset{U}=\orderset{S^*}$.

As the update detector uses $\orderrel$ to order the commands in $U$, trivially $U\in\orderset{U}$,
which means $U\in\orderset{S^*}$,
and so from \cref{simple_reorder_equiv} we know that $U\equiv S^*$,
from which $U\aFS=\FS^*$.
\end{proof}
