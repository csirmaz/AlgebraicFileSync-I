
\subsection{The Correctness of Update Detection}

Based on \cref{connected_changes} we can also show that
the update detector does function as intended as:

% TODO
TODO ENSURE UPDATE DETECTOR IS DEFINED.

TODO What is not trivial here is that $U$ works at all.

\begin{myth}\mlabel{update_works}
For a sequence of commands $U$ returned by the update detector
when comparing the non-broken $\FS^*$ to the original $\FS$,
$U\aFS = \FS^*$.
\end{myth}
\begin{proof}
Let $\FS$ be the original filesystem, and $\FS^*$
be the filesystem after the changes we intend to detect.
Let us assume that some mechanism recorded all changes that occurred
to $\FS$ until it reached the state $\FS^*$.
Our set of commands is sufficient to record any change, as
there are commands for every input and output type pair, therefore
we know that the changes can be recorded as a sequence of commands $S$
where $S\aFS=\FS^*$.
Note that $S$ will not contain assertion commands, as they do not
represent an actual change in the filesystem.

\newcommand{\ucx}{\alpha}
\newcommand{\ucy}{\beta}
\newcommand{\ucz}{\gamma}

We aim to find a simple sequence that is equivalent to $S$.
If $S$ is already simple, there is nothing to do.
If it is not simple, let $\ucx$ be the first command
on a node $n$ that is already present in $S$,
and let $\ucy$ be the previous command on $n$.
Splitting $S$ around these commands we get
\[ S = S_0 \cc \ucy \cc S_1 \cc \ucx \cc S_2. \]

We now show that $S_1\indep\ucx$,
which is equivalent to $S_1\indep\ucy$ as both commands are on $n$.
We use an inverse proof and assume that there is a command on $m$ in $S_1$
where $m\descendant n$ or $m\ancestor n$.
We know both $\ucy\cc S_1$ and $S_1\cc\ucx$ are simple,
and both are defined on a filesystem
(on $S_0\aFS$ and $(S_0\cc\beta)\FS$, respectively).
Therefore from \cref{connected_changes} we know that
$S_1$ must also contain a command on $m'$ where
$m'=\parentf{n}$ or $n=\parentf{m'}$.
Let this command be $\ucz$.

If $n=\parentf{m'}$,
from \cref{simple_distant_pairs} 
and the relationship of nodes in construction and destruction pairs
we know that the subsequence
$\ucy\cc\ucz$ must be a construction pair,
and both $\ucy$ and $\ucz$ must be construction commands.
But we also know that $\ucz\cc\ucx$ must be a destruction pair,
and both $\ucz$ and $\ucx$ must be destruction commands.
This is a contradiction as a command cannot be both.
We arrive at the same contradiction if $m'=\parentf{n}$.

We therefore know that $S_1\indep\ucx$ and so
from \cref{ax_separate_commute}
\[ S \equiv S_0\cc\ucy\cc\ucx\cc S_1 \cc S_2. \]
We can apply \cref{ax_same_emptyseq,ax_same_singlec} to
$\ucy\cc\ucx$ and get a $S'\eqext S$ that contains one or two less commands on $n$.
Repeating the swap and the simplification we
can convert $S$ into an equivalent simple sequence.
Let this sequence be $S^*$.

As $S^*\equiv S$, we know $S^*\aFS=\FS^*$, and that
$\FS^*$ and $\FS$ differ at exactly those nodes that $S^*$ has commands on.
If $U$ is the simple sequence returned by the update detector
when comparing $\FS^*$ to $\FS$, it must also contain commands on the same set of nodes,
and it must contain the same commands,
as for the command $\cxynv$, $[x]$ must be $[\FS(n)]$, and $y$ must be $\FS^*(n)$.
From this we know that $U$ is a permutation of $S^*$ and so $\orderset{U}=\orderset{S^*}$.

As the update detector uses $\ordersetsign$ to order the commands in $U$,
trivially $U\in\orderset{U}$,
and as $S^*$ is defined on $\FS$,
from \cref{order_is_only_possible} we know $S^* \in \orderset{S^*}$.
This means $U\in\orderset{S^*}$,
and so from \cref{simple_reorder_equiv} we know that $U\equiv S^*$,
from which $U\aFS=\FS^*$.
\end{proof}

An important consequence of this proof is that any sequence
of recorded updates that were applied to a filesystem can be
converted to a simple sequence using the rules listed in
\cref{section_axioms} alone.
