
\subsection{The Correctness of Update Detection}

With the relation $\orderrel$ we can now define our update detection algorithm.
Its inputs are the original filesystem $\FS$ which has been modified to yield $\FS^*$.
\begin{mydef}[Update detection]\mlabel{def_upddetect}~
\begin{enumerate}
    \item For each node $n$ where the value in $\FS$ and $\FS^*$ differ, add the command $\caaa{\FS(n)}{\FS^*(n)}{n}$ to
        a set of commands $U^*$. The result is a simple set of commands.
    \item Order the commands according to $\orderrel$, that is, 
        return any sequence $U$ from $\orderset{U^*}$. \qedhere
\end{enumerate}
\end{mydef}

\Cref{update_works} below proves that $U$ functions as expected, that is, $U\aFS=\FS^*$.
While this is trivial if we know that $U$ does not break $\FS$, we
still need to show that it is defined on $\FS$.

On systems where filesystem updates are recorded,
an alternative update detection algorithm is to
omit step 1, and use \cref{can_simplify} to simplify the series of recorded updates
into a simple sequence containing the necessary updates.
In fact, we use such a hypothetical record in the proof of \cref{update_works}.

% There is a sequence between any two states: destroy one completely and construct the other one. (finite FS)

\begin{myth}\mlabel{update_works}
For a simple sequence of commands $U$ returned by the update detector
when comparing the non-broken $\FS^*$ to the original $\FS$,
$U\aFS = \FS^*$.
\end{myth}

\begin{proof}
First, we create a sequence of commands $S$ for which $S\aFS=\FS^*$.
First, create a topological sort of all nodes in $\setn$ that are
not empty in $\FS$ so that children come first,
that is, if $n'\descendant n$, then $n$ precedes $n'$.
For each node in the given order, add a destruction command to a sequence $S_0$
that deletes the value at $n$.
We know that the commands will never break the filesystem, as the types of nodes
with empty children can be freely changed, and that $S_0\aFS$ is empty at all nodes.
Next, order nodes in $\setn$ that re not empty in $\FS^*$ in the reverse order,
that is, if $n'\descendant n$, then $n'$ precedes $n$.
Add a construction command to a sequence $S_1$ for each node that changes an empty value
to $\FS^*(n)$.
We know that $S_1$ is defined on an empty filesystem for the same reason,
and clearly $(S_0\cc S_1)\aFS=\FS^*$.
By \cref{can_simplify} there exists a simple $S$ for which $S\eqnrw (S_0\cc S_1)$.

Therefore we know $S\aFS=\FS^*$, and that
$\FS^*$ and $\FS$ differ at exactly those nodes that $S$ has commands on.
$U$ must also contain commands on the same set of nodes,
and it must contain the same commands,
as for the command $\cxynv$, $[x]$ must be $[\FS(n)]$, and $y$ must be $\FS^*(n)$.
From this we know that $U$ is a permutation of $S$ and so $\orderset{U}=\orderset{S}$.

As the update detector uses the $\orderrel$ relation to order the commands in $U$, trivially $U\in\orderset{U}$,
which means $U\in\orderset{S}$,
and so from \cref{simple_reorder_equiv} we know that $U\equiv S$,
from which $U\aFS=\FS^*$.
\end{proof}
