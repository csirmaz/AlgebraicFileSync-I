
\section{Related Work}\mlabel{sec_relatedwork}

\subsection{Liberal and Conservative Reconciliation}

We consider the reconciliation algorithm described here an improvement over
the one derived during our previous research \cite{NREC}
as the previous algorithm not only fails to propagate all possible commands
wherever it is possible (that is, where it does not break the filesystem),
but the current algorithm is also simpler.
This is because the previous reconciliation algorithm excludes
commands from being propagated which must be preceded by a command that conflicts.

The former observation is supported by Berzan and Ramsey, who in \cite{CBNR} 
describe different general reconciliation policies.
The liberal (maximal) policy propagates all updates to all replicas where
the update does not break the filesystem, while a conservative policy
refrains from updating any node that is below a node with conflicting commands.
They show that the reconciliation algorithm in \cite{NREC} implements
an intermediate policy as one can easily construct two- or three-replica scenarios
where an update could clearly be propagated, but it is excluded.
To rephrase the example in \cite{CBNR}, consider the following two
update sequences that have been applied to replicas $\FS_A$ and $\FS_B$:
\begin{align*}
A&=\caaa{\empt}{\valdi}{\pn}\cc\caaa{\empt}{\valf}{n} \\
B&=\caaa{\empt}{\valdii}{\pn}
\end{align*}
(\cite{NREC} did not require all directories to have the same value).
Clearly $\caaa{\empt}{\valf}{n}$ could be applied to $\FS_B$, but it is not as
it must be preceded by creating the directory, which conflicts with the same command in $B$
due to the different value.
(Introducing a third replica which has not changed at all further complicates the picture.)
The current algorithm no longer needs to specify that there can be no conflicts
on preceding commands, and, to use the above terminology, implements a fully liberal policy.


\subsection{Comparing Definitions of Conflicts}

As a test of the proposed reconciliation algorithm, 
we compare our definition of conflicting
updates to how conflicts are defined by Balasubramaniam and Pierce
in their state-based approach implemented in the Unison synchronizer \cite{BP}.
As we noted in \cite{NREC}, the update detector they describe provides a safe estimate of nodes
(paths) at which updates occurred.
It marks some nodes as \emph{dirty} in a way that we know that at non-dirty nodes
the filesystems (replicas) have not changed between their common original state and current state
(see Definition 3.1.1 in \cite{BP}).
The \emph{dirty} marks are also up-closed, that is, all ancestor nodes of a dirty node
are also dirty (Fact 3.1.3 in \cite{BP}).
And finally, in our notation, 
a conflict is detected between replicas $\FS_A$ and $\FS_B$ at node $n$
if $n$ is marked as dirty in both $\FS_A$ and $\FS_B$, and
$\FS_A(n)\neq\FS_B(n)$, and $n$ does not point to a directory in both replicas.
(In other words, 
$[\FS_A(n)]\neq[\vald]$ or $[\FS_B(n)]\neq[\vald]$; see section 4.1 in \cite{BP}.
Let us note that there is an alternative approach to defining an algorithm for
the same synchronizer by Pierce and Vouillon in \cite{PV}.)

It can be easily seen that due to an edge case, 
not all conflicts detected based on the above definition
entails a conflict based on our system; that is, \cite{BP} describes a more
conservative policy.
We use the example we described in \cite{NREC},
where $\FS(\pn)$ is a directory and $\FS(n)$ is a file, and
the two replicas are derived in the following way:
\begin{gather*}
\FS_A = (\cfba{n} \cc \cdba{\pn}) \aFS \\
\FS_B = \cfba{n} \aFS.
\end{gather*}
Then, $\pn$ is dirty in both replicas:
in $\FS_A$ it was modified, and in $\FS_B$ one of its descendants was modified.
Moreover, $\FS_A(\pn)\neq\FS_B(\pn)$ and $[\FS_A(\pn)]\neq[\vald]$ as it is empty.
Therefore, a conflict is detected at $\pn$.
(This behavior is preserved in the more recent Harmony synchronizer.
See ``delete/delete conflicts'' in \cite{PSG,FGKPS}.)
Our reconciliation algorithm detects no conflicts;
instead, it propagates $\cdba{\pn}$ to $\FS_B$, which we think is as expected
and desired.

At the same time, it can be shown that if our command-based reconciler
detects a conflict, it entails a conflict in the state-based reconciler.
We note here that Balasubramaniam and Pierce also suppose that all directories are equal,
therefore, as elsewhere, we are safe to continue to assume that $|\setd|=1$.

\begin{proof}
Let $A$ and $B$ be two simple sequences returned by the update detector
for the two replicas $\FS_A$ and $\FS_B$.
A conflict between the commands $\cxynv\in A$ and $\czwmv\in B$
means that even though $\cxynv\in A\setminus B$
and $\czwmv\in B\setminus A$, they cannot be included in
$\reca$ and $\recb$, respectively, because
$\cxynv\nindep\czwmv$ (see \cref{def:reconciliation}).

From \cref{incomparable_is_independent} we therefore know that $n\nunrel m$.
Without loss of generality we can assume that $n\descendantEq m$.
From this we see that $n$ is \emph{dirty} in both $\FS_A$ and $\FS_B$,
as the filesystem changes at $n$ in $\FS_A$ and at $n$ or its descendant in $\FS_B$
(and none of the commands are assertion commands).

Now we only need to show that $\FS_A(n)\neq\FS_B(n)$, because from this 
and $|\setd|=1$ we will also
know that at least one of these values is not a directory,
and therefore there is a conflict at $n$ in the state-based reconciler.

If $n=m$, this follows from the fact that $\cxynv\neq\czwmv$ 
(because they are not in $A\cap B$),
as from \cref{worksinputmatch} we have $[x]=[z]$,
and therefore necessarily $y\neq w$,
that is, $\FS_A(n)=y\neq w=\FS_B(n)$.

If $n\descendant m$ and there is no command on $n$ in $B$, then 
because $\cxynv$ is not an assertion command, we again have
$\FS_A(n)\neq\FS(n)=\FS_B(n)$, where $\FS$ is the common ancestor of the replicas.
Finally, if there is a command on $n$ in $B$, then 
from $\cxynv\in A\setminus B$ we know
it must be different from $\cxynv$, and similarly to the first case,
we have $\FS_A(n)\neq\FS_B(n)$.
\end{proof}

