
\subsection{Ordering commands}\label{ordering}

We often encounter the case where we only have a set of commands without a specified order.
As we have seen above, this can occur after the first stage of update detection,
but, more importantly,
it is actually the task of the reconciler to determine whether there is an order,
and if yes, what order,
in which updates from different replicas can be applied to a filesystem.

In this section we describe an algorithm with which all possible orderings of a
simple set of commands can be generated, but before doing so, let us consider the following lemma.

\begin{mylem}\label{connected_changes}
Given a set of commands that
does not contain assertion commands,
and which can be applied to a filesystem in some order without breaking it,
and which contains commands on $n$ and $\nn$ where $\nn\descendant n$,
then the set must also contain a command
on each node between $n$ and $\nn$.
\end{mylem}
\begin{proof}
Without loss of generality, we can assume that $\nn\neq\parent(n)$.
We prove that under the given conditions, the set must contain a command on $\parent(n)$.
Then, by reapplying this result, we know that the set must contain commands on every
ancestor of $n$ up to $\nn$.

Furthermore,
we prove this proposition for sequences, not sets, as if all sequences must contain a command on $\parent(n)$,
then so must all sets because otherwise there would be no order in which the commands they contain could be
applied to any filesystem.

Let $A$ be a sequence that satisfies the conditions;
we therefore know that it contains a command $\cxynv$ on $n$
and another command $\czwnnv$ on $\nn$.
We use a proof by contradiction and assume that there are no commands on $\parent(n)$ in $A$.
Next, we create a new sequence $A'\equiv A$ in which $\cxynv$ and $\czwnnv$ are next to each other.
If they are already next to each other in $A$, there is nothing to do.
Otherwise, consider the command next to $\cxynv$ in the direction where $\czwnnv$ is.
Let this command be $\cqrmv$.
If $\czwnnv$ is to the right, then $A$ looks like the following:
\[ A = \cdots\cc\cxynv\cc\cqrmv\cc\cdots\cc\czwnnv\cc\cdots \]
If $n\unrel m$, then swap $\cxynv$ and $\cqrmv$. Based on \cref{ax_separate_commute} we know that the new
sequence is equivalent to $A$.
Otherwise, we know $m\neq\parent(n)$ as there are no commands on $\parent(n)$, and so
from \cref{ax_distantrel_breaks} we get $A\equiv\cbrk$ which contradicts our assumptions.
(Note that $A$ cannot contains assertion commands.)
By repeating this step we can therefore convert $A$ into $A'$ where $\cxynv$ and $\czwnnv$ are neighboring commands.
However, then \cref{ax_distantrel_breaks} applies to the sequence and therefore $A\equiv A'\equiv\cbrk$ which
is again a contradiction.
\end{proof}

\bigskip

\noindent
We can now define our algorithm to generate orders of sets of commands.

\begin{mydef}[$\ordersetsign$]
We use $\orderset{U}$ to denote the set of sequences 
generated by the algorithm / (all possible orderings)
from the commands in the simple-? set or sequence $U$.
\end{mydef}

Let $U$ be a simple set of commands
that we know can be applied to at least one filesystem in some order without breaking it.
Then we can split $U$ into a maximum number of disjoint subsets
(components) so that the nodes of two commands from two different subsets are incomparable.
These will simply be the components of the subset of the node forest that contains 
the nodes affected by $U$.

From \cref{ax_separate_commute} we know that commands from different 
components can be applied in any order.
Therefore once we have ordered the commands inside the components, we generate
all permutations of the components themselves, including the ones
in which commands from separate components mix and overlap
(but where the possible orders determined for commands inside the components are respected).

Next, consider the ordering inside a component.

This is not a problem in the case of components containing a single command only.
(Note that a file-to-file command ($\cff$) can only appear in a component of one.)

In each component that is larger than one, 
we know that all nodes are related, and from \cref{connected_changes}
we also know that it must contain commands on nodes in between the related nodes.
Therefore the nodes
form a rooted ordered tree,
and the pairs of commands connected by the edges in the tree 
must be construction or destruction pairs.
(Otherwise, there would be no ordering of the component that is not equivalent to $\cbrk$
and via \cref{ax_separate_commute} there would be no ordering of the whole set, either.)

To order the commands in the component,
select the command with the topmost node, so that there is no command
on its parent node (if there is one). 
If the command removes content at this node (it is of the form $\caa{X}{\ccharb}$), 
then it must follow all other commands in the component, and
all parent--child pairs in the whole component must be destruction pairs.
Otherwise, the command with the topmost node needs to precede the other commands,
and the component must be made of construction pairs.
These follow from the patterns of construction and destruction pairs, which
at no point can mix without causing any ordering to break all filesystems.

Then move the command with the topmost node to the beginning or the end of
the sequence we are creating out of the component,
and recursively start the whole algorithm again on the commands in the component that remain,
potentially splitting them into sub-components.
The result of this process is a set of possible orders for each component,
which are then combined into a set of possible orders for the whole set of commands.

\bigskip

\noindent As a specific case, we can regard a simple sequence $S$
as a set, and use the above algorithm to generate all possible orderings
of the commands it contains.
\Cref{simple_reorder_equiv} is
an important property of simple sequences: namely, that
all of their valid reorderings are equivalent,
and all equivalent sequences are mere reorderings.
To prove this, first we prove the following:

\begin{mylem}\label{equiv_simple_same_commands}
If two simple sequences $A$ and $B$ are equivalent
and do not break all filesystems ($A\equiv B\nequiv\cbrk$),
then they must contain the same commands.
\end{mylem}
\begin{proof}
We use a proof by contradiction. Let $A$ and $B$ be simple sequences
such that $A\equiv B\nequiv\cbrk$,
and $\FS$ be a filesystem that they do not break.

We also assume that they do not contain the same commands.
Without loss of generality, we can assume
that $A$
contains $\cxynv$, and $B$ either contains a different command
$\czwnv$ on $n$, or no command on $n$ at all.
As $A$ is simple, we know that $\cxynv$ is not an assertion command,
and therefore $X\neq Y$ and $\FS(n)\neq\valvy$ as it has type $X$.
If so, then if $B$ has no command on $n$, then $B\aFS(n)=\FS(n)$ and
$A$ and $B$ cannot be equivalent.
Otherwise we know that $Z=X$ as $B$ does not break $\FS$ either,
and that $\valvw=\valvy$ and therefore $W=Y$ as $B\aFS(n)$ must
be $\valvy$ and $B$ only has one command on $n$.
However, then $\czwnv=\cxynv$, which is a contradiction.
\end{proof}



\begin{mylem}\label{simple_reorder_equiv}
For a simple sequence $S\nequiv\cbrk$,
\[ \orderset{S} = [S]_\equiv, \]
where $[S]_\equiv$ denotes the set of simple sequences equivalent to $S$.
\end{mylem}
\begin{proof}
First, as $S$ only has at most one command on each node,
any sequence in $\orderset{S}$ either breaks a filesystem,
or it leaves it in the same state as $S$.

Second, based on the algorithm we also know that $\orderset{S}$ contains
all possible orderings of $S$ as the only ordering between commands
it prescribes are when the commands in the reverse order would break every filesystem.

If, for example, in a component or sub-component
made up of construction pairs,
the command on the topmost node, $\cxynv$, was not before all other commands
in the component as prescribed by the algorithm,
then any resulting sequence would necessarily contain
a $\czwmv$ where $n\descendant m$, potentially followed by commands
on incomparable nodes, and then followed by $\cxynv$.
From \cref{ax_separate_commute} we know we can convert this sequence
to an equivalent sequence by swapping the commands on incomparable nodes with
$\czwmv$, which sequence would therefore have the commands on
$m$ and $n$ next to each other. Then from \cref{ax_distantrel_breaks,ax_directparent_breaks}
we know the sequence would break all filesystems
as a reversed construction pair is never a valid destruction pair.

We can prove that an order not generated by the algorithm
in the case of a component made of destruction pairs breaks all filesystems
in a similar way.

Also, from \cref{ax_separate_commute} we know
that the order between any other command pairs is not fixed in the sense
that if one order works, so will the other.

From these we know that all sequences in $\orderset{S}$
are equivalent, and that any sequence of the same commands outside it
necessarily breaks all filesystems.
We also know $S\in\orderset{S}$ as $S\nequiv\cbrk$
and therefore $T\in\orderset{S} \Rightarrow T\equiv S$.

From \cref{equiv_simple_same_commands} we know that 
if $T\equiv S$, then $T$ contains the same commands as $S$,
from which we see
that $T\equiv S\Rightarrow T\in\orderset{S}$.
\end{proof}






%% TODO https://en.wikipedia.org/wiki/Equivalence_class


It follows from the algorithm that:
\begin{mycor}\label{lemma:neighbor}
If $U$ contains commands on both $n$ and $\parent(n)$, then
there is a simple $U'\equiv U$ in which they are next to each other.
\end{mycor}



It follows from this that given a simple set,
we can simply assume that it has an order.
Accordingly, we will at times treat not only sequences as sets,
but simple sets as sequences.

