
\subsection{Commands on Filesystems}

Next, we define commands on filesystems.
As described above, we aim to select a set of commands
that captures as much information
about the operations as possible, and is also symmetric.
We start by summarizing our reasons for doing so.

% What is encoded in a command?

Let us consider what kind of information is usually encoded in filesystem operations.
A minimal set of commands, based on the most frequent tools implemented by filesystems,
may be the following, where 
$n\in\setn$ and $\valv\in\setv$ (but $\valv\neq\empt$):
\begin{itemize}
\item \textit{create}$(n,\valv$), which creates a file or directory ($\valv$) at $n$
where the file system contains no file or directory ($\empt$);
\item \textit{edit}$(n,\valv)$, which replaces the earlier file or directory at $n$ with $\valv$;
\item \textit{remove}$(n)$, which removes the file or directory at $n$, and replaces it with $\empt$.
\end{itemize}
Regarding their output, that is, the state of the filesystem at $n$
after applying the command,
we know that after \textit{create} or \textit{edit,} $\FS(n)\neq\empt$, whereas after \textit{remove,}
$\FS(n)$ will be $\empt$. 
However, from \cite{NREC} and \cite{CBNR} we know that a useful set of axioms
will in some cases need to distinguish between, for example,
\textit{edit}s that result in directories (\textit{edit}$(n,\vald)$) and
ones that result in files (\textit{edit}$(n,\valf)$), and treat them as separate commands,
as their behaviors are quite different when combined with other commands.
Indeed, Bill Zissimopoulos' work
demonstrated \cite{BZ}
that extending this distinction to more commands ultimately simplifies
the definition of conflicting commands, as our model will then able to predict the behavior of commands
more precisely.

Notice, however, that the commands listed above also encode some information about 
their input, the state of the filesystem
before the command is applied. In particular, \textit{create}$(n,\valv)$ requires that there are no files
or directories at $n$, while \textit{edit}$(n,\valv)$ and \textit{remove}$(n)$ require the opposite.
This creates an arbitrary asymmetry where
there is now more information available about their output than about their input.
As, based on the above, we expect that encoding more information in the commands
results in a model with greater predictive powers,
and in order to resolve this asymmetry, 
we propose a set of commands that encode
the type of the input value $\FS(n)$ as well.
(Some real-life filesystem commands like \textit{rmdir} do this already.)
Because the success of failure of commands depends on types of values,
it is not necessary to encode the actual input value.

% Definition
% Break / empty command

We model filesystem commands as partial functions 
that map filesystems to filesystems, but which
are only defined where the commands succeed.
As usual, if the function is not defined, we say that that it returns the bottom element, $\fsbroken$,
or that it \emph{breaks} the filesystem.
We describe a filesystem commands using its input, output, and
the node it is applied to.

\begin{mydef}[Filesystem commands]
Filesystem commands are partial endofunctions on filesystems:
\[ \alpha: \setfs \nrightarrow \setfs \]
A command $\alpha$ is not defined on filesystems where it returns an error.
The commands are represented using triplets of the form
\[ \alpha = \caaaa{x}{y}{n}, \]
where $x,y\in\setv$ and $n\in\setn$.
The equivalence class $[x]$ is the \emph{input type} of the function;
$y$ is its \emph{output value} implicitly specifying the \emph{output type} $[y]$,
and $n$ is the node the command is applied to.
\end{mydef}
For example, $\cbfaa{n}$ represents \texit{create}$(n,\valf)$,
and $\cdba{n}$ represents \textit{rmdir}$(n)$.

We write $\cbrk$ for the empty partial function which is not defined anywhere.
This does not naturally occur in sequences of commands we will investigate,
but is useful when reasoning about the combined effects of commands.


TODO DELETE:
As suggested above, we will frequently only investigate
the types of the input and output values of a command
(its input and output types), and write
$\cxy=\czw$, by which we mean $[x]=[z]$ and $[y]=[w]$.

We note that a command or a sequence of commands is applied to a filesystem
by prefixing the command or sequence to it, for example: $\cbrk\aFS$, $\cbdaa{n}\aFS$, 
or $S\aFS$ if $S$ is a sequence of commands.

To define the effect of a command on a filesystem, we use
the replacement operator of the form
$\fsreplacement{\FS}{n}{\valv}$ to denote a filesystem derived from $\FS$ 
by replacing its value at $n$ with $\valv$:
\[ \fsreplacement{\FS}{n}{\valv}(m) =
   \begin{cases}
   \valv &\textrm{if~} m=n\\
   \FS(m) &\textrm{otherwise.}
   \end{cases}
\]

\begin{mydef}[Effect of commands]
The effect of the command $\cxynv$ is defined as follows:
\cxynv\aFS = 
   \begin{cases}
   \fsbroken &\textrm{if $\FS(n)]\neq[x]$,}\\
   \fsbroken &\textrm{if $\fsreplacement{\FS}{n}{y}$ violates the tree property,}\\
   \fsreplacement{\FS}{n}{y} &\textrm{otherwise.}
   \end{cases}
\end{mydef}
In other words, the command $\cxynv$ breaks a filesystem if the input type $[x]$
does not match the type of the value in the filesystem at $n$, or if after replacing
the value at $n$ with $y$, the resulting filesystem ceases to satisfy the tree property.

% Simplification
% --------------

\bigskip

\noindent
For reasons also listed in \cite{NREC}, in this paper we will not consider
a \textit{move} or \textit{rename} command.
This turns out to be useful because this would be the only command that affects
filesystems at two nodes at once, therefore describing 
the dependencies for \textit{move} would call for a more complicated model.
From a pragmatic perspective, this restriction does not mean that in an application
implementing conflict resolution using the algorithm described here would not be
able to handle renames by pre- and post-processing changes in the filesystem to
discover them, which can enhance usability, but which
(especially when a rename is combined with changes to the content)
is a non-trivial problem in itself.
