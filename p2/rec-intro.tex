
If we start with two copies of the filesystem $\FS$,
and two different sequences are applied to the copies to yield $\FS_A=A\aFS$
and $\FS_B=B\aFS$, then our aim is to define sequences of commands $\reca$ and $\recb$
so that $\recb\aFS_A$ and $\reca\aFS_B$ would be as close to each other as possible.

We work based on the assumption that to achieve this, we need
to apply to $\FS_B$ the commands that have been applied to $\FS_A$, and \emph{vice versa}.
As some commands may have been applied to both filesystems, our first approximation
is $\reca = \amb$ and $\recb = \bma$.
This, however, will break both filesystems if there have been incompatible updates
in $A$ and $B$. 
Our aim is therefore to provide an algorithm that selects the commands 
$\reca \subset A\setminus B$
and $\recb \subset B\setminus A$ 
so that $\reca\aFS_B\neq\fsbroken$ and $\recb\aFS_A\neq\fsbroken$,
and show that these are the longest sequences with this property, that is,
adding any further command to $\reca$ or $\recb$ will break the filesystems.

We assume that $A$ and $B$ are returned by an update detector,
and so they are simple sequences.
(Otherwise, they can be simplified to become simple sequences; see \cref{can_simplify}.)
We also assume that $\FS$, $\FS_A$ and $\FS_B$ are not broken.
With these assumptions,
we can now define our reconciliation algorithm.

\begin{mydef}[Reconciliation]\mlabel{def:reconciliation}
We determine $\reca$---which we apply to $\FS_B$ 
to reconcile it with $\FS_A$---to be
the largest subset of $A\setminus B$
that is independent of $B\setminus A$,
where $A$ and $B$ are simple sequences representing
the updates applied to $\FS$ to arrive at replicas $\FS_A$ and $\FS_B$:
\[ \reca = \ords{\alpha \whr \alpha\in \ambnp  \mbox{~and~}  \alpha\indep \bmanp }. \]
$\recb$ can be obtained in a similar way by reversing $A$ and $B$
in the definition.
\end{mydef}

Note that $\alpha$ is not an assertion command, and $\bmanp$ contains no assertion commands;
also, $\alpha\not\in\bmanp$, so independence in this case is equivalent to
the node $\alpha$ affects being incomparable to the nodes commands in $\bmanp$ affect.
Therefore whether two commands are independent is easy to determine programmatically
in at most $O(x^2)$ time, where $x$ is the number of commands. % TODO
Then, to generate the sequence $\reca$ from the resulting set, 
we can order the commands according to $\orderrel$ as defined in \cref{ordering}.


\myskip
Next, we would now like to prove that $\reca$ and $\recb$ can be applied to the replicas,
that is, without loss of generality,
\[ \reca\aFS_B\neq\fsbroken. \]
So that we can formalize statements needed to prove this,
let us introduce partial orders that describe under what conditions
sequences of commands work, that is, do not break a filesystem.
