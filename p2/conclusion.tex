
\section{Conclusions and Further Research}

% TODO
... ... ...

(Same approach to other systems)
\medskip

% Further research
% TODO Merge with / incorporate other ideas?

Apart from constructing an algebra,
there are many other ways in which further work can extend the current results.
We think one of the most interesting extensions would be to
consider reconciling not only two, but more replicas in a single step and
prove the correctness and maximality of the algorithm proposed.
A related problem is to model cases where reconciliation cannot
complete fully
or if only a subset of the replicas are reconciled 
(e.g. due to network partitioning),
both of which would result in a state where different replicas
have different common ancestors, that is,
the updates specific to the replicas start from different points
in the update history of the filesystem.
Parker et al. (\cite{PPRS}) and Cox and Josephson (\cite{CJ})
describe version vectors (update histories) kept as metadata that could solve this problem,
while Chong and Hamadi present distributed algorithms that allow incremental synchronization (\cite{CH}).
Extending the algebraic approach to represent individual updates to files
in their modification histories (as described in \cite{CJ})
as separate commands could also enable it to reconcile otherwise
conflicting updates and resolve partial reconciliations.

An extension of these results could also entail relaxing
the simplifying assumptions we used.
Identifying which results depend on the assumption that $|\setd|=1$
and devising updated proofs or algorithms,
or a pre-processing step that reconciles meta-information in directories
can enable the algebraic system or an implementation
to identify and resolve a wider range of conflicts.
Similarly, designing pre- and post-processors that
present certain changes not as removal and creation of a node,
but as a rename,
could contribute to the user-friendliness of the reconciler
and help avoid human error.
