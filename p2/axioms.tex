

\section{Command Pairs and Sequences}\mlabel{section_axioms}

% Sequences
% ---------

So that we can describe the effects of commands independently of filesystems,
let us introduce some notation
and note some observations.
We already know that
commands usually do not occur in isolation,
and are applied to filesystems in time.
Therefore we investigate sequences of commands with a well-defined order.
\begin{mydef}[Sequences of commands and $\emptyseq$]
We concatenate commands in writing to note that they form a sequence,
and concatenate sequences of commands to form a longer sequence,
with the meaning that the resulting sequence is executed from left to right:
\[ (\alpha\cc\beta)\aFS = \beta(\alpha\aFS). \]
%% monoid, but not free monoid
%% monoid: semigroup with identity element
%% Sequences of commands form a free semigroup
Sequences are also partial endofunctions on filesystems,
defined only if all commands they contain succeed in the given order.
Sequences form a monoid, and, as usual,
we write $\emptyseq$ to denote the unit element, the empty sequence,
which is defined on all filesystems and, by definition, leaves all filesystems unchanged.
\end{mydef}

\begin{mydef}[$\Dom{S}$]
For a sequence of commands $S$, $\Dom{S}$ is the domain of $S$, that is,
the set of filesystems $S$ does not break.
\end{mydef}


The following two relations 
echo the ones defined in \cite{NREC}.
In the definitions, $A,B,S$ and $T$ stand for arbitrary sequences.

\begin{mydef}[$\eqext$, $\strext$ and $\equiv$]
% Relation in the algebra
We write $A\eqext B$, or say that $B$ \emph{extends} $A$,
% intended interpretation:
to mean that they behave in the same way
on any filesystem $A$ does not break:
$\forall \FS\in\Dom{A}:\,A\aFS=B\aFS$.
% old inference rule: $ A\eqext B \Rightarrow S\cc A\cc T\eqext S\cc B\cc T$,
We can also see that $A\eqext B$ and $S\eqext T$ implies $A\cc S\eqext B\cc T$.
% symmetric:     a R b <=> b R a
% antisymmetric: a R b & b R a => a = b
% reflexive:     a R a
% transitive:    a R b & b R C => a R c
% preorder:      reflexive, transitive
% partial order: reflexive, transitive, antisymmetric
% equivalence:   reflexive, transitive, symmetric
% congruence:    https://en.wikipedia.org/wiki/Congruence_relation
% precongruence: congruence of a preorder
In other words, $\eqext$ is a preorder, and also a precongruence.

% Relation in the algebra
We write $A\equiv B$,
or say that $A$ and $B$ are \emph{equivalent,}
iff $A\eqext B$ and $B\eqext A$;
that is, $\equiv$ is the intersection of the preorder $\eqext$ with its inverse,
and so it is a congruence.
% old inference rule: $ A\equiv B \Rightarrow S\cc A\cc T\equiv S\cc B\cc T $.

Finally, we write $A\strext B$ to mean $A\eqext B$ and $A\nequiv B$.
In particular, we write $\wrks{A}$
to mean that $\Dom{A}$ is not empty, that is, if $A$ is defined on some filesystems.
\end{mydef}

It is easy to see that the equivalence $\equiv$ holds on the level of filesystems:
\begin{mylem}\mlabel{equiv_on_fs}
$A\equiv B$
iff $A$ and $B$ behave in the same way on
all filesystems, that is, $\forall \FS: A\aFS=B\aFS$.
\end{mylem}
\begin{proof}
As $A\equiv B$ means that both $A$ extends $B$ and $B$ extends $A$, it necessarily
follows that $A$ and $B$, as partial functions, are identical.
\end{proof}

% Rules
% -----

\myskip
One aim of our algebraic model is to
derive as much information about the effects of sequences
of commands independently of the actual filesystems as possible.
In order to make this possible, we investigate the smallest building
blocks of sequences: pairs of commands that act on a filesystem directly one after the other.
This approach is useful as there are a limited number of command pairs,
because, as we argued above, we can disregard the exact output values of commands apart from their type,
and we can also abstract the relationship between the nodes in the two commands
to a finite number of cases.
These properties of command pairs are crucial as they determine
how a set of commands can be re-ordered to be applied to a filesystem
during synchronization, and what command pairs will never be compatible.

Command pairs in general have the form
\[ \cxynv\cc  \czwmv \]
where $x,y,z,w\in\setv$ and $n,m\in\setn$. 
Extending \cref{def:command_categories}, we call certain pairs
\emph{construction} or \emph{destruction pairs}.

\begin{mydef}[Construction and destruction pairs]
A pair of commands on nodes $\pn$ and $n$
is a \emph{construction pair} if the input and output types match
one of the following patterns:
   \begin{align*}
            \cbda{\pn}&\cc  \cbfa{n} \\
            \cbda{\pn}&\cc  \cbda{n} \\
            \cfda{\pn}&\cc  \caaa{\empt}{\valff}{n} \\
            \cfda{\pn}&\cc  \cbda{n}
   \end{align*}
The pair is a \emph{destruction pair} if the types match one of the following:
   \begin{align*}
            \cfba{n}&\cc  \cdba{\pn} \\
            \cfba{n}&\cc  \caaa{\vald}{\valff}{\pn} \\
            \cdba{n}&\cc  \cdba{\pn} \\
            \cdba{n}&\cc  \cdfa{\pn}
   \end{align*}
\end{mydef}

We can see that construction pairs consist of construction commands,
while destruction pairs consist of destruction commands.




\Cref{rules_lemma} summarizes the basic properties of command pairs.
The claims listed are named \emph{\namecrefs{ax_separate_commute}}
as they can also be interpreted as inference rules in a pure
algebraic treatment of filesystem synchronization.
\Cref{sec_algebra} elaborates this approach.
In \cref{axiom_proof} we offer proofs of each of the
\emph{\namecrefs{ax_separate_commute}} based on our model of filesystems.

\begin{mylem}\mlabel{rules_lemma}
\newcounter{rulecounter}
\begin{list}{\bf Rule~\arabic{rulecounter}.}{\usecounter{rulecounter}}

\item[] % line break after the lemma label

\item\mlabel{ax_separate_commute}
\axaxseparatecommute

\item\mlabel{ax_separate_nobreaks}
\axaxseparatenobreaks

\item\mlabel{ax_same_breaks}
\axaxsamebreaks

\item\mlabel{ax_same_emptyseq}
\axaxsameemptyseq

\item\mlabel{ax_same_singlec}
\axaxsamesinglec

\item\mlabel{ax_directchild_breaks}
\axaxdirectchildbreaks

\item\mlabel{ax_directparent_breaks}
\axaxdirectparentbreaks

\item\mlabel{ax_distantrel_breaks}
\axaxdistantrelbreaks

\item\mlabel{ax_child_assert}
\axaxchildassert

\item\mlabel{ax_parent_assert}
\axaxparentassert

\item\mlabel{ax_assert}
\axaxassert

\end{list}
\end{mylem}



We have included \cref{ax_directchild_breaks,ax_directparent_breaks}
for completeness, but we will use the more generic \cref{simple_distant_pairs},
which extends these rules to non-adjacent commands, although only in the case
of command sequences without any assertion commands.




% Independent commands
% --------------------

\myskip
We also define the concept of two commands being independent.

\begin{mydef}[$A\indep B$: Independent commands, sequences and sets of commands]\mlabel{def_indep}
Two commands $\alpha$ and $\beta$ 
are independent, and we write $\alpha\indep\beta$ if 
they commute and do not break all filesystems:
\[ \alpha\cc\beta \equiv \wrksx{\beta\cc\alpha}. \]
For two sequences or unordered sets of commands $A$ and $B$ we write $A\indep B$ if
for all $\alpha$ in $A$ and all $\beta$ in $B$, $\alpha\indep\beta$.
We also write $\alpha\indep B$ for $\{\alpha\}\indep B$.
\end{mydef}

It is intentional that we use the same symbol for independent commands
and sequences as for incomparable nodes. As
\namecref{incomparable_is_independent},
shows, these concepts are closely related.

\begin{mycor}\mlabel{incomparable_is_independent}
If $\cxynv$ and $\czwmv$ are different commands,
and none of them is an assertion command, then
$\cxynv\indep\czwmv$ if and only if $n\unrel m$.
\end{mycor}

To prove this, we are going to prove the following lemma,
which characterizes independent commands in a more detailed way.

\begin{mylem}\mlabel{independent_details}
For two commands $\cxynv\indep\czwmv$ if and only if
one of the following holds:
\begin{itemize}
\item $n\unrel m$;
\item or $n=m$ and the commands are equivalent assertion or replacement commands;
\item or $n\descendant m$ or $m\descendant n$,
and either the command on the ancestor asserts a directory,
or the command on the descendant asserts an empty node (or both).
\end{itemize}
\end{mylem}

\begin{proof}
By definition we know that $\cxynv\cc\czwmv\equiv\wrksx{\czwmv\cc\cxynv}$.
The first case is a restatement of
\cref{ax_separate_commute,ax_separate_nobreaks}.
Otherwise $n\nunrel m$, in which case either $n=m$, or, without loss of generality, $n\descendant m$.

If $n=m$, then from \cref{ax_same_breaks} we must have $[y]=[z]$ and $[w]=[x]$
as otherwise the commands, in one order or the other, would break all filesystems.
The equivalence also implies $[x]=[z]$ as otherwise the two sides would not be defined on the same filesystem,
as well as $y=w$ as their effects is the same.
The two commands are therefore the same, and they are either assertion or replacement commands.

The remaining case is that $n\descendant m$.
If the nodes are not directly related, then \cref{ax_distantrel_breaks} gives
the conditions in the third case in the lemma.
If they are directly related, 
\cref{ax_directchild_breaks,ax_directparent_breaks}
gives the same conditions, while they also allow the commands being
a construction or destruction pair.
However, the reversals of construction and destruction pairs break all filesystems.
\end{proof}

\Cref{incomparable_is_independent} follows directly from \cref{independent_details}.
While equivalent commands can hardly be thought of as independent,
we retain this name for this relation as in most
cases we use it on pairs of different commands that are also not assertion commands.
