
In this section we offer proofs for the \namecrefs{ax_separate_commute} listed in
\cref{rules_lemma}.
For the purposes of the proofs we extend the notion of filesystems 
to the full set of functions mapping $\setn$ to $\setv$, even if they
do not have the tree property.
In this case we continue to say that the filsystem is broken,
but we also describe the filesystem values at certain nodes.
Also, in order to pinpoint the region that violates the tree property,
we say that the filesystem is \emph{broken at $n$} if 
$n$ is not a dictionary, but has at least one non-empty child.

Accordingly, we also extend filsystem commands from partial to full functions,
and assume that they return a filesystem we can describe in all cases.
We note that once a filesystem is broken, it stays broken, even if
after a subsequent command it no longer violates the tree property.

It follows from the definition of the tree property that
whether a filesystem is \emph{broken at a node $n$} is determined only
by the values at $n$, $\parentf{n}$, and the children of $n$.
In other words,

\begin{myclm}[Locality of tree property violations]\mlabel{broken_local}
If two filesystems over $\setn$ have the same values at a node $n$,
$\parentf{n}$ (if exists), and at all children of $n$ (if they exist),
then either both filesystems are broken at $n$, or none of them are.
\end{myclm}

\cref{ax_separate_commute}

We use an inverse proof and assume $\cxynv\cc\czwmv \nequiv \czwmv\cc\cxynv$.
Since the values in the resulting filesystems are the same,
this is only possible if, for an initial non-broken filesystem $\FS$,
one side results in a broken filesystem, and the other side does not.
Without loss of generality, we can assume
$(\cxynv\cc\czwmv)\aFS$ is not broken, but $(\czwmv\cc\cxynv)\aFS$ is.
This means that either $\czwmv\aFS$ is already broken, or it is not, and applying $\cxynv$
breaks the filesystem.

In the former case, the filesystem must be broken at $m$ or $\parentf{m}$.
Since the set of $n$ with its granparent, parent and children; and $m$ with its grandparent, parent and children
are disjunct, we know $\cxynv\aFS$ contains the same values around $m$ as $\aFS$.
This leads to contradiction as from \cref{broken_local} we know
that being broken at $m$ or $\parentf{m}$
... WHAT ABOUT SIBLINGS? ...
while applying $\czwmv$ to $\cxynv\aFS$ on the left-hand side resulted in a working filesystem,
it resulted in a broken filesystem when applied to $\FS$ on the right-hand side.
The proof is similar in the case when applying $\cxynv$ breaks the filesystem on the right-hand side.
