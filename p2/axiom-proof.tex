TODO From filesystem definition


In this section we offer proofs for the \namecrefs{ax_separate_commute} listed in
\cref{rules_lemma}.
For the purposes of the proofs we extend the notion of filesystems 
to the full set of functions mapping $\setn$ to $\setv$, even if they
do not have the tree property.
In such cases we continue to say that the filsystem is broken,
but we also continue to describe the filesystem values at nodes in $\setn$.
Also, in order to pinpoint the region that violates the tree property,
we say that the filesystem is \emph{broken at $n$} if 
$n$ is not a dictionary, but has at least one non-empty child.

Accordingly, where needed, we also implicitly extend filesystem commands
so that they would return a filesystem even if it violates the tree property.
They continue to be undefined in cases where their input types are incompatible
with the input filesystem.
We note that once a filesystem is broken, it stays broken, even if
after a subsequent command it would no longer violate the tree property.

It follows from the definition of the tree property that
whether a filesystem is \emph{broken at a node $n$} is determined only
by the values at $n$ and the children of $n$.
In other words,

\begin{myclm}[Locality of tree property violations]\mlabel{broken_local}
If two filesystems over $\setn$ have the same values at a node $n$,
and at all children of $n$ (if they exist),
then either both filesystems are broken at $n$, or none of them are.
\end{myclm}

\cref{ax_separate_commute}

\begin{proof}
We use an inverse proof and assume $\cxynv\cc\czwmv \nequiv \czwmv\cc\cxynv$.
Since the values in the resulting filesystems are the same,
this is only possible if, for an initial non-broken filesystem $\FS$,
one side results in a broken filesystem, and the other side does not.
Without loss of generality, we can assume
$(\cxynv\cc\czwmv)\aFS$ is not broken, but $(\czwmv\cc\cxynv)\aFS$ is.
This means that either $\czwmv\aFS$ is already broken, or it is not, and applying $\cxynv$
breaks the filesystem.

In the former case, based on \cref{broken_local}, 
$\czwmv\aFS$ must be broken at $m$ or $\parentf{m}$, since it only changed at $m$.
Let us look at the first case.
Since $n\indep m$, we know that the parent and children of $m$ have the same value
in $\FS$ as they do in $\cxynv\aFS$. However, this is a contradiction,
as applying $\czwmv$ to $\FS$ leads to a broken filesystem,
but applying it to $\cxynv\aFS$ (from the left side) does not.

Therefore $\czwmv\aFS$ must be broken at $\parentf{m}$.
By definition this means that its value at $\parentf{m}$ cannot be a directory.
Also, reasoning similar to the above shows that this is only possible if $\cxynv$ changes
the environment of $\parentf{m}$, which, since $n\indep m$, is only possible
if $n$ and $m$ are siblings.

Since then the value at $\parentf{n}=\parentf{m}$ is not changed by either command,
we know it cannot be a directory in $\FS$, either, and that therefore
$\FS$ is empty at all children of the parent node.
Since by assumption $\czwmv\aFS$ is broken, $[w]\neq[\empt]$ must hold.
However, this is a contradiction, as this would necessarily mean that
the left side, $\czwmv(\cxynv\aFS)$ must also be broken.

We can proceed in the same fashion if $\czwmv\aFS$ is not broken, but
$(\czwmv\cc\cxynv)\aFS$ is.
\end{proof}

\cref{ax_separate_nobreaks}

\begin{proof}
It is easy to see that over any $\setn$ one can construct a filesystem that neither command breaks.
Since $n\indep m$, it is always possible to set all descendants of both $n$ and $m$ empty,
and have a directory at all ancestors of both $n$ and $m$. In such positions any values are permissible,
so neither command will break the filesystem.
\end{proof}

\cref{ax_same_breaks}

\begin{proof}
This is trivial, as either $\cxynv$ breaks the filesystem, or, if it does not, then
we know $[\cxynv\aFS(n)]=[y]\neq [z]$, and therefore $\czwnv$ will break the filesystem.
\end{proof}

\cref{ax_same_emptyseq}

\begin{proof}
From the conditions we see that for every $\FS$,
$\FS$ and $(\cxynv\cc\czwnv)\aFS$ have the same values at every node,
while the latter can still be broken if $\cxynv\aFS$ is broken.
This is equivalent to saying that where the $\cxynv\cc\czwnv$ function is defined,
it is the identity function, that is, it is extended by $\emptyseq$.
\end{proof}

\cref{ax_same_singlec}

\begin{proof}
It is easy to see that where both $\cxynv\cc \czwnv$ and $\cxwnv$ are defined
(they do not break the filesystem), they are equivalent.
What remains to show is that they break the same set of filesystems.
If $(\cxynv\cc \czwnv)\aFS$ is not broken, then we know that neither of
$\FS$, $\fsreplacement{\FS}{n}{y}$ or $\fsreplacement{\FS}{n}{w}$ is broken,
and therefore $\cxwnv$ does not break $\FS$.
Conversely, if $\cxwnv$ does not break $\FS$, then 
we know that $\FS$ and $\fsreplacement{\FS}{n}{w}$ are not broken.
Also, as
either $[x]\neq[\empt]$ or $[w]\neq[\empt]$,
there must be a directory at $\FS(\parentf{n})$,
and so $\fsreplacement{\FS}{n}{y}$ cannot be broken, either,
as any value is permissible at $n$.
This is equivalent to saying that $\cxynv\cc \czwnv$ does not break the filesystem.
\end{proof}

\cref{ax_directchild_breaks}

\begin{proof}
If neither of the commands is an assertion command, then the claim follows from
\cref{simple_distant_pairs} in the following way.
We use an inverse proof and assume that $\cxypnv\cc\czwnv$ is not a construction pair,
none of the commands is an assertion command, and $\wrksx{\cxypnv\cc\czwnv}$.
Therefore \cref{simple_distant_pairs} applies to this sequence, and the two commands
must form a construction or destruction pair. This is a contradiction as due to
the relationship between the nodes, they cannot form a destruction pair, and
they do not form a construction pair by assumption.

Otherwise, let $\FS_1$ be $\cxypnv\aFS$, and $\FS_2$ be $(\cxypnv\cc\czwnv)\aFS$.
If $\cxypnv=\cbba{\pn}$ (the only assertion command it can be given the conditions),
then either $\FS_1$ is broken, or
$\FS(\pn)=\FS_1(\pn)=\FS_2(\pn)=\empt$.
Since $\czwnv\neq\cbba{n}$, we know that either $\czwnv$ breaks $\FS_1$,
or, depending on $z$ and $w$, either $\FS_1(n)\neq\empt$ or $\FS_2(n)\neq\empt$.
In any case $\FS_2$ will be broken at $\pn$ which is empty but at one point has a non-empty child.

Finally, if $\czwnv=\cdda{n}$, we can proceed in a similar way,
as since $\cxypnv\neq\cdda{n}$, we know at one point the value at $\pn$ is not a directory.
\end{proof}

\cref{ax_directparent_breaks}

\begin{proof}
This proof follows the same logic as that of \cref{ax_directchild_breaks},
using destruction pairs instead of construction pairs.
\end{proof}

\cref{ax_distantrel_breaks}

\begin{proof}
Suppose $\cxynnv$ and $\czwnv$ satisfy the conditions,
and there is an $\FS$ for which $(\cxynnv\cc\czwnv)\aFS\neq\fsbroken$.
If $\czwnv\neq\cbba{n}$, then either $\FS(n)\neq\empty$ or $\czwnv\aFS(n)\neq\empty$,
and therefore $\FS(\parentf{n})$ must be a directory for the filesystem to satisfy
the tree property at all times.
However, from $\cxynnv\neq\cdda{\nn}$ we also know that either $\FS(\nn)$ 
(before applying $\cxynv$) or $\cxynv\aFS(\nn)$ (after applying $\cxynv$)
is not a directory. Depending on which is true, we have a point in the sequence
of applying the commands where the filesystem cannot satify the tree property
and therefore becomes broken as $\parentf{n}$, a descendant of $\nn$, contains
a directory and so must all its ancestors.
\end{proof}

\cref{ax_child_assert}

\begin{proof}
Since $\cxynnv\neq\cdda{\nn}$, we know that
either $\cxynnv\aFS$ is broken, or, depending on $x$ and $y$,
either $\FS(\nn)\neq\vald$, or $\cxynnv\aFS(\nn)\neq\vald$.
In the latter two cases $\FS$ must be empty at all descendants of $\nn$
for it to satisfy the tree property.
The assertion command $\cbba{n}$ will therefore not break $\FS$.
\end{proof}

\cref{ax_parent_assert}

\begin{proof}
This proof follows the same logic as that of \cref{ax_child_assert}.
\end{proof}

\cref{ax_assert}

\begin{proof}
As for an assertion command $\cxynv$, $x=y$, we see that it either
breaks a filesystem, or leaves it in the same state.
\end{proof}
